
\section{Exkurs: Der Klassen-Generator \texttt{EP}}
Die meisten Parser haben die Aufgabe, einen Parse-Baum zu erstellen.  In diesem Parsebaum wird jeder
Knoten durch ein Objekt einer Klasse dargestellt.  Die Klassen dieser Objekte orientieren sich
typischerweise an den Symbolen der Grammatik, so dass in der Regel f\"ur jedes Symbol auch eine Klasse
existiert.  Das f\"uhrt dazu, dass wir bei der Entwicklung eines solchen Parses eine gro{\ss}e Zahl von
\textsl{Java}-Klassen definieren m\"ussen.  Das ist zwar nicht weiter schwer, aber doch mit einem
hohen Schreibaufwand verbunden.  Um hier die Arbeit abzuk\"urzen, k\"onnen wir einen
\emph{Klassen-Generator} einsetzen, die die ben\"otigten Klassen aus einer Spezifikation erzeugt.  Da
diese Spezifikation meist wesentlich k\"urzer ist als die \textsl{Java}-Klassen, k\"onnen wir dadurch 
einiges an Schreibarbeit einsparen.  In diesem Abschnitt m\"ochte ich Ihnen ein solches System
vorstellen.  Es handelt sich um das System EP, dass Sie unter der Adresse
\\[0.2cm]
\hspace*{1.3cm}
\texttt{http://www.ba-stuttgart.de/stroetmann/EP/ep.tar.gz}
\\[0.2cm]
im Netz finden.

\subsection{Installation}
Wir beschreiben zun\"achst die Installation dieses Systems.  Die nachfolgende Anleitung geht
von einem Linux oder Unix-System aus.  Wir nehmen an, dass der Klassen-Generator in
dem Ordner \texttt{Software}, der sich im Home-Verzeichnis des Benutzers befindet, installiert
werden soll.  Die Installation besteht dann aus folgenden Schritten:
\begin{enumerate}
\item Zun\"achst schieben wir die Datei ``\texttt{ep.tar.gz}'' 
      in das Verzeichnis \texttt{\symbol{126}/Software}: \\[0.2cm]
      \hspace*{1.3cm} \texttt{mv ep.tar.gz \symbol{126}/Software}
\item Nun wechseln wir in dieses Verzeichnis: \\[0.2cm]
      \hspace*{1.3cm} \texttt{cd \symbol{126}/Software}
\item Dort dekomprimieren wir die Datei \texttt{ep.tar.gz} mit dem Befehl \\[0.2cm]
      \hspace*{1.3cm} \texttt{gunzip ep.tar.gz} \\[0.2cm]
      Dabei entsteht die Datei \texttt{ep.tar}.
\item Die Datei \texttt{ep.tar} wird nun entpackt: \\[0.2cm]
      \hspace*{1.3cm} \texttt{untar xf ep.tar} \\[0.2cm]
      Durch diesen Befehl wird im Verzeichnis \texttt{Software} ein Unterverzeichnis mit dem Namen 
       \texttt{EP} angelegt.  Dieses Unterverzeichnis enth\"alt drei Verzeichnisse:
      \begin{enumerate}
      \item \texttt{src} enth\"alt die \textsl{Java}-Quellen zusammen mit den Klassen-Dateien.
            Falls die Version Ihrer \textsl{Java}-Umgebung nicht mit der Version \"ubereinstimmt,
            f\"ur die die Klassen-Dateien erzeugt wurden, m\"ussen Sie in diesem Verzeichnis mit dem Befehl
            \\[0.2cm]
            \hspace*{1.3cm}
            \texttt{javac *.java}
            \\[0.2cm]
            alle \textsl{Java}-Quelldateien neu \"ubersetzen.
      \item \texttt{doc} enth\"alt die Dokumentation.
      \item \texttt{examples} enth\"alt die Beispiel-Datei ``\texttt{expr.ep}''.
            In dieser Datei werden Klassen spezifiziert, die arithmetische Ausdr\"ucke beschreiben.
      \end{enumerate}
      Zus\"atzlich enth\"alt das Verzeichnis \texttt{EP} noch die Datei \texttt{antlr-2.7.5.jar}.
      Hierbei handelt es sich um die Version des Parser-Generator \textsc{Antlr}
      \cite{parr95antlr},
      die ich bei der Erstellung des Parsers verwendet habe.
\item Um EP verwenden zu k\"onnen, m\"ussen Sie die Datei \texttt{antlr-2.7.5.jar} und das
      Verzeichnis \texttt{src} in den \texttt{CLASSPATH} aufnehmen.  Das geht in einem
      \textsl{Unix}-basiertem Betriebssystem beispielsweise,
      indem Sie in der Datei \texttt{\symbol{126}/.bashrc} folgende Zeile hinzuf\"ugen:
      \\[0.2cm]
      \texttt{export
        CLASSPATH=\symbol{36}CLASSPATH:\symbol{126}/Software/EP/antlr-2.7.5.jar:\symbol{126}/Software/EP/src} 
\item Um die Installation zu testen, wechseln Sie in das Verzeichnis ``\texttt{examples}'' and
      starten dort EP, indem Sie das Kommando 
      \\[0.2cm]
      \hspace*{1.3cm} \texttt{java EP expr} 
      \\[0.2cm]
      ausf\"uhren.  
      Dieses Kommando sollte eine Reihe von \textsl{Java}-Dateien erzeugen, die dann durch den Befehl
      \\[0.2cm]
      \hspace*{1.3cm} \texttt{javac *.java} \\[0.2cm]
      \"ubersetzt werden k\"onnen.
      Anschlie{\ss}end k\"onnen wir das Kommando \\[0.2cm]
      \hspace*{1.3cm} \texttt{java Main} \\[0.2cm]
      ausf\"uhren.  Dieses Kommando sollte das folgende Ergebnis produzieren: \\[0.2cm]
      \hspace*{1.3cm} \texttt{Product(Variable(x), Variable(x))/dx = } \\
      \hspace*{2.3cm} \texttt{Sum(Product(Number(1.0), Variable(x)),} \\
      \hspace*{3.04cm}     \texttt{Product(Variable(x), Number(1.0)))}. 
\end{enumerate}

\section{Eine Beispiel-Datei}
Abbildung \ref{fig:expr.ep} zeigt eine EP-Spezifikation f\"ur Klassen, die arithmetische Ausdr\"ucke
spezifizieren.   Die eigentliche Spezifikation der Klassen findet sich in den Zeilen 1 -- 6.
Hier wird eine abstrakte Klasse \texttt{Expr} definiert, von der insgesamt 6 Klassen abgeleitet
werden.  Als ein Beispiel betrachten wir die Klasse \texttt{Sum}.  Der Ausdruck
\\[0.2cm]
\hspace*{1.3cm}
\texttt{Sum(Expr lhs, Expr rhs)}
\\[0.2cm]
spezifiziert, dass die Klasse \texttt{Sum} zwei Member-Variablen enth\"alt, die beide den Typ
\texttt{Expr} haben.  Weiterhin werden die Namen dieser Member-Variablen definiert.  Der Name der
ersten Member-Variable ist sp\"ater \texttt{mLhs}, die zweite Member-Variable hei{\ss}t \texttt{mRhs}.
Dem in der Spezifikation angegebenen Namen wird also ein ``\texttt{m}'' vorangestellt und der darauf
folgende Buchstabe wird gro{\ss} geschrieben.

\begin{figure}[thb]
  \centering
\begin{Verbatim}[ frame         = lines, 
                  framesep      = 0.3cm, 
                  labelposition = bottomline,
                  numbers       = left,
                  numbersep     = -0.2cm,
                  xleftmargin   = 0.0cm,
                  xrightmargin  = 0.0cm,
                ]
    Expr = Number(Double value)
         + Variable(String name)
         + Sum(Expr lhs, Expr rhs) 
         + Difference(Expr lhs, Expr rhs)
         + Product(Expr lhs, Expr rhs)     
         + Quotient(Expr lhs, Expr rhs);
    
    diff: Expr * String -> Expr;

    Number(value).diff(x) = Number(0.0);

    v.equals(x) -> Variable(v).diff(x) = Number(1.0);
    Variable(v).diff(x) = Number(0.0);

    Sum(l, r).diff(x) = Sum(l.diff(x), r.diff(x));
    Difference(l, r).diff(x) = Difference(l.diff(x), r.diff(x));
    Product(l, r).diff(x) = 
        Sum(Product(l.diff(x), r), Product(l, r.diff(x)));
    Quotient(l, r).diff(x) = 
        Quotient( Difference(Product(l.diff(x), r), Product(l, r.diff(x))), 
                  Product(r, r) );
\end{Verbatim} 
\vspace*{-0.3cm}
  \caption{Die Spezifikation arithmetischer Ausdr\"ucke.}
  \label{fig:expr.ep}
\end{figure} %\$

\begin{figure}[thb]
  \centering
\begin{Verbatim}[ frame         = lines, 
                  framesep      = 0.3cm, 
                  labelposition = bottomline,
                  numbers       = left,
                  numbersep     = -0.2cm,
                  xleftmargin   = 0.0cm,
                  xrightmargin  = 0.0cm,
                ]
    public class Sum extends Expr {
        private Expr mLhs;
        private Expr mRhs;
    
        public Sum(Expr lhs, Expr rhs) {
            mLhs = lhs;
            mRhs = rhs;
        }
        public Expr diff(String x) {
            return new Sum(mLhs.diff(x), mRhs.diff(x));
        }
        public Boolean equals(Expr rhs) {
            if (!(rhs instanceof Sum)) {
                return false;
            }
            Sum r = (Sum) rhs;
            if(!mLhs.equals(r.mLhs)) {
                return false;
            }
            if(!mRhs.equals(r.mRhs)) {
                return false;
            }
            return true;
        }
        public Expr getLhs() {
            return mLhs;
        }
        public Expr getRhs() {
            return mRhs;
        }
        public String toString() {
            return "Sum(" + mLhs.toString() + ", " + mRhs.toString() + ")";
        }
    }
\end{Verbatim} 
\vspace*{-0.3cm}
  \caption{Die Klasse \texttt{Sum}.}
  \label{fig:Sum.java.ep}
\end{figure} %\$

Neben der Spezifikation der Klassen enth\"alt die in Abbildung \ref{fig:expr.ep} gezeigte Datei noch
die Spezifikation der Signatur einer Methode: In Zeile 8 wird die Methode \texttt{diff}
spezifiziert.  Das implizite Argument dieser Methode ist vom Typ \texttt{Expr}, denn es handelt sich
ja um eine Methode der Klasse \texttt{Expr}, das erste Argument ist vom Type \texttt{String} und das Ergebnis
ist vom Type \texttt{Expr}.  Die nachfolgenden bedingten Gleichungen zeigen, wie die Methode
$\textsl{diff}()$, die das symbolische Differenzieren beschreibt, spezifiziert werden kann.   Aus
dieser Spezifikation erzeugt \textsc{EP} nun die in Zeile 1 -- 6 spezifizierten Klassen.  Wir
betrachten exemplarisch die Klasse \texttt{Sum}, die in Abbildung \ref{fig:Sum.java.ep} gezeigt ist.
Wir diskutieren den erzeugten \textsl{Java}-Code jetzt im Detail.
\begin{enumerate}
\item In Zeile 1 sehen wir, dass die Klasse \texttt{Sum} von der Klasse \texttt{Expr}
      abgeleitet wird.
\item In Zeile 2 und 3 werden die Member-Variablen \texttt{mLhs} und \texttt{mRhs} definiert.
\item In Zeile 5 -- 8 finden wir einen Konstruktor, der die Member-Variablen initialisiert.
\item In Zeile 9 -- 11 finden wir die Implementierung der Methode $\textsl{diff}$.  Diese Zeile
      setzt die Zeile 15 aus der Abbildung \ref{fig:expr.ep} um.
\item Des weiteren enth\"alt die Klasse \texttt{Sum} noch eine Methode $\texttt{equals}()$, die
      automatisch erzeugt wird.
\item Au{\ss}erdem werden noch Getter-Methoden f\"ur die Member-Variablen in den Zeilen 25 -- 30
      erzeugt.
\item Schlie{\ss}lich zeigt Zeile 31 -- 33 eine automatisch erzeugte \texttt{toString}-Methode.
\end{enumerate}
Um die von \textsc{EP} erzeugten Klassen zu testen, ben\"otigen wir noch eine Methode
$\texttt{main}()$.  Abbildung \ref{fig:Main.java} zeigt die Klasse \texttt{Main}, die eine
solche Methode enth\"alt.  \"Ubersetzen wir diese Klasse und f\"uhren wir das erzeugte Programm
aus, so erhalten wir die folgende Ausgabe:
\\[0.2cm]
\hspace*{1.3cm}
\texttt{Sum(Product(Number(1.0), Variable(x)), Product(Variable(x), Number(1.0)))}
\\[0.2cm]


\begin{figure}[!ht]
\centering
\begin{Verbatim}[ frame         = lines, 
                  framesep      = 0.3cm, 
                  labelposition = bottomline,
                  numbers       = left,
                  numbersep     = -0.2cm,
                  xleftmargin   = 0.8cm,
                  xrightmargin  = 0.8cm,
                ]
    public class Main {
        public static void main(String[] args) {
            Expr x      = new Variable("x");
            Expr square = new Product(x, x);
            System.out.println(square.diff("x"));
        }
    }
\end{Verbatim}
\vspace*{-0.3cm}
\caption{Eine Klasse zum Testen der von EP erzeugten Klassen.}
\label{fig:Main.java}
\end{figure}


\begin{figure}[!ht]
\centering
\begin{Verbatim}[ frame         = lines, 
                  framesep      = 0.3cm, 
                  labelposition = bottomline,
                  numbers       = left,
                  numbersep     = -0.2cm,
                  xleftmargin   = 0.0cm,
                  xrightmargin  = 0.0cm,
                ]
    import java_cup.runtime.*;
          
    %%
       
    %char
    %line
    %column
    %cup
       
    %{   
        private Symbol symbol(int type) {
            return new Symbol(type, yychar, yychar + yylength());
        }
        
        private Symbol symbol(int type, Object value) {
            return new Symbol(type, yychar, yychar + yylength(), value);
        }
    %}
       
    %%
       
    "+"          { return symbol( sym.PLUS   ); }
    "-"          { return symbol( sym.MINUS  ); }
    "*"          { return symbol( sym.TIMES  ); }
    "/"          { return symbol( sym.DIVIDE ); }
    "("          { return symbol( sym.LPAREN ); }
    ")"          { return symbol( sym.RPAREN ); }
    
    [1-9][0-9]*  { return symbol(sym.NUMBER,     new Integer( yytext())); }
    [a-zA-Z]+    { return symbol(sym.IDENTIFIER, new Variable(yytext())); }
    [ \t\n]      { /* skip white space */ }   
\end{Verbatim}
\vspace*{-0.3cm}
\caption{\textsl{JFlex}-Spezifikation des Scanners}
\label{fig:differentiator.jflex}
\end{figure}
\noindent
Wir zeigen nun, wie wir alle  drei Werkzeuge
\textsl{JFlex}, \textsl{JavaCup} und \textsc{Ep} gemeinsam verwenden k\"onnen um ein
Programm zum symbolischen Differenzieren erstellen zu k\"onnen.  
\begin{enumerate}
\item Abbildung \ref{fig:differentiator.jflex} zeigt die Spezifikation des Scanners.
      Gegen\"uber dem in Abbildung \ref{fig:calc.jflex} gezeigten Scanner kommt hier die Zeile 30
      hinzu, in der Variablen erkannt werden.
\item Abbildung \ref{fig:differentiator.cup} zeigt die Spezifikation der Grammatik.
      Der wesentliche Unterschied zu der in Abbildung \ref{fig:calc.cup} gezeigten Grammatik besteht
      darin, dass mit der syntaktische Variable \textsl{expr} nun nicht mehr ein Objekt vom Typ
      \texttt{Integer} assoziiert ist, sondern stattdessen ein Objekt der Klasse \texttt{Expr}.
      Dieses Objekt repr\"asentiert einen Syntax-Baum.
\item Abbildung \ref{fig:expr.ep} aus dem letzten Abschnitt zeigt die Definition der abstrakten Klasse
      \texttt{Expr} und der davon abgeleiteten Klassen mit Hilfe einer \textsc{Ep}-Spezifikation.
\item Abbildung \ref{fig:Differentiator.java} zeigt die Definition der Klasse
      Differentiator, in der die Methode $\textsl{main}()$ enthalten ist.
  
      Hier muss die Zeile 7 n\"aher erl\"autert werden:
      Die Methode $\textsl{parse}()$ gibt ein Objekt vom Typ \texttt{Symbol} zur\"uck.  Die
      Klasse \texttt{Symbol} ist in dem Paket 
      \texttt{java\_cup.runtime} definiert und enth\"alt eine als \texttt{public} deklarierte
      Member-Variable \texttt{value}.  Diese Variable enth\"alt den Wert, welcher dem von der Methode
      $\textsl{parse}()$ erkannten Nichtterminal \textsl{expr} zugeordnet ist.  Dieser Wert ist ein
      Element der Klasse \texttt{Expr}, der in der Klasse \texttt{Symbol} aber als \texttt{Object} 
      deklariert ist.  Daher m\"ussen wir diesen Wert erst durch den Cast in Zeile 8 in den Typ 
      \texttt{Expr} \"uberf\"uhren, damit wird dann in Zeile 9 die Methode $\textsl{diff}()$  Klasse
      \texttt{Expr} aufrufen d\"urfen.
\end{enumerate}


\begin{figure}[!ht]
\centering
\begin{Verbatim}[ frame         = lines, 
                  framesep      = 0.3cm, 
                  labelposition = bottomline,
                  numbers       = left,
                  numbersep     = -0.2cm,
                  xleftmargin   = 0.0cm,
                  xrightmargin  = 0.0cm,
                ]
    import java_cup.runtime.*;
    
    terminal           PLUS, MINUS, TIMES, DIVIDE, LPAREN, RPAREN;
    terminal Integer   NUMBER;
    terminal Variable  IDENTIFIER;
    
    nonterminal Expr   expr;
    
    precedence left    PLUS, MINUS;
    precedence left    TIMES, DIVIDE;
    
    /* The grammar */
    expr    ::= expr:e1 PLUS   expr:e2 {: RESULT = new Sum(        e1, e2 ); :} 
             |  expr:e1 MINUS  expr:e2 {: RESULT = new Difference( e1, e2 ); :}
             |  expr:e1 TIMES expr:e2  {: RESULT = new Product(    e1, e2 ); :}
             |  expr:e1 DIVIDE expr:e2 {: RESULT = new Quotient(   e1, e2 ); :}
             |  NUMBER:n               {: RESULT = new Number(     n      ); :} 
             |  IDENTIFIER:v           {: RESULT = v;                        :} 
             |  LPAREN expr:e RPAREN   {: RESULT = e;                        :}
             ;
\end{Verbatim}
\vspace*{-0.3cm}
\caption{Grammatik f\"ur symbolische Ausdr\"ucke}
\label{fig:differentiator.cup}
\end{figure}


\begin{figure}[!ht]
\centering
\begin{Verbatim}[ frame         = lines, 
                  framesep      = 0.3cm, 
                  labelposition = bottomline,
                  numbers       = left,
                  numbersep     = -0.2cm,
                  xleftmargin   = 0.8cm,
                  xrightmargin  = 0.8cm,
                ]
    import java_cup.runtime.*;
    
    public class Differentiator {
        public static void main(String[] args) {
            try { 
                parser p = new parser(new Yylex(System.in)); 
                Symbol s = p.parse(); 
                Expr   e = (Expr) s.value;
                Expr   d = e.diff("x");
                System.out.println("d/dx " + e + " = " + d);
            } catch (Exception e) {}
        }
    }
\end{Verbatim}
\vspace*{-0.3cm}
\caption{Die Klasse \texttt{Differentiator}}
\label{fig:Differentiator.java}
\end{figure}

%%% Local Variables: 
%%% mode: latex
%%% TeX-master: "formale-sprachen"
%%% End: 
