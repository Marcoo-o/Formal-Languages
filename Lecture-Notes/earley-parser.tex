\chapter{Earley-Parser}
In diesem Kapitel stellen wir ein effizientes Verfahren vor, mit dem es m\"oglich ist, f\"ur eine
\underline{beliebi}g\underline{e} vorgegebene kontextfreie Grammatik
\\[0.2cm]
\hspace*{1.3cm}
$G = \langle V, \Sigma, R, S \rangle$ \quad und einen vorgegebenen String $s \in \Sigma^*$
\\[0.2cm]
zu entscheiden, ob $s$ ein Element der Sprache $L(G)$ ist, ob also $s \in L(G)$ gilt.
Der Algorithmus, den wir gleich diskutieren werden, wurde 1970 von Jay Earley publiziert
\cite{earley:70}.  Neben dem Algorithmus von Earley gibt es noch den
\blue{Cocke-Younger-Kasami-Algorithmus}, \index{Cocke-Younger-Kasami-Algorithmus} in der Literatur auch als
\blue{\textsc{Cyk}-Algorithmus} bekannt, \index{\textsc{Cyk}-Algorithmus} der unabh\"angig von John Cocke
\cite{cocke:1970}, Daniel H.~Younger \cite{younger:1967} und Tadao Kasami \cite{kasami:1965} entdeckt wurde.
Der \textsc{Cyk}-Algorithmus ist nur anwendbar, wenn
die Grammatik in \blue{Chomsky-Normalform} vorliegt.  Da es sehr aufwendig ist, eine Grammatik in
Chomsky-Normalform zu transformieren, wird der \textsc{Cyk}-Algorithmus in der
Praxis nicht eingesetzt.  
Demgegen\"uber kann der von Earley angegebene Algorithmus auf beliebige kontextfreie Grammatiken
angewendet werden.  Im allgemeinen Fall hat dieser Algorithmus die Komplexit\"at
$\mathcal{O}(n^3)$, aber falls die vorgegebene Grammatik eindeutig ist, dann ist die Komplexit\"at
lediglich $\mathcal{O}(n^2)$.  Geschickte Implementierungen von Earley's Algorithmus
erreichen f\"ur viele praktisch relevante Grammatiken sogar eine lineare Laufzeit.  
Im Gegensatz hat der \textsc{Cyk}-Algorithmus \textbf{immer} die Komplexit\"at $\mathcal{O}(n^3)$.
Earley's Algorithmus hat sowohl f\"ur $LL(k)$-Grammatiken als auch f\"ur $LR(1)$-Grammatiken, die wir in einem
sp\"ateren Kapitel analysieren werden, nur eine lineare Laufzeit.

Dieses Kapitel gliedert sich in die folgenden Abschnitte:
\begin{enumerate}[(a)]
\item Zun\"achst skizzieren wir die Theorie, die Earley's Algorithmus zu Grunde liegt.
\item Danach geben wir eine einfache Implementierung des Algorithmus in \textsl{Python} an.
%\item Anschlie�end beweisen wir die Korrektheit und Vollst\"andigkeit des Algorithmus.
%\item Zum Abschluss des Kapitels untersuchen wir die Komplexit\"at.
\end{enumerate}

\section{Der Algorithmus von Earley}
Der zentrale Begriff des von Earley angegebenen Algorithmus ist der Begriff des \blue{Earley-Objekts},
das wie folgt definiert ist:

\begin{Definition}[Earley-Objekt]
  Gegeben sei eine kontextfreie Grammatik $G = \langle V, \Sigma, R, S \rangle$ und ein String
  $s = x_1x_2 \cdots x_n \in \Sigma^*$ der L\"ange $n$.  Wir bezeichnen ein Paar der Form
  \\[0.2cm]
  \hspace*{1.3cm}
  $\langle A \rightarrow \alpha \bullet \beta, k \rangle$
  \\[0.2cm]
  genau dann als ein \blue{Earley-Objekt},\index{Earley-Objekt} falls folgendes gilt:
  \begin{enumerate}[(a)]
  \item $(A \rightarrow \alpha \beta) \in R$ \quad und
  \item $k \in \{0,1,\cdots,n\}$. \qed
  \end{enumerate}
\end{Definition}

\noindent
\textbf{Erkl\"arung}: 
Ein Earley-Objekt beschreibt einen Zustand, in dem ein Parser sich befinden kann.  
Ein Earley-Parser, der einen String $x_1 \cdots x_n$ parsen soll,
verwaltet $n+1$ Mengen von Earley-Objekten.  Diese Mengen bezeichnen wir mit
\\[0.2cm]
\hspace*{1.3cm}
$Q_0, Q_1, \cdots, Q_n$.
\\[0.2cm]
Die Interpretation von 
\\[0.2cm]
\hspace*{1.3cm}
$\langle A \rightarrow \alpha \bullet \beta, k \rangle \in Q_j$ \quad mit  $j \geq k$
\\[0.2cm]
ist dann wie folgt:
\begin{enumerate}
\item Der Parser versucht, die Regel $A \rightarrow \alpha \beta$ auf den Teilstring 
      $x_{k+1} \cdots x_n$ anzuwenden und am Anfang dieses Teilstrings ein $A$ mit Hilfe
      der Regel $A \rightarrow \alpha \beta$ zu erkennen.
\item In dem Teilstrings $x_{k+1} \cdots x_j$ hat der Parser bereits $\alpha$ erkannt, es gilt also
      \\[0.2cm]
      \hspace*{1.3cm}
      $\alpha \Rightarrow^* x_{k+1} \cdots x_j$.
\item Folglich versucht der Parser, am Anfang des Teilstrings $x_{j+1} \cdots x_n$ ein
      $\beta$ zu erkennen.
\end{enumerate}
Der Algorithmus von Earley verwaltet f\"ur $j=0,1,\cdots,n$  Mengen $Q_j$ von
Earley-Objekten, die den Zustand beschreiben, in dem der Parser ist, wenn der Teilstring
$x_1 \cdots x_j$ verarbeitet ist.  Zu Beginn des Algorithmus wird der Grammatik ein neues Start-Symbol $\widehat{S}$ 
sowie die Regel $\widehat{S} \rightarrow S$ hinzugef\"ugt.  Die Menge $Q_0$ wird definiert als
\\[0.2cm]
\hspace*{1.3cm}
$Q_0 := \bigl\{ \pair(\widehat{S} \rightarrow \bullet S, 0) \bigr\}$,
\\[0.2cm]
denn der Parser soll ja das Start-Symbol $S$ am Anfang des Strings $x_1 \cdots x_n$ erkennen.
Die restlichen Mengen $Q_j$ sind f\"ur $j=1,\cdots,n$ zun\"achst leer.  Die Mengen $Q_j$ werden nun durch die
folgenden drei Operationen so lange wie m\"oglich erweitert: 
\begin{enumerate}
\item \emph{Lese-Operation}

      Falls der Zustand $Q_j$ ein Earley-Objekt der Form 
      $\pair(A \rightarrow \beta \bullet a \gamma, k)$ enth\"alt, wobei $a$ ein
      Terminal ist, so versucht der Parser, die rechte Seite der Regel
      $A \rightarrow \beta a \gamma$ zu erkennen und hat bis zur Position $j$ bereits den Teil $\beta$ erkannt.
      Folgt auf dieses $\beta$ nun, wie in der Regel $A \rightarrow \beta a \gamma$
      vorgesehen, an der Position $j+1$ das Terminal $a$,
      so muss der Parser nach der Position $j+1$ nur noch $\gamma$ erkennen.  Daher wird in diesem Fall das
      Earley-Objekt 
      \\[0.2cm]
      \hspace*{1.3cm}
      $\pair(A \rightarrow \beta a \bullet \gamma, k)$
      \\[0.2cm]
      dem Zustand $Q_{j+1}$ hinzugef\"ugt:
      \\[0.2cm]
      \hspace*{1.3cm}
      $\pair(A \rightarrow \beta \bullet a \gamma, k) \in Q_j \wedge x_{j+1} = a
       \;\Rightarrow\;
       Q_{j+1} := Q_{j+1} \cup \bigl\{ \pair(A \rightarrow \beta a \bullet \gamma, k) \bigr\}$.
\item \emph{Vorhersage-Operation}

      Falls der Zustand $Q_j$ ein Earley-Objekt der Form $\pair(A \rightarrow \beta \bullet C \delta, k)$
      enth\"alt, wobei $C$ eine syntaktische Variable ist, so versucht der Parser im Zustand
      $Q_j$ den Teilstring $C\delta$ zu erkennen.  Dazu muss der Parser an diesem Punkt ein $C$ erkennen.  
      Wir f\"ugen daher f\"ur jede Regel $C \rightarrow \gamma$ der Grammatik das Earley-Objekt 
      $\pair(C \rightarrow \bullet \gamma, j)$ zu der Menge $Q_j$ hinzu:
      \\[0.2cm]
      \hspace*{1.3cm}
      $\pair(A \rightarrow \beta \bullet C \delta, k) \in Q_j 
       \wedge (C \rightarrow \gamma) \in R 
       \;\Rightarrow\;
       Q_j := Q_j \cup\bigl\{ \pair(C \rightarrow \bullet\gamma, j)\bigr\}$.
\item \emph{Vervollst\"andigungs-Operation}

      Falls der Zustand $Q_i$ ein Earley-Objekt der Form $\pair(C \rightarrow \gamma \bullet, j)$
      enth\"alt und weiter der Zustand $Q_j$ ein Earley-Objekt der Form 
      $\pair(A \rightarrow \beta \bullet C \delta,k)$ enth\"alt, dann hat der Parser im Zustand $Q_j$
      versucht, ein $C$ zu parsen und das $C$ ist im Zustand $Q_i$ erkannt worden.  
      Daher f\"ugen wir dem Zustand
      $Q_i$ nun das Earley-Objekt $\pair(A \rightarrow \beta C \bullet \delta,k)$ hinzu:
      \\[0.2cm]
      \hspace*{1.3cm}
      $\pair(C \rightarrow \gamma \bullet, j) \in Q_i \wedge
       \pair(A \rightarrow \beta \bullet C \delta,k) \in Q_j \;\Rightarrow\;
       Q_i := Q_i \cup \bigl\{ \pair(A \rightarrow \beta C \bullet \delta,k) \bigr\}
      $.
\end{enumerate}
Der Algorithmus von Earley, der einen String der Form $s = x_1 \cdots x_n$ parsen will,  funktioniert nun so:
\begin{enumerate}
\item Wir initialisieren die Zust\"ande $Q_i$ wie folgt:
      \\[0.2cm]
      \hspace*{1.3cm}
      $Q_0 := \bigl\{ \pair(\widehat{S} \rightarrow \bullet S, 0) \bigr\}$,
      \\[0.2cm]
      \hspace*{1.3cm}
      $Q_i := \bigl\{ \bigr\}$ \quad f\"ur $i=1,\cdots,n$.
\item Anschlie{\ss}end lassen wir in einer Schleife $i$ von $0$ bis $n$ laufen und f\"uhren die folgenden 
      Schritte durch:
      \begin{enumerate}[(a)]
      \item Wir vergr\"o{\ss}ern $Q_i$ mit der Vervollst\"andigungs-Operation so lange, bis mit dieser Operation
            keine neuen Earley-Objekte mehr gefunden werden k\"onnen.
      \item Anschlie{\ss}end vergr\"o{\ss}ern wir $Q_i$ mit Hilfe der Vorhersage-Operation.  Diese Operation
            wird ebenfalls so lange durchgef\"uhrt, wie neue Earley-Objekte gefunden werden.
      \item Falls $i < n$ ist, wenden wir die Lese-Operation auf $Q_i$ an und initialisierend damit
            $Q_{i+1}$. 
      \end{enumerate}
      Falls die betrachtete Grammatik $G$ auch $\varepsilon$-Regeln enth\"alt, also Regeln der Form
      \\[0.2cm]
      \hspace*{1.3cm}
      $C \rightarrow \varepsilon$,
      \\[0.2cm]
      dann kann es passieren, dass durch die Anwendung einer Vorhersage-Operation eine neue Anwendung
      der Vervollst\"andigungs-Operation m\"oglich wird.  In diesem Fall m\"ussen Vorhersage-Operation und
      Vervollst\"andigungs-Operation so lange iteriert werden, bis durch Anwendung dieser beiden
      Operationen keine neuen Earley-Objekte mehr erzeugt werden k\"onnen.
\item Falls nach Beendigung des Algorithmus die Menge $Q_n$ das Earley-Objekt 
      $\pair(\widehat{S} \rightarrow S \bullet,0)$ enth\"alt, dann war das Parsen erfolgreich und 
      der String $x_1 \cdots x_n$ liegt in der von der Grammatik erzeugten Sprache.
\end{enumerate}
  
\example
Abbildung \ref{fig:expr-small} zeigt eine vereinfachte Grammatik f\"ur arithmetische
Ausdr\"ucke, die nur aus den Zahlen ``1'', ``2'' und ``3'' und den beiden Operator-Symbolen
``\texttt{+}'' und ``\texttt{*}'' aufgebaut ist.  Die Menge $T$ der Terminale dieser
Grammatik ist also durch
\\[0.2cm]
\hspace*{1.3cm}
 $T = \{ \quoted{1}, \quoted{2}, \quoted{3}, \quoted{+}, \quoted{*} \}$
\\[0.2cm]
gegeben.
Wir zeigen, wie sich der String
``\texttt{1+2*3}'' mit dieser Grammatik und dem Algorithmus von Earley parsen l\"asst.
In der folgenden Darstellung werden wir die syntaktische Variable \texttt{expr} mit
dem Buchstaben $E$ abk\"urzen, f\"ur \texttt{prod} schreiben wir $P$ und f\"ur \texttt{fact} verwenden wir die
Abk\"urzung $F$.


\begin{figure}[!ht]
\centering
\begin{Verbatim}[ frame         = lines, 
                  framesep      = 0.3cm, 
                  labelposition = bottomline,
                  numbers       = left,
                  numbersep     = -0.2cm,
                  xleftmargin   = 0.8cm,
                  xrightmargin  = 0.8cm,
                ]
    expr : expr '+' prod
         | prod
         ;
    
    prod : prod '*' fact
         | fact
         ;
    
    fact : '1'
         | '2'
         | '3'
         ;
\end{Verbatim}
\vspace*{-0.3cm}
  \caption{Eine vereinfachte Grammatik f\"ur arithmetische Ausdr\"ucke.}
  \label{fig:expr-small} 
\end{figure}


\begin{enumerate}
\item Wir initialisieren $Q_0$ als
      \\[0.2cm]
      \hspace*{1.3cm}
      $Q_0 = \{ \pair(\widehat{S} \rightarrow \bullet\, E, 0) \}$. 
      \\[0.2cm]
      Die Mengen $Q_1$, $Q_2$, $Q_3$, $Q_4$ und $Q_5$ sind zun\"achst alle leer.
      Wenden wir die Vervollst\"andigungs-Operation auf $Q_0$ an, so finden wir keine neuen
      Earley-Objekte.

      Anschlie{\ss}end wenden wir die Vorhersage-Operation auf das Earley-Objekt 
      $\pair(\widehat{S} \rightarrow \bullet\, E, 0)$ an.  Dadurch werden der Menge $Q_0$ 
      zun\"achst die beiden Earley-Objekte 
      \\[0.2cm]
      \hspace*{1.3cm}
      $\pair(E \rightarrow \bullet\; E \squoted{+} P, 0)$ 
      \quad und \quad
      $\pair(E \rightarrow \bullet\; P, 0)$ 
      \\[0.2cm]
      hinzugef\"ugt.  Auf das Earley-Objekt $\pair(E \rightarrow \bullet\, P, 0)$ 
      k\"onnen wir die Vorhersage-Operation ein weiteres Mal anwenden und erhalten dann die beiden
      neuen Earley-Objekte
      \\[0.2cm]
      \hspace*{1.3cm}
      $\pair(P \rightarrow \bullet\; P \squoted{*} F, 0)$ 
      \quad und\quad 
      $\pair(P \rightarrow \bullet\; F, 0)$. 
      \\[0.2cm]
      Wenden wir auf das Earley-Objekt $\pair(P \rightarrow \bullet\; F, 0)$
      die Vorhersage-Operation an, so erhalten wir schie{\ss}lich noch die folgenden Earley-Objekte in $Q_0$:
      \\[0.2cm]
      \hspace*{1.3cm}
      $\pair(F \rightarrow \bullet \squoted{1}, 0)$, \quad 
      $\pair(F \rightarrow \bullet \squoted{2}, 0)$, \quad und \quad
      $\pair(F \rightarrow \bullet \squoted{3}, 0)$. 
      \\[0.2cm]
      Insgesamt enth\"alt $Q_0$ nun die folgenden Earley-Objekte:
      \begin{enumerate}
      \item $\pair(\widehat{S} \rightarrow \bullet\; E, 0)$,
      \item $\pair(E \rightarrow \bullet\; E \squoted{+} P, 0)$
      \item $\pair(E \rightarrow \bullet\; P, 0)$,
      \item $\pair(P \rightarrow \bullet\; P \squoted{*} F, 0)$,
      \item $\pair(P \rightarrow \bullet\; F, 0)$,
      \item $\pair(F \rightarrow \bullet \squoted{1}, 0)$,
      \item $\pair(F \rightarrow \bullet \squoted{2}, 0)$,
      \item $\pair(F \rightarrow \bullet \squoted{3}, 0)$.
      \end{enumerate}

      Jetzt wenden wir die Lese-Operation auf $Q_0$ an.  Da das erste Zeichen des zu parsenden Strings eine
      ``1'' ist, hat die Menge  $Q_1$ danach die folgende Form:
      \\[0.2cm]
      \hspace*{1.3cm}
      $Q_1 = \bigl\{ \pair(F \rightarrow \squoted{1} \bullet, 0) \bigr\}$.
\item Nun setzen wir $i= 1$ und wenden zun\"achst auf $Q_1$ die Vervollst\"andigungs-Operation an.
      Aufgrund des Earley-Objekts $\pair(F \rightarrow \squoted{1} \bullet, 0) $ in $Q_1$
      suchen wir in $Q_0$ ein Earley-Objekt, bei dem die Markierung ``$\bullet$'' vor der Variablen
      $F$ steht.  Wir finden das Earley-Objekt
      $\pair(P \rightarrow \bullet\; F, 0)$.  Daher f\"ugen wir nun
      $Q_1$ das Earley-Objekt
      \\[0.2cm]
      \hspace*{1.3cm}
      $\pair(P \rightarrow F \;\bullet, 0)$ 
      \\[0.2cm]
      hinzu.  Hierauf k\"onnen wir wieder die
      Vervollst\"andigungs-Operation anwenden und finden (nach mehrmaliger Anwendung) f\"ur $Q_1$ insgesamt die folgenden Earley-Objekte:
      \begin{enumerate}
      \item $\pair(P \rightarrow F \;\bullet, 0)$, 
      \item $\pair(P \rightarrow  P\;\bullet \squoted{*} F, 0)$, 
      \item $\pair(E \rightarrow P \; \bullet, 0)$,
      \item $\pair(E \rightarrow E\;\bullet \squoted{+} P, 0)$,
      \item $\pair(\widehat{S} \rightarrow  E\;\bullet, 0)$.
      \end{enumerate}
      
      Als n\"achstes wenden wir auf diese Earley-Objekte die Vorhersage-Operation an.  Da das
      Markierungs-Zeichen ``$\bullet$'' aber in keinem der in $Q_i$ auftretenden Earley-Objekte vor einer 
      Variablen steht, ergeben sich hierbei keine neuen Earley-Objekte.

      Als letztes wenden wir die Lese-Operation auf $Q_1$ an.  Da in dem String
      ``\texttt{1+2*3}'' das Zeichen ``\texttt{+}'' an der Position 2 liegt ist und $Q_1$ das Earley-Objekt 
      \\[0.2cm]
      \hspace*{1.3cm}
      $\pair(E \rightarrow E\;\bullet \squoted{+} P, 0)$
      \\[0.2cm]
      enth\"alt, f\"ugen wir in $Q_2$  das Earley-Objekt
      \\[0.2cm]
      \hspace*{1.3cm}
      $\pair(E \rightarrow E \squoted{+}\bullet\; P, 0)$
      \\[0.2cm]
      ein.
\item Nun setzen wir $i= 2$ und wenden zun\"achst auf $Q_2$ die Vervollst\"andigungs-Operation
      an.  Zu diesem Zeitpunkt gilt
      \\[0.2cm]
      \hspace*{1.3cm}
      $Q_2 = \{ \pair(E \rightarrow E \squoted{+}\bullet\; P, 0) \}$.
      \\[0.2cm]
      Da in dem einzigen Earley-Objekt, das hier auftritt, das Markierungs-Zeichen ``$\bullet$''
      nicht am Ende der Grammatik-Regel steht, finden wir durch die Vervollst\"andigungs-Operation in
      diesem Schritt keine       neuen Earley-Objekte. 

      Als n\"achstes wenden wir auf $Q_2$ die Vorhersage-Operation an.  Da das Markierungs-Zeichen
      vor der Variablen $P$ steht, finden wir zun\"achst die beiden Earley-Objekte
      \\[0.2cm]
      \hspace*{1.3cm}
      $\pair(P \rightarrow \bullet\; F, 2)$ \quad und \quad
      $\pair(P \rightarrow \bullet\; P \squoted{*} F, 2)$.
      \\[0.2cm]
      Da in dem ersten Earley-Objekt das Markierungs-Zeichen vor der Variablen $F$ steht, kann 
      die Vorhersage-Operation ein weiteres 
      Mal angewendet werden und wir finden noch die folgenden Earley-Objekte:
      \begin{enumerate}
      \item $\pair(F \rightarrow \bullet \squoted{1}, 2)$, 
      \item $\pair(F \rightarrow \bullet \squoted{2}, 2)$,
      \item $\pair(F \rightarrow \bullet \squoted{3}, 2)$.
      \end{enumerate}

      Als letztes wenden wir die Lese-Operation auf $Q_2$ an.  Da das dritte Zeichen in dem zu lesenden
      String ``\texttt{1+2*3}'' die Ziffer ``2'' ist, hat $Q_3$ nun die Form
      \\[0.2cm]
      \hspace*{1.3cm}
      $Q_3 = \{ \pair(F \rightarrow \squoted{2}\bullet, 2) \}$.
\item Wir setzen $i = 3$ und wenden auf $Q_3$ die Vervollst\"andigungs-Operation an.
      Dadurch f\"ugen wir 
      \\[0.2cm]
      \hspace*{1.3cm}
      $\pair(P \rightarrow F \;\bullet, 2)$ 
      \\[0.2cm]
      in $Q_3$ ein.  Hier k\"onnen wir ein weiteres Mal die Vervollst\"andigungs-Operation anwenden. 
      Durch iterierte Anwendung der Vervollst\"andigungs-Operation erhalten wir zus\"atzlich die folgenden 
      Earley-Objekte:
      \begin{enumerate}
      \item $\pair(P \rightarrow P \bullet \squoted{*} F, 2)$,
      \item $\pair(E \rightarrow E \squoted{+} P\;\bullet, 0)$,
      \item $\pair(E \rightarrow E\;\bullet \squoted{+} P, 0)$
      \item $\pair(\widehat{S} \rightarrow E\;\bullet, 0)$.
      \end{enumerate}
      Als letztes wenden wir die Lese-Operation an.  Da der n\"achste zu lesende Buchstabe das Zeichen
      ``\texttt{*}'' ist, erhalten wir
      \\[0.2cm]
      \hspace*{1.3cm}
      $Q_4 = \{ \pair(P \rightarrow P \squoted{*} \bullet\, F, 2) \}$.
\item Wir setzen $i= 4$.  Die Vervollst\"andigungs-Operation liefert keine neuen Earley-Objekte.
      Die Vorhersage-Operation liefert folgende Earley-Objekte:
      \begin{enumerate}
      \item $\pair(F \rightarrow \bullet \squoted{1}, 4)$, 
      \item $\pair(F \rightarrow \bullet \squoted{2}, 4)$,
      \item $\pair(F \rightarrow \bullet \squoted{3}, 4)$.
      \end{enumerate}
      Da das n\"achste Zeichen die Ziffer ``3'' ist, liefert die Lese-Operation f\"ur $Q_5$:
      \\[0.2cm]
      \hspace*{1.3cm}
      $Q_5 = \pair(F \rightarrow \squoted{3}\bullet, 4) \}$.
\item Wir setzen $i=5$.  Die Vervollst\"andigungs-Operation liefert nacheinander die folgenden
      Earley-Objekte:
      \begin{enumerate}
      \item $\pair(P \rightarrow P \squoted{*} F\;\bullet, 2)$,
      \item $\pair(E \rightarrow E \squoted{+} P\;\bullet, 0)$,
      \item $\pair(P \rightarrow  P \;\bullet\squoted{*} F, 2)$,
      \item $\pair(E \rightarrow E \;\bullet\squoted{+} P, 0)$ 
      \item $\pair(\widehat{S} \rightarrow  E\;\bullet, 0)$.
      \end{enumerate}
\end{enumerate}
Da die Menge $Q_5$ das Earley-Objekt $\pair(\widehat{S} \rightarrow  E\;\bullet, 0)$ enth\"alt,
k\"onnen wir schlie{\ss}en, dass der String ``\texttt{1+2*3}'' tats\"achlich in der von der Grammatik erzeugten
Sprache liegt.
\vspace*{0.3cm}

\exercise
Zeigen Sie, dass der String ``\texttt{1*2+3}'' in der Sprache der Grammatik liegt, die in 
Abbildung \ref{fig:expr-small} gezeigt wird.  Benutzen Sie dazu den von Earley
angegebenen Algorithmus.

\section{Implementing Earley's Algorithm in \textsl{Python}}
The \textsl{Jupyter} notebook
\\[0.2cm]
\hspace*{-0.3cm}
\href{https://github.com/karlstroetmann/Formal-Languages/blob/master/ANTLR4-Python/Earley-Parser/Earley-Parser.ipynb}{https://github.com/karlstroetmann/Formal-Languages/blob/master/ANTLR4-Python/Earley-Parser/Earley-Parser.ipynb}
\\[0.2cm]
contains an implementation of Earley's algorithm.

\section{Pr\"ufen Sie Ihr Verst\"andnis}
\begin{enumerate}[(a)]
\item Was ist ein Earley-Objekt und wie werden die Komponenten eines Earley-Objekts interpretiert?
\item Wie funktioniert die Lese-Operation beim Algorithmus von Earley?
\item Wie funktioniert die Vorhersage-Operation beim Algorithmus von Earley?
\item Wie funktioniert die Vervollst\"andigungs-Operation beim Algorithmus von Earley?
\item Skizzieren Sie den Algorithmus von Earley f\"ur den Fall, dass die Grammatik keine $\varepsilon$-Regeln enth\"alt.
\item Welche Komplexit\"at hat der Algorithmus von Earley im allgemeinen Fall?
\item Welche Komplexit\"at hat der Algorithmus von Earley in dem Fall, dass die Grammatik eindeutig ist?
\end{enumerate}


%%% Local Variables: 
%%% mode: latex
%%% TeX-master: "formal-languages"
%%% End: 
