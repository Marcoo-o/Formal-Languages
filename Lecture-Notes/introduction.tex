\chapter{Introduction and Motivation}
This lecture covers both the theory of formal languages as well as some of their applications.
In particular, we discuss the construction of scanners, parsers, interpreters, and compilers.  Furthermore, we 
present a number of tools that can be used to build scanners and parsers.  In particular, 
the following tools will be introduced:
\begin{enumerate}
\item \blue{\textsc{Ply}} \index{\textsc{Ply}} generates a parser for \textsl{Python} programs. 
\item \blue{\textsc{Antlr}} \index{\textsc{Antlr}} can generate parsers for various programming languages.  In
      particular, \textsc{Antlr} can be used to generate parsers for both \textsl{Python} and \textsl{Java}.  
\end{enumerate}
All of these tools are \emph{program generators}, i.e.~they take as input the description of a
language and produce a parser as output. 

Some parts of these lecture notes are currently written in the German language, while other parts 
are written in English.  As time permits, I hope to eventually translate everything into the English language.
As I am currently rewriting parts of these lecture notes, these notes will undoubtedly contain their
fair amount of typos and other errors.  If you spot an error, I would like you to either
send an email to 
\\[0.2cm]
\hspace*{1.3cm}
\href{mailto:karl.stroetmann@dhbw-mannheim.de}{\texttt{karl.stroetmann@dhbw-mannheim.de}}
\\[0.2cm]
or to contact me via discord.  Alternatively, you are also welcome to clone my
\href{https://github.com/karlstroetmann/Formal-Languages/}{github} repository and send a pull request.

\section{Basic Definitions}
The central notion of this lecture is the notion of a 
\href{http://en.wikipedia.org/wiki/Formal_language}{\emph{formal language}}, \index{formal language} which
basically is a set of strings that is defined in some precise mathematical way.  In order to be able to define
this notion we require some definitions.  

\begin{Definition}[Alphabet]
An \blue{alphabet} $\Sigma$ \index{alphabet} is a finite, non-empty set of \emph{characters}:
\\[0.2cm]
\hspace*{1.3cm}
$\Sigma = \{ c_1, \cdots, c_n \}$. 
\\[0.2cm]
Sometimes, we use the term \blue{symbol} \index{symbol} to denote a character.
\qed
\end{Definition}

\examplesEng
\begin{enumerate}
\item $\Sigma = \{ 0, 1\}$ is an alphabet that can be used to represent binary numbers.
\item $\Sigma = \{ \mathtt{a}, \cdots, \mathtt{z}, \mathtt{A}, \cdots, \mathtt{Z} \}$ 
      is the alphabet used for the English language.
\item The set $\Sigma_{\textsc{\scriptsize Ascii}} = \{ 0, 1, \cdots, 127 \}$ is known as the
      \href{http://en.wikipedia.org/wiki/ASCII}{\textsc{Ascii}-Alphabet}. \index{\textsc{Ascii}-Alphabet} The numbers are
      interpreted as letters, digits, punctuation symbols, and control characters.
      For example, the numbers in the set $\{65, \cdots, 90 \}$ represent the letters
      $\{\mathtt{A}, \cdots, \mathtt{Z}\}$.  
      \eox
\end{enumerate}

\begin{Definition}[Strings]
Given an alphabet $\Sigma$, a \blue{string} \index{string} is a list of characters from $\Sigma$.
In the theory of formal languages, these lists are written without bracket symbols and without
separating comma symbols.  If $c_1,\cdots,c_n \in \Sigma$, then we write 
\\[0.2cm]
\hspace*{1.3cm}
$w = c_1\cdots c_n$ \quad instead of \quad $w = [c_1,\cdots,c_n]$.
\\[0.2cm]
The empty string is denoted as $\varepsilon$,\index{$\varepsilon$} i.e.~we have $\varepsilon =
\texttt{\symbol{34}\symbol{34}}$.
The set of all strings that can be constructed
from the alphabet $\Sigma$ is written as $\Sigma^*$. \index{$\Sigma^*$} For emphasis, strings are often
surrounded by quotation marks. \qed
\end{Definition}

\examplesEng
\begin{enumerate}
\item Assume that $\Sigma = \{0, 1\}$.  If we define
      \\[0.2cm]
      \hspace*{1.3cm}
      $w_1 := \mathtt{\symbol{34}01110\symbol{34}}$ \quad and \quad $w_2 := \mathtt{\symbol{34}11001\symbol{34}}$,
      \\[0.2cm]
      then both $w_1$and $w_2$ are strings.  Therefore we have
      \\[0.2cm]
      \hspace*{1.3cm}
      $w_1 \in \Sigma^*$ \quad and \quad $w_2 \in \Sigma^*$.
\item Assume that $\Sigma = \{\mathtt{a}, \;\cdots\!,\; \mathtt{z}\}$.   If we define
      \\[0.2cm]
      \hspace*{1.3cm}
      $w := \mathtt{\symbol{34}example}\symbol{34}$,
      \\[0.2cm]
      then we have $w \in \Sigma^*$. \eox
\end{enumerate}
The \emph{length} of a string $w$ is defined as the number of characters composing $w$.
The length of $w$ is written as $|w|$.  \index{length of a string}  We use square brackets to extract the characters from a string:
Given a string $w$ and a natural number $i \leq |w|$, we agree that $w[i]$ denotes the $i$-th
character of the string $w$.   We start to count the characters at 0 as this is the convention used in many
many modern programming languages like \texttt{C}, \textsl{Java}, and \textsl{Python}.

Next, we define the \blue{concatenation} \index{concatenation} of two strings $w_1$ and $w_2$ as the string $w$ that
results from appending the string $w_2$ at the end of  $w_1$.  The concatenation of $w_1$ and $w_2$
is written as $w_1 \cdot w_2$ or sometimes even shorter as $w_1w_2$.  
\vspace*{0.3cm}

\exampleEng
If $\Sigma = \{\mathtt{0},\mathtt{1}\}$ and, furthermore,  $w_1 = \mathtt{\symbol{34}01\symbol{34}}$ and $w_2 = \mathtt{\symbol{34}10\symbol{34}}$, then we have
\\[0.2cm]
\hspace*{1.3cm}
$w_1 \cdot w_2 = \mathtt{\symbol{34}0110\symbol{34}}$ \quad and \quad $w_2 \cdot w_1 = \mathtt{\symbol{34}1001\symbol{34}}$.  \eox

\begin{Definition}[Formal Language] \hspace*{\fill} \linebreak
If $\Sigma$ is an alphabet, then a \emph{precisely defined} subset $L \subseteq \Sigma^*$
is called a \blue{formal language}. \index{formal language} \qed
\end{Definition}

At this point, your first reaction might be to ask what exactly is a \emph{precisely defined} subset.  The idea
is to have some kind of unambiguous definition of the subset $L$ that makes up the language.  We have an
unambiguous definition that specifies whether a given string is indeed part of the defined language.  To give
one negative example, a language like English is not a formal language as their is no precise definition of
what constitutes a valid sentence of the English language.

The previous definition is very general.  As the lecture proceeds, we will define several
specializations of this concept.  For us, the two most important specializations are 
\href{http://en.wikipedia.org/wiki/Regular_language}{regular languages} \index{regular language} and
\href{http://en.wikipedia.org/wiki/Context-free_language}{context-free languages}, \index{context-free language}
because these two categories are most important in computer science. 

\examplesEng
\begin{enumerate}
\item Assume that $\Sigma = \{\mathtt{0},\mathtt{1}\}$.  Define
      \\[0.2cm]
      \hspace*{1.3cm}
      $L_\mathbb{N} = \{ \mathtt{1}+w \mid w \in \Sigma^* \} \cup \{ \mathtt{0} \}$
      \\[0.2cm]
      Then $L_\mathbb{N}$ is the language consisting of all strings that can be interpreted as
      natural numbers given in binary notation.  The language contains all strings from $\Sigma^*$  that start with 
      the character \texttt{1} as well as the string \texttt{0}, which only contains the character
      \texttt{0}.  For example, we have
      \\[0.2cm]
      \hspace*{1.3cm}
      $\mathtt{\symbol{34}100\symbol{34}} \in L_\mathbb{N}$, \quad but \quad $\mathtt{010} \not\in L_\mathbb{N}$.
      \\[0.2cm]
      Let us define a function 
      \\[0.2cm]
      \hspace*{1.3cm}
      $\textsl{value}: L_\mathbb{N} \rightarrow \mathbb{N}$
      \\[0.2cm]
      on the set $L_\mathbb{N}$.  We define $\textsl{value}(w)$ by induction on the length of $w$.
      We call $\textsl{value}(w)$ the \blue{interpretation} of $w$.  The idea is that
      $\textsl{value}(w)$ computes the number represented by the string $w$:
      \begin{enumerate}
      \item $\textsl{value}(\mathtt{0}) = 0$, $\textsl{value}(1) = 1$,
      \item $|w| > 0 \rightarrow \textsl{value}(w\mathtt{0}) = 2 \cdot \textsl{value}(w)   $,
      \item $|w| > 0 \rightarrow \textsl{value}(w\mathtt{1}) = 2 \cdot \textsl{value}(w) + 1$.
      \end{enumerate}
\item Again we have $\Sigma = \{0,1\}$. Define the language $L_\mathbb{P}$
      to be the set of all strings from $L_\mathbb{N}$ that are prime numbers:
      \\[0.2cm]
      \hspace*{1.3cm}
      $L_\mathbb{P} := \{ w \in L_\mathbb{N} \mid \textsl{value}(w) \in \mathbb{P} \}$
      \\[0.2cm]
      Here, $\mathbb{P}$ denotes the set of \blue{prime numbers}, \index{prime number} which is the set of all natural
      numbers $p$ bigger than $1$ that have no divisor other than $1$ or $p$:
      \\[0.2cm]
      \hspace*{1.3cm}
      $\mathbb{P} = \bigl\{ p \in \mathbb{N} \;\big|\; \{ t \in \mathbb{N} \mid \exists k \in
      \mathbb{N} : k \cdot t = p \} = \{1, p\} \bigr\}$.
\item Define $\Sigma_{\textsc{\scriptsize Ascii}} = \{ 0, \cdots, 127\}$.  Furthermore, define $L_C$
      as the set of all strings of the form
      \\[0.2cm]
      \hspace*{1.3cm}
      \texttt{char* $f$(char* $x$) \{ $\cdots$ \}}
      \\[0.2cm]
      that are, furthermore, valid \texttt{C} function definitions.
      Therefore,  $L_C$ contains all those strings that can be interpreted as a \texttt{C} function $f$
      such that $f$ takes a single argument which is a string and returns a value which is also a
      string.
\item Define $\Sigma := \Sigma_{\textsc{\scriptsize Ascii}} \cup \{\dag\}$, where
      $\mathtt{\dag}$ is some new symbol that is different from all symbols in
      $\Sigma_{\textsc{\scriptsize Ascii}}$.
      The \blue{universal language} \index{universal language} $L_u$ is the set of all strings of the form
      \\[0.2cm]
      \hspace*{1.3cm}
      $p$\dag$x$\dag$y$
      \\[0.2cm]
      such that
      \begin{enumerate}
      \item $p \in L_C$,
      \item $x,y \in \Sigma_{\textsc{Ascii}}^*$,
      \item if $f$ is the function that is defined by $p$, then $f(x)$ yields the result $y$.
            \eox
      \end{enumerate}
\end{enumerate}
The examples given above demonstrate that the notion of a formal language is very broad.
While it is easy to recognize the strings of the language $L_\mathbb{N}$, it is quite a bit more
difficult to decide whether a string is a member of 
$L_\mathbb{P}$ or $L_C$.  Finally, since the
\href{https://en.wikipedia.org/wiki/Halting_problem}{halting problem} is undecidable, there can be 
no algorithm that is able to decide whether a string  
$w$ is an element of the language $L_u$. However, this language is still \blue{semi-decidable}:  If there is a string
$w$ such that $w \in L_u$, then we are able to prove this.

\section{Overview}
My goal in this lecture is to cover the following topics.  In order to describe these topics, I will need to
use some notions that I cannot define precisely at this point.  Don't worry if you don't yet understand these 
notions, as they will be explained once we get there.  Basically, this lecture comes in three parts.
\begin{enumerate}
\item In the first part we discuss the theory of \blue{regular languages}.
  \begin{enumerate}[(a)]
  \item We will start with \emph{regular expressions}.  After a formal definition of this notion,
        we discuss how regular expressions are used in \textsl{Python}.
  \item Next, we show how the tool \textsc{Ply} can be used to generate scanners.
  \item Then, we turn to the implementation of regular expressions via \blue{finite state machines}.
  \item We show how the \blue{equivalence} of regular expressions can be checked.
  \item We finish our discussion of regular languages by discussing their limits.
        In particular, we discuss the \blue{Pumping Lemma}. 
  \end{enumerate}
\item In the second part of this lecture we discuss \blue{context free languages}.
      Context free languages are a formalism to describe the syntax of programming languages.
      In particular, the theory of regular languages can be used to implement parsers.
      \begin{enumerate}[(a)]
      \item We discuss the definition of \blue{context free grammars}.  These are used to specify 
            context free languages.  
      \item We discuss the parser generators \textsc{Antlr} and \textsc{Ply}.
      \item We present the theory that is necessary to understand the parser generator \textsc{Ply}.
      \item we proceed to discuss the limits of context free languages.
      \end{enumerate}
\item In the last part of this lecture we discuss interpreters and compilers.
\end{enumerate}

\section{Literature}
In addition to these lecture notes there are three books that I would like to recommend:
\begin{enumerate}[(a)]
\item \emph{Introduction to Automata Theory, Languages, and Computation}
      \cite{hopcroft:06}

      This book is the bible with respect to the theory of formal languages and it contains all the theoretical
      results discussed in this lecture. 
      Obviously, we will only be able to cover a small part of the results discussed in this book.
\item \emph{Introduction to the Theory of Computation}
      \cite{sipser:2012}

      This is another readable introduction to the theory of formal languages.  It also discusses
      the theory of computability, which is not covered in this lecture.
\item \emph{Compilers --- Principles, Techniques and Tools}
      \cite{aho:2006}

      This book is one of the standard references with respect to the theory of compilers.  It also covers a fair amount of
      the theory of formal languages.
\end{enumerate}

\section{Check your Understanding}
\begin{enumerate}[(a)]
\item Define the notion of an \blue{alphabet}.
\item Given an alphabet, define the notion of a \blue{string}.
\item Define the notion of a \blue{formal language}.
\end{enumerate}


%%% Local Variables: 
%%% mode: latex
%%% TeX-master: "formal-languages.tex"
%%% End: 
