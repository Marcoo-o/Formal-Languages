\chapter{Bottom-Up-Parser\label{chapter:bottom-up}}
Bei der Konstruktion eines Parsers gibt es generell zwei M\"oglichkeiten:  Wir k\"onnen
\emph{Top-Down} oder \emph{Bottom-Up} vorgehen.  Den Top-Down-Ansatz haben wir bereits
ausf\"uhrlich diskutiert.  In diesem Kapitel erl\"autern wir nun den Bottom-Up-Ansatz.
Dazu stellen wir im n\"achsten Abschnitt das allgemeine Konzept vor, das einem
\textsl{Bottom-Up-Parser} zu Grunde liegt.  
Im darauf folgenden Abschnitt zeigen wir, wie Bottom-Up-Parser implementiert werden k\"onnen
und stellen als eine Implementierungsm\"oglichkeit die \emph{Shift-Reduce-Parser} vor.
Ein Shift-Reduce-Parser arbeitet mit Hilfe einer Tabelle, in der
hinterlegt ist, wie der Parser in einem bestimmten Zustand die Eingaben verarbeiten muss.
Die Theorie, wie eine solche Tabelle sinnvoll mit Informationen gef\"ullt werden kann,
entwickeln wir dann in dem folgenden Abschnitt: Zun\"achst diskutieren wir die
\emph{SLR-Parser} (\emph{simple LR-Parser}).  Dies ist die einfachste Klasse von Shift-Reduce-Parsern.
Das Konzept der SLR-Parser ist leider f\"ur die Praxis nicht m\"achtig genug.  Daher verfeinern wir
dieses Konzept und erhalten so die Klasse der
\emph{kanonischen LR-Parser}.  Da die Tabellen f\"ur LR-Parser in der Praxis
h\"aufig gro{\ss} werden, vereinfacht man diese Tabellen etwas und erh\"alt dann das Konzept der
\emph{LALR-Parser}, das von der M\"achtigkeit zwischen dem Konzept der \emph{SLR-Parser} und dem
Konzept der \emph{LR-Parser} liegt.  In dem folgenden Kapitel werden wir dann den Parser-Generator
\textsl{JavaCup} diskutieren, der ein LALR-Parser ist.

\section{Bottom-Up-Parser}
Die mit \textsc{Antlr} erstellten Parser sind sogenannte \emph{Top-Down-Parser}: Ausgehend von dem
Start-Symbol der Grammatik wurde versucht, eine gegebene Eingabe durch Anwendung der verschiedenen
Grammatik-Regeln zu parsen.  Die Parser, die wir nun entwickeln werden, sind
\emph{Bottom-Up-Parser}.  Bei einem solchen Parser ist die Idee, dass wir von dem zu parsenden
String ausgehen und dort Terminale anhand der rechten Seiten der Grammatik-Regeln zusammenfassen.  


Wir versuchen den String ``\texttt{1 + 2 * 3}'' mit der Grammatik, die durch die
Regeln 
\[ 
\begin{array}[t]{rcl}
  E & \rightarrow & E \quoted{+} P \;|\; P  \\[0.1cm]
  P & \rightarrow & P \quoted{*} F \;|\; F  \\[0.1cm]
  F & \rightarrow & \squoted{1} \;|\; \squoted{2} \;|\; \squoted{3} 
\end{array}
\]
gegeben ist,  zu parsen.  Dazu suchen wir in diesem String
Teilstrings, die den rechten Seiten von Grammatikregeln entsprechen, wobei wir den String
von links nach rechts durchsuchen.  Auf diese Art versuchen wir,
einen Parse-Baum r\"uckw\"arts von unten aufzubauen:
\\[0.2cm]
\hspace*{0.3cm} 
$
\begin{array}[t]{lcll}
\texttt{1 + 2 * 3} & \Leftarrow & F \texttt{ + 2 * 3} 
                                & (\mbox{Regel:\ }  F \rightarrow  \squoted{1}) \\
                   & \Leftarrow & P \texttt{ + 2 * 3} 
                                & (\mbox{Regel:\ } P \rightarrow F) \\
                   & \Leftarrow & E \texttt{ + 2 * 3} 
                                & (\mbox{Regel:\ }  E \rightarrow  P) \\
                   & \Leftarrow & E \texttt{ + } F \texttt{ * 3} 
                                & (\mbox{Regel:\ } F \rightarrow  \squoted{2}) \\
                   & \Leftarrow & E \texttt{ + } P \texttt{ * 3} 
                                & (\mbox{Regel:\ } P \rightarrow F) \\
                   & \Leftarrow & E \texttt{ + } P \texttt{ * } F 
                                & (\mbox{Regel:\ } F \rightarrow \squoted{3}) \\
                   & \Leftarrow & E \texttt{ + } P & (\mbox{Regel:\ } P \rightarrow P \quoted{*} F) \\
                   & \Leftarrow & E                & (\mbox{Regel:\ } E \rightarrow E \quoted{+} P) 
\end{array}
$
\\[0.2cm]
Im ersten Schritt haben wir beispielsweise die Grammatik-Regel $F \rightarrow \squoted{1}$
benutzt, um den String ``\texttt{1}'' durch $F$ zu ersetzen und dabei dann den String 
``F\texttt{ + 2 * 3}'' erhalten.  Im zweiten Schritt haben wir die Regel
$P \rightarrow F$ benutzt, um $F$ durch $P$ zu
ersetzen.  Auf diese Art und Weise haben wir am Ende den urspr\"unglichen String 
``\texttt{1 + 2 * 3}'' auf E zur\"uck gef\"uhrt.  Wir k\"onnen an dieser Stelle zwei
Beobachtungen machen:
\begin{enumerate}
\item Wir ersetzen bei unserem Vorgehen immer den am weitesten links stehenden Teilstring,
      der ersetzt werden kann, wenn wir den anfangs gegebenen String auf das Start-Symbol
      der Grammatik zur\"uck f\"uhren wollen.
\item Schreiben wir die Ableitung, die wir r\"uckw\"arts konstruiert haben, noch einmal
      in der richtigen Reihenfolge hin, so erhalten wir:
\\[0.2cm]
\hspace*{1.3cm}
$
\begin{array}[t]{lcl}
E  & \Rightarrow & E \texttt{ + } P \\
   & \Rightarrow & E \texttt{ + } P \texttt{ * } F \\
   & \Rightarrow & E \texttt{ + } P \texttt{ * 3} \\
   & \Rightarrow & E \texttt{ + } F \texttt{ * 3} \\
   & \Rightarrow & E \texttt{ + 2 * 3} \\
   & \Rightarrow & P \texttt{ + 2 * 3} \\
   & \Rightarrow & F \texttt{ + 2 * 3} \\
   & \Rightarrow & \texttt{1 + 2 * 3}
\end{array}
$
\\[0.2cm]
       Wir sehen hier, dass bei dieser Ableitung immer die am weitesten rechts stehende
       syntaktische Variable ersetzt worden ist.  Eine derartige Ableitung wird als
       \emph{Rechts-Ableitung} bezeichnet.   

       Im Gegensatz dazu ist es bei den Ableitungen, die ein \emph{Top-Down-Parser}
       erzeugt, genau umgekehrt:  Dort wird immer die am weitesten links stehende
       syntaktische Variable ersetzt.  Die mit einem solchen Parser erzeugten Ableitungen
       hei{\ss}en daher \emph{Links-Ableitungen}.
\end{enumerate}
Die obigen beiden Beobachtungen sind der Grund, weshalb die Parser, die wir in diesem
Kapitel diskutieren, als \emph{LR-Parser} bezeichnet werden.  Das \emph{L} steht f\"ur
\emph{\underline{l}eft to right} und beschreibt die Vorgehensweise, dass der String immer
von links nach rechts durchsucht wird, w\"ahrend das \emph{R} f\"ur 
\emph{\underline{r}everse rightmost derivation} steht
und ausdr\"uckt, dass solche Parser eine Rechts-Ableitung r\"uckw\"arts konstruieren.
\vspace*{0.2cm}

\noindent
Bei der Implementierung eines LR-Parsers stellen sich zwei Fragen:
\begin{enumerate}
\item Welche Teilstrings ersetzen wir?
\item Welche Regeln verwenden wir dabei?
\end{enumerate}
Die Beantwortung dieser Fragen ist im Allgemeinen nicht trivial.  Zwar gehen wir die Strings immer
von links nach rechts durch, aber damit ist noch nicht unbeding klar, welchen Teilstring wir
ersetzen, denn die potentiell zu ersetzenden Teilstrings k\"onnen sich durchaus \"uberlappen.
Betrachten wir beispielsweise das Zwischenergebnis
\\[0.2cm]
\hspace*{1.3cm}
$E \texttt{ + } P \texttt{ * 3}$,
\\[0.2cm]
das wir oben im f\"unften Schritt erhalten haben.
Hier k\"onnten wir den Teilstring ``P'' mit Hilfe der Regel
\\[0.2cm]
\hspace*{1.3cm}
$E \rightarrow P$
\\[0.2cm]
durch ``E'' ersetzen.  Dann w\"urden wir den String
\\[0.2cm]
\hspace*{1.3cm}
$E \texttt{ + } E \texttt{ * 3}$
\\[0.2cm]
erhalten.  Die einzigen Reduktionen, die wir jetzt noch durchf\"uhren k\"onnen, f\"uhren \"uber die
Zwischenergebnisse $E \texttt{ + } E \texttt{ * }  F$ und $E \texttt{ + } E \texttt{ * } P$
zu dem String
\\[0.2cm]
\hspace*{1.3cm}
$E \texttt{ + } E \texttt{ * }  E$,
\\[0.2cm]
der sich dann  aber mit der oben angegebenen Grammatik nicht mehr reduzieren l\"asst.  
Die Antwort auf die obigen Fragen, welchen Teilstring wir ersetzen und welche Regel wir
verwenden, setzt einiges an Theorie voraus, die wir
in den folgenden Abschnitten entwickeln werden.


\section{Shift-Reduce-Parser}
Wollen wir einen Bottom-Up-Parser implementieren, so m\"ussen wir uns zun\"achst die Frage stellen,
welche Datenstrukturen wir bei der Implementierung verwenden wollen.
Wenn wir uns dabei f\"ur einen Stack entscheiden, dann sprechen wir von einem
\emph{Shift-Reduce-Parser}.  Ist $G = \langle V, T, R, S \rangle$ eine kontextfreie Grammatik, so
wird ein Shift-Reduce-Parser $P$ durch ein 4-Tupel
\\[0.2cm]
\hspace*{1.3cm}
$P = \langle Q, q_0, \textsl{action}, \textsl{goto} \rangle$
\\[0.2cm]
beschrieben.  Dabei gilt:
\begin{enumerate}
\item $Q$ ist die Menge der \emph{Zust\"ande} des Shift-Reduce-Parsers.  

      Zun\"achst werden wir die einzelnen Zust\"ande rein abstrakt sehen und ihre Natur nicht n\"aher
      analysieren.  Sp\"ater, wenn wir 
      die Theorie der SLR-Parser diskutieren, werden wir erkennen, dass ein Zustand
      die Information speichert, mit welcher Grammatik-Regel der Parser gerade zu parsen versucht
      und welcher Teil der rechten Seite der Regel bereits erkannt worden ist.
\item $q_0 \in Q$ ist der Start-Zustand.
\item $\textsl{action}$ ist eine Funktion, die als Argumente einen Zustand $q \in Q$
      und ein Terminal $t \in T$ erh\"alt.  Das Ergebnis ist ein Element der Menge
      \\[0.2cm]
      \hspace*{1.3cm}
      $\textsl{Action} :=
       \bigl\{ \pair(\texttt{shift}, q)  \mid q \in Q \bigr\}               \cup 
       \bigl\{ \pair(\texttt{reduce}, r) \mid r \in R \bigr\} \cup 
       \bigl\{ \texttt{accept} \bigr\}                        \cup
       \bigl\{ \texttt{error}  \bigr\}                         $,
      \\[0.2cm]
      wobei \texttt{shift}, \texttt{reduce}, \texttt{accept} und \texttt{error} hier einfach als
      Strings interpretiert werden, mit denen die verschiedenen Arten von Ergebnissen der Funktion
      $\textsl{action}()$ unterschieden werden k\"onnen.  
      Zusammenfassend haben wir also:
      \\[0.2cm]
      \hspace*{1.3cm}
      $\textsl{action}: Q \times T \rightarrow \textsl{Action}$.
\item \textsl{goto} ist eine Funktion, die jedem Zustand $q \in Q$ und jeder syntaktischen Variablen
      $v \in V$ einen neuen Zustand aus $Q$ zuordnet:
      \\[0.2cm]
      \hspace*{1.3cm}
      $\textsl{goto}: Q \times V \rightarrow Q$.
\end{enumerate}
Ein Shift-Reduce-Parser arbeitet nun mit den folgenden Daten-Strukturen.
\begin{enumerate}
\item Einem Stack \textsl{states}, auf dem Zust\"ande aus der Menge $Q$ abgelegt werden:
      \\[0.2cm]
      \hspace*{1.3cm}
      $\textsl{states} \in \textsl{Stack}\langle Q \rangle$.
\item Einem Stack \textsl{symbols}, auf dem Grammatik-Symbole, also Terminale und Variablen-Symbole
      abgelegt werden: 
      \\[0.2cm]
      \hspace*{1.3cm}
      $\textsl{symbols} \in \textsl{Stack}\langle T \cup V \rangle$.
\end{enumerate}
Zur Vereinfachung der folgenden \"Uberlegungen nehmen wir an, dass die Menge $T$ der Terminale das
spezielle Symbol ``\textsc{Eof}'' enth\"alt und dass dieses Symbol das Ende des zu parsenden
Strings spezifiziert aber sonst in dem String nicht auftritt.  

\begin{figure}[!ht]
\centering
\begin{Verbatim}[ frame         = lines, 
                  framesep      = 0.3cm, 
                  firstnumber   = 1,
                  labelposition = bottomline,
                  numbers       = left,
                  numbersep     = -0.2cm,
                  xleftmargin   = 0.8cm,
                  xrightmargin  = 0.8cm,
                ]
    class srParser(actionTable, gotoTable, stateTable) {
        mActionTable := actionTable;
        mGotoTable   := gotoTable;
    
        parseSR := procedure(tl) {
            index   := 1;      // point to next token
            symbols := [];     // stack of symbols
            states  := ["s0"]; // stack of states, "s0" is the start state
            while (true) {            
                q := states[-1];
                t := tl[index];
                p := mActionTable[q,t];
                match (p) {
                case om: 
                     return false;
                case Shift(s):
                     symbols := symbols + [t];
                     states  := states  + [s];
                     index   += 1;
                case Rule(head, body):
                     n       := #body;
                     symbols := symbols[.. -(n+1)];
                     states  := states[.. -(n+1)]; 
                     symbols := symbols + [head];
                     state   := states[-1];
                     states  := states + [ mGotoTable[state, head] ];
                case Accept():
                     return true;
                } 
            }
        };
    }
\end{Verbatim}
\vspace*{-0.3cm}
\caption{Implementation of a shift-reduce parser in \textsc{SetlX}}
\label{fig:shift-reduce-parser.stlx}
\end{figure}

Figure \ref{fig:shift-reduce-parser.stlx} on page \pageref{fig:shift-reduce-parser.stlx}
shows the implementation of the 
\href{http://randoom.org/Software/SetlX}{\textsc{SetlX}}
class \texttt{srParser} that implements
shift-reduce-parsing via its method \textsl{parseSR}.  
This method assumes that the function \textsl{action} is coded
as a binary relation that is stored in the member variable \texttt{mActionTable}.
The function \textsl{goto} is also represented as a binary relation.  It is stored in the
member variable \texttt{mGotoTable}.  The method \textsl{parseSR} is called with one
argument \textsl{tl}.  This is the list of tokens that have to be parsed.  The last
element of this list is the special token ``\textsc{Eof}'' denoting the end of file.  The invocation 
$\textsl{parseSR}(\textsl{tl})$ returns \texttt{true} if the token list \textsl{tl} can be
parsed successfully and \texttt{false} otherwise.  The implementation of \textsl{parseSR} works as follows:
\begin{enumerate}
\item The variable \textsl{index} points to the next token in the token list that is to be
      read.  Therefore, this variable is initialized to 1.
\item The variable \textsl{symbols} stores the stack of symbols.  The top of this stack is
      at the end of this list.  Initially, the stack of symbols is empty.
\item The variable \textsl{states} is the stack of states.  The start state is assumed to
      be the state ``\texttt{s0}''.  Therefore this stack is initialized to contain
      only this state.
\item The main loop of the parser 
      \begin{itemize}
      \item sets the variable $q$ to the current state,
      \item initializes $t$ to the next token, and then
      \item sets $p$ by looking up the appropriate action in the action table.
            Therefore $p$ is equal to $\textsl{action}(q, t)$.
      \end{itemize}
      What happens next depends on this value of $\textsl{action}(q, t)$.
      \begin{enumerate}
      \item $\textsl{action}(q,t)$ is undefined.

            If $\textsl{action}(q,t)$ is undefined, then we really have $\textsl{action}(q, t) = \mathtt{error}$.
            However, for reasons of space efficiency, the action table does not store
            error entries.  Therefore, in this case the parser has found a syntax error
            and returns \texttt{false}.
      \item $\textsl{action}(q,t) = \pair(\texttt{shift}, s)$.
        
            This action is represented in \textsc{SetlX} as the term \texttt{Shift($s$)}.
            In this case, the token $t$ is pushed onto the symbol stack in line 17,
            while the state $s$ is pushed onto the stack of states.  Furthermore,
            the variable \textsl{index} is incremented to point to the next unread token.            
      \item $\textsl{action}(q,t) = \pair(\texttt{reduce}, A \rightarrow X_1 \cdots X_n)$.

            This action is represented as the \textsc{SetlX} term $\texttt{Rule}(A, [X_1, \cdots,X_n])$.
            In this case, we use the grammar rule
            \\[0.2cm]
            \hspace*{1.3cm}
            $r = (A \rightarrow X_1 \cdots X_n)$
            \\[0.2cm]
            to reduce the symbol stack.  The \textsc{SetlX} variable \textsl{head} represents the left
            hand side $A$ of this rule, while the list $[X_1, \cdots,X_n]$ is represented
            by the \textsc{SetlX} variable \textsl{body}.

            In this case, it can be shown that the symbols $X_1$, $\cdots$, $X_n$ are on top of the
            symbol stack.  As we are going to reduce the symbol stack with the rule $r$,
            we remove these $n$ symbols from the symbol stack and replace them with the
            variable $A$.
            
            Furthermore, we have to remove $n$ states from the stack of states.
            After that, we set \textsl{state} to the state that is then on top of the
            stack of states.  Next, the new state  $\textsl{goto}(\textsl{state}, A)$ is put on top of
            the stack of states in line 26.
      \item $\textsl{action}(q,t) = \texttt{accept}$.

            In this case parsing is successful and therefore the function returns \texttt{true}.
      \end{enumerate}
\end{enumerate}
In order to make the function \textsl{parseSR} work we have to provide an implementation
of the functions $\textsl{action}$ and $\textsl{goto}$.
The tables  \ref{tab:action} and \ref{tab:goto} show these functions for the grammar given
in Figure \ref{fig:Expr.grammar}.  For this grammar, there are 16 different states, which have
been baptized as $s_0$, $s_1$, $\cdots$, $s_{15}$.  The tables use two different abbreviations:
\begin{enumerate}
\item \shft{i} is short for $\pair(\texttt{shift}, s_i)$.
\item \rdc{i} is short for  $\pair(\texttt{reduce}, r_i)$, where $r_i$ denothes the
      grammar rule number $i$.  Here, we have numbered the rules as follows:
      \begin{enumerate}
      \item $r_1 = (\textsl{expr} \rightarrow \textsl{expr} \quoted{+} \textsl{product})$
      \item $r_2 = (\textsl{expr} \rightarrow \textsl{expr} \quoted{-} \textsl{product})$
      \item $r_3 = (\textsl{expr} \rightarrow \textsl{product})$
      \item $r_4 = (\textsl{product} \rightarrow \textsl{product} \quoted{*} \textsl{factor})$
      \item $r_5 = (\textsl{product} \rightarrow \textsl{product} \quoted{/} \textsl{factor})$
      \item $r_6 = (\textsl{product} \rightarrow \textsl{factor})$
      \item $r_7 = (\textsl{factor} \rightarrow  \quoted{(} \textsl{expr} \quoted{)})$
      \item $r_8 = (\textsl{factor} \rightarrow  \textsc{Number})$
      \end{enumerate}
\end{enumerate}
The corresponding grammar is shown in Figure \ref{fig:Expr.grammar}.  The coding of the
functions \textsl{action} and \textsl{goto} is shown in the Figures
\ref{fig:parse-table.stlx:action1}, \ref{fig:parse-table.stlx:action2}, and
\ref{fig:parse-table.stlx:goto} on the following pages.   Of course, at present we do not have any
idea how the functions \textsl{action} and \textsl{goto} are computed.  This requires some theory
that will be presented in the next subsection.


\begin{figure}[htbp]
  \begin{center}    
  \framebox{
  \framebox{
  \begin{minipage}[t]{8cm}
  \begin{eqnarray*}
  \textsl{expr}    & \rightarrow & \;\textsl{expr} \quoted{+} \textsl{product}  \\
                   & \mid        & \;\textsl{expr} \quoted{-} \textsl{product}  \\
                   & \mid        & \;\textsl{product}                           \\[0.2cm]
  \textsl{product} & \rightarrow & \;\textsl{product} \quoted{*} \textsl{factor}\\
                   & \mid        & \;\textsl{product} \quoted{/} \textsl{factor}\\
                   & \mid        & \;\textsl{factor}                            \\[0.2cm]
  \textsl{factor}  & \rightarrow &   \quoted{(} \textsl{expr} \quoted{)}        \\
                   & \mid        & \;\textsc{Number} 
  \end{eqnarray*}
  \vspace*{-0.5cm}

  \end{minipage}}}
  \end{center}
  \caption{A grammar for arithmetical expressions.}
  \label{fig:Expr.grammar}
\end{figure}

\begin{table}[!ht]
  \centering
\framebox{
\begin{tabular}{|r|l|l|l|l|l|l|l|l|}
\hline
 State & \texttt{EOF} & \texttt{+} & \texttt{-} & \texttt{*} & \texttt{/} & \texttt{(} & \texttt{)} & \textsc{Number} \\
\hline
\hline
$s_{0}$ &              &            &            &            &            & \shft{5}   &            & \shft{2} \\
\hline
$s_{1}$ & \rdc{6}      & \rdc{6}    & \rdc{6}    & \rdc{6}    & \rdc{6}    &            & \rdc{6}    &          \\
\hline
$s_{2}$ & \rdc{8}      & \rdc{8}    & \rdc{8}    & \rdc{8}    & \rdc{8}    &            & \rdc{8}    &          \\
\hline
$s_{3}$ & \rdc{3}      & \rdc{3}    & \rdc{3}    & \shft{12}  & \shft{11}  &            & \rdc{3}    &          \\
\hline
$s_{4}$ & \accept      & \shft{8}   & \shft{9}   &            &            &            &            &          \\
\hline
$s_{5}$ &              &            &            &            &            & \shft{5}   &            & \shft{2} \\
\hline
$s_{6}$ &              & \shft{8}   & \shft{9}   &            &            &            & \shft{7}   &          \\
\hline
$s_{7}$ & \rdc{7}      & \rdc{7}    & \rdc{7}    & \rdc{7}    & \rdc{7}    &            & \rdc{7}    &          \\
\hline
$s_{8}$ &              &            &            &            &            & \shft{5}   &            & \shft{2} \\
\hline
$s_{9}$ &              &            &            &            &            & \shft{5}   &            & \shft{2} \\
\hline
$s_{10}$ & \rdc{2}     & \rdc{2}    & \rdc{2}    & \shft{12}  & \shft{11}  &            & \rdc{2}    &          \\
\hline
$s_{11}$ &             &            &            &            &            & \shft{5}   &            & \shft{2} \\
\hline
$s_{12}$ &             &            &            &            &            & \shft{5}   &            & \shft{2} \\
\hline
$s_{13}$ & \rdc{4}     & \rdc{4}    & \rdc{4}    & \rdc{4}    & \rdc{4}    &            & \rdc{4}    &          \\
\hline
$s_{14}$ & \rdc{5}     & \rdc{5}    & \rdc{5}    & \rdc{5}    & \rdc{5}    &            & \rdc{5}    &          \\
\hline
$s_{15}$ & \rdc{1}     & \rdc{1}    & \rdc{1}    & \shft{12}  & \shft{11}  &            & \rdc{1}    &          \\
\hline
  \end{tabular}}
  \caption{The function $\textsl{action}()$.}
  \label{tab:action}
\end{table}

\begin{table}[!ht]
  \centering
\framebox{
\begin{tabular}{|r|r|r|r|}
\hline
State   & \textsl{expr} & \textsl{product} & \textsl{factor} \\
\hline
\hline
$s_{0}$ & $s_{4}$       & $s_{3}$          & $s_{1}$         \\
\hline
$s_{1}$ &               &                  &                 \\
\hline
$s_{2}$ &               &                  &                 \\
\hline
$s_{3}$ &               &                  &                 \\
\hline
$s_{4}$ &               &                  &                 \\
\hline
$s_{5}$ & $s_{6}$       & $s_{3}$          & $s_{1}$         \\
\hline
$s_{6}$ &               &                  &                 \\
\hline
$s_{7}$ &               &                  &                 \\
\hline
$s_{8}$ &               & $s_{15}$         & $s_{1}$         \\
\hline
$s_{9}$ &               & $s_{10}$         & $s_{1}$         \\
\hline
$s_{10}$ &              &                  &                 \\
\hline
$s_{11}$ &              &                  & $s_{14}$        \\
\hline
$s_{12}$ &              &                  & $s_{13}$        \\
\hline
$s_{13}$ &              &                  &                 \\
\hline
$s_{14}$ &              &                  &                 \\
\hline
$s_{15}$ &              &                  &                 \\
\hline
  \end{tabular}}
  \caption{The function $\textsl{goto}()$.}
  \label{tab:goto}
\end{table}

\begin{figure}[!ht]
\centering
\begin{Verbatim}[ frame         = lines, 
                  framesep      = 0.3cm, 
                  firstnumber   = 1,
                  labelposition = bottomline,
                  numbers       = left,
                  numbersep     = -0.2cm,
                  xleftmargin   = 0.8cm,
                  xrightmargin  = 0.8cm,
                ]
    actionTable := {};
    
    r1 := Rule("E", ["E", "+", "P"]);      r4 := Rule("P", ["P", "*", "F"]);
    r2 := Rule("E", ["E", "-", "P"]);      r5 := Rule("P", ["P", "/", "F"]);
    r3 := Rule("E", ["P"]);                r6 := Rule("P", ["F"]);          
      
    r7 := Rule("F", ["(", "E", ")"]);
    r8 := Rule("F", ["int"]);
    
    actionTable["s0", "("  ] := Shift("s5");
    actionTable["s0", "int"] := Shift("s2");
    
    actionTable["s1", "EOF"] := r6;    actionTable["s2", "EOF"] := r8;
    actionTable["s1", "+"  ] := r6;    actionTable["s2", "+"  ] := r8;
    actionTable["s1", "-"  ] := r6;    actionTable["s2", "-"  ] := r8;
    actionTable["s1", "*"  ] := r6;    actionTable["s2", "*"  ] := r8;
    actionTable["s1", "/"  ] := r6;    actionTable["s2", "/"  ] := r8;
    actionTable["s1", ")"  ] := r6;    actionTable["s2", "("  ] := r8;
                                       actionTable["s2", ")"  ] := r8;

    actionTable["s3", "EOF"] := r3;
    actionTable["s3", "+"  ] := r3;
    actionTable["s3", "-"  ] := r3;
    actionTable["s3", "*"  ] := Shift("s12");
    actionTable["s3", "/"  ] := Shift("s11");
    actionTable["s3", ")"  ] := r3;
    
    actionTable["s4", "EOF"] := Accept();
    actionTable["s4", "+"  ] := Shift("s8");
    actionTable["s4", "-"  ] := Shift("s9");
    
    actionTable["s5", "("  ] := Shift("s5");
    actionTable["s5", "int"] := Shift("s2");
    
    actionTable["s6", "+"  ] := Shift("s8");
    actionTable["s6", "-"  ] := Shift("s9");
    actionTable["s6", ")"  ] := Shift("s7");
    
    actionTable["s7", "EOF"] := r7;
    actionTable["s7", "+"  ] := r7;
    actionTable["s7", "-"  ] := r7;
    actionTable["s7", "*"  ] := r7;
    actionTable["s7", "/"  ] := r7;
    actionTable["s7", ")"  ] := r7;   
\end{Verbatim}
\vspace*{-0.3cm}
\caption{Action table coded in \textsc{SetlX}, first part.}
\label{fig:parse-table.stlx:action1}
\end{figure}


\begin{figure}[!ht]
\centering
\begin{Verbatim}[ frame         = lines, 
                  framesep      = 0.3cm, 
                  firstnumber   = 1,
                  labelposition = bottomline,
                  numbers       = left,
                  numbersep     = -0.2cm,
                  xleftmargin   = 0.8cm,
                  xrightmargin  = 0.8cm,
                ]
    actionTable["s8", "("  ] := Shift("s5");
    actionTable["s8", "int"] := Shift("s2");
    
    actionTable["s9", "("  ] := Shift("s5");
    actionTable["s9", "int"] := Shift("s2");
    
    actionTable["s10", "EOF"] := r2;
    actionTable["s10", "+" ] := r2;
    actionTable["s10", "-" ] := r2;
    actionTable["s10", "*" ] := Shift("s12");
    actionTable["s10", "/" ] := Shift("s11");
    actionTable["s10", ")" ] := r2;
    
    actionTable["s11", "("  ] := Shift("s5");
    actionTable["s11", "int"] := Shift("s2");
    
    actionTable["s12", "("  ] := Shift("s5");
    actionTable["s12", "int"] := Shift("s2");
    
    actionTable["s13", "EOF"] := r4;
    actionTable["s13", "+"  ] := r4;
    actionTable["s13", "-"  ] := r4;
    actionTable["s13", "*"  ] := r4;
    actionTable["s13", "/"  ] := r4;
    actionTable["s13", ")"  ] := r4;
    
    actionTable["s14", "EOF"] := r5;
    actionTable["s14", "+"  ] := r5;
    actionTable["s14", "-"  ] := r5;
    actionTable["s14", "*"  ] := r5;
    actionTable["s14", "/"  ] := r5;
    actionTable["s14", ")"  ] := r5;
    
    actionTable["s15", "EOF"] := r1;
    actionTable["s15", "+"  ] := r1;
    actionTable["s15", "-"  ] := r1;
    actionTable["s15", "*"  ] := Shift("s12");
    actionTable["s15", "/"  ] := Shift("s11");
    actionTable["s15", ")"  ] := r1;
\end{Verbatim}
\vspace*{-0.3cm}
\caption{Action table coded in \textsc{SetlX}, second part.}
\label{fig:parse-table.stlx:action2}
\end{figure}

\begin{figure}[!ht]
\centering
\begin{Verbatim}[ frame         = lines, 
                  framesep      = 0.3cm, 
                  firstnumber   = 1,
                  labelposition = bottomline,
                  numbers       = left,
                  numbersep     = -0.2cm,
                  xleftmargin   = 0.8cm,
                  xrightmargin  = 0.8cm,
                ]
    gotoTable   := {};

    gotoTable["s0", "E"] := "s4";
    gotoTable["s0", "P"] := "s3";
    gotoTable["s0", "F"] := "s1";
    
    gotoTable["s5", "E"] := "s6";
    gotoTable["s5", "P"] := "s3";
    gotoTable["s5", "F"] := "s1";
    
    gotoTable["s8", "P"] := "s15";
    gotoTable["s8", "F"] := "s1";
    
    gotoTable["s9", "P"] := "s10";
    gotoTable["s9", "F"] := "s1";
    
    gotoTable["s11", "F"] := "s14";
    gotoTable["s12", "F"] := "s13";
\end{Verbatim}
\vspace*{-0.3cm}
\caption{Goto table coded in \textsc{SetlX}.}
\label{fig:parse-table.stlx:goto}
\end{figure}


 
%%% Local Variables: 
%%% mode: latex
%%% TeX-master: "formal-languages"
%%% End: 
