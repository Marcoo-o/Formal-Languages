\chapter{Bottom-Up-Parser\label{chapter:bottom-up}}
Bei der Konstruktion eines Parsers gibt es generell zwei M�glichkeiten:  Wir k�nnen
\blue{Top-Down} oder \blue{Bottom-Up} vorgehen.  Den Top-Down-Ansatz haben wir bereits
diskutiert.  In diesem Kapitel erl�utern wir nun den Bottom-Up-Ansatz.
Dazu stellen wir im n�chsten Abschnitt das allgemeine Konzept vor, das einem
\blue{Bottom-Up-Parser} zu Grunde liegt.  
Im darauf folgenden Abschnitt zeigen wir, wie Bottom-Up-Parser implementiert werden k�nnen
und stellen als eine Implementierungsm�glichkeit die \blue{Shift-Reduce-Parser} vor.
Ein Shift-Reduce-Parser arbeitet mit Hilfe einer Tabelle, in der
hinterlegt ist, wie der Parser in einem bestimmten Zustand die Eingaben verarbeiten muss.
Die Theorie, wie eine solche Tabelle sinnvoll mit Informationen gef�llt werden kann,
entwickeln wir dann in dem folgenden Abschnitt: Zun�chst diskutieren wir die
\blue{SLR-Parser} (\blue{simple LR-Parser}).  Dies ist die einfachste Klasse von Shift-Reduce-Parsern.
Das Konzept der SLR-Parser ist leider f�r die Praxis nicht m�chtig genug.  Daher verfeinern wir
dieses Konzept und erhalten so die Klasse der
\blue{kanonischen LR-Parser}.  Da die Tabellen f�r LR-Parser in der Praxis
h�ufig gro� werden, vereinfachen wir diese Tabellen etwas und erhalten dann das Konzept der
\blue{LALR-Parser}, das von der M�chtigkeit zwischen dem Konzept der \blue{SLR-Parser} und dem
Konzept der \blue{LR-Parser} liegt.  In dem folgenden Kapitel werden wir dann den Parser-Generator
\textsc{Ply} diskutieren, der ein LALR-Parser ist.

\section{Bottom-Up-Parser}
Die mit \textsc{Antlr} erstellten Parser sind sogenannte \blue{Top-Down-Parser}: Ausgehend von dem
Start-Symbol der Grammatik wurde versucht, eine gegebene Eingabe durch Anwendung der verschiedenen
Grammatik-Regeln zu parsen.  Die Parser, die wir nun entwickeln werden, sind
\blue{Bottom-Up-Parser}.  Bei einem solchen Parser ist die Idee, dass wir von dem zu parsenden
String ausgehen und dort Terminale anhand der rechten Seiten der Grammatik-Regeln zusammenfassen.  
Wir geben ein Beispiel und versuchen den String ``\texttt{1 + 2 * 3}'' mit der Grammatik, die durch die
Regeln 
\[ 
\begin{array}[t]{rcl}
  E & \rightarrow & E \quoted{+} P \;|\; P  \\[0.1cm]
  P & \rightarrow & P \quoted{*} F \;|\; F  \\[0.1cm]
  F & \rightarrow & \squoted{1} \;|\; \squoted{2} \;|\; \squoted{3} 
\end{array}
\]
gegeben ist,  zu parsen.  Dazu suchen wir in diesem String
Teilstrings, die den rechten Seiten von Grammatikregeln entsprechen, wobei wir den String
von links nach rechts durchsuchen.  Auf diese Art versuchen wir,
einen Parse-Baum r�ckw�rts von unten aufzubauen:
\\[0.2cm]
\hspace*{0.3cm} 
$
\begin{array}[t]{lcll}
\texttt{1 + 2 * 3} & \Leftarrow & F \texttt{ + 2 * 3} 
                                & (\mbox{Regel:\ }  F \rightarrow  \squoted{1}) \\
                   & \Leftarrow & P \texttt{ + 2 * 3} 
                                & (\mbox{Regel:\ } P \rightarrow F) \\
                   & \Leftarrow & E \texttt{ + 2 * 3} 
                                & (\mbox{Regel:\ }  E \rightarrow  P) \\
                   & \Leftarrow & E \texttt{ + } F \texttt{ * 3} 
                                & (\mbox{Regel:\ } F \rightarrow  \squoted{2}) \\
                   & \Leftarrow & E \texttt{ + } P \texttt{ * 3} 
                                & (\mbox{Regel:\ } P \rightarrow F) \\
                   & \Leftarrow & E \texttt{ + } P \texttt{ * } F 
                                & (\mbox{Regel:\ } F \rightarrow \squoted{3}) \\
                   & \Leftarrow & E \texttt{ + } P & (\mbox{Regel:\ } P \rightarrow P \quoted{*} F) \\
                   & \Leftarrow & E                & (\mbox{Regel:\ } E \rightarrow E \quoted{+} P) 
\end{array}
$
\\[0.2cm]
Im ersten Schritt haben wir beispielsweise die Grammatik-Regel $F \rightarrow \squoted{1}$
benutzt, um den String ``\texttt{1}'' durch $F$ zu ersetzen und dabei dann den String 
``F\texttt{ + 2 * 3}'' erhalten.  Im zweiten Schritt haben wir die Regel
$P \rightarrow F$ benutzt, um $F$ durch $P$ zu
ersetzen.  Auf diese Art und Weise haben wir am Ende den urspr�nglichen String 
``\texttt{1 + 2 * 3}'' auf $E$ zur�ck gef�hrt.  Wir k�nnen an dieser Stelle zwei
Beobachtungen machen:
\begin{enumerate}
\item Wir ersetzen bei unserem Vorgehen immer den am weitesten links stehenden Teilstring,
      der ersetzt werden kann, wenn wir den anfangs gegebenen String auf das Start-Symbol
      der Grammatik zur�ck f�hren wollen.
\item Schreiben wir die Ableitung, die wir r�ckw�rts konstruiert haben, noch einmal
      in der richtigen Reihenfolge hin, so erhalten wir:
\\[0.2cm]
\hspace*{1.3cm}
$
\begin{array}[t]{lcl}
E  & \Rightarrow & E \texttt{ + } P \\
   & \Rightarrow & E \texttt{ + } P \texttt{ * } F \\
   & \Rightarrow & E \texttt{ + } P \texttt{ * 3} \\
   & \Rightarrow & E \texttt{ + } F \texttt{ * 3} \\
   & \Rightarrow & E \texttt{ + 2 * 3} \\
   & \Rightarrow & P \texttt{ + 2 * 3} \\
   & \Rightarrow & F \texttt{ + 2 * 3} \\
   & \Rightarrow & \texttt{1 + 2 * 3}
\end{array}
$
\\[0.2cm]
       Wir sehen hier, dass bei dieser Ableitung immer die am weitesten rechts stehende
       syntaktische Variable ersetzt worden ist.  Eine derartige Ableitung wird als
       \blue{Rechts-Ableitung} \index{Rechts-Ableitung} bezeichnet.   

       Im Gegensatz dazu ist es bei den Ableitungen, die ein \blue{Top-Down-Parser}
       erzeugt, genau umgekehrt:  Dort wird immer die am weitesten links stehende
       syntaktische Variable ersetzt.  Die mit einem solchen Parser erzeugten Ableitungen
       hei�en daher \blue{Links-Ableitungen}. \index{Links-Ableitung}
\end{enumerate}
Die obigen beiden Beobachtungen sind der Grund, weshalb die Parser, die wir in diesem
Kapitel diskutieren, als \blue{LR-Parser} bezeichnet werden.  Das \blue{L} steht f�r
\blue{\underline{l}eft to right} und beschreibt die Vorgehensweise, dass der String immer
von links nach rechts durchsucht wird, w�hrend das \blue{R} f�r 
\blue{\underline{r}everse rightmost derivation} steht
und ausdr�ckt, dass solche Parser eine Rechts-Ableitung r�ckw�rts konstruieren.
\vspace*{0.2cm}

\noindent
Bei der Implementierung eines LR-Parsers stellen sich zwei Fragen:
\begin{enumerate}[(a)]
\item Welche Teilstrings ersetzen wir?
\item Welche Regeln verwenden wir dabei?
\end{enumerate}
Die Beantwortung dieser Fragen ist im Allgemeinen nicht trivial.  Zwar gehen wir die Strings immer
von links nach rechts durch, aber damit ist noch nicht unbeding klar, welchen Teilstring wir
ersetzen, denn die potentiell zu ersetzenden Teilstrings k�nnen sich durchaus �berlappen.
Betrachten wir beispielsweise das Zwischenergebnis
\\[0.2cm]
\hspace*{1.3cm}
$E \texttt{ + } P \texttt{ * 3}$,
\\[0.2cm]
das wir oben im f�nften Schritt erhalten haben.
Hier k�nnten wir den Teilstring ``P'' mit Hilfe der Regel
\\[0.2cm]
\hspace*{1.3cm}
$E \rightarrow P$
\\[0.2cm]
durch ``E'' ersetzen.  Dann w�rden wir den String
\\[0.2cm]
\hspace*{1.3cm}
$E \texttt{ + } E \texttt{ * 3}$
\\[0.2cm]
erhalten.  Die einzigen Reduktionen, die wir jetzt noch durchf�hren k�nnen, f�hren �ber die
Zwischenergebnisse $E \texttt{ + } E \texttt{ * }  F$ und $E \texttt{ + } E \texttt{ * } P$
zu dem String
\\[0.2cm]
\hspace*{1.3cm}
$E \texttt{ + } E \texttt{ * }  E$,
\\[0.2cm]
der sich dann  aber mit der oben angegebenen Grammatik nicht mehr reduzieren l�sst.  
Die Antwort auf die obigen Fragen, welchen Teilstring wir ersetzen und welche Regel wir
verwenden, setzt einiges an Theorie voraus, die wir
in den folgenden Abschnitten entwickeln werden.


\section{Shift-Reduce-Parser}
\blue{Shift-reduce parsing} is one way to implement bottom up parsing.
Assume a grammar $G = \langle V, T, R, S \rangle$ is given.  A  \blue{shift-reduce parser}\index{shift-reduce parser}
is defined as a 4-Tuple
\\[0.2cm]
\hspace*{1.3cm}
$P = \langle Q, q_0, \textsl{action}, \textsl{goto} \rangle$
\\[0.2cm]
where
\begin{enumerate}
\item $Q$ is the set of \blue{states} of the shift-reduce parser.  

      At first, states are purely abstract. 
\item $q_0 \in Q$ is the start state.
\item $\textsl{action}$ is a function taking two arguments. The first argument is a state $q \in Q$
      and the second argument is a terminal $t \in T$.  The result of this function is an element from the set
      \\[0.2cm]
      \hspace*{1.3cm}
      $\textsl{Action} :=
       \bigl\{ \pair(\texttt{shift}, q)  \mid q \in Q \bigr\}               \cup 
       \bigl\{ \pair(\texttt{reduce}, r) \mid r \in R \bigr\} \cup 
       \bigl\{ \texttt{accept} \bigr\}                        \cup
       \bigl\{ \texttt{error}  \bigr\}                         $.
      \\[0.2cm]
      Here \texttt{shift}, \texttt{reduce}, \texttt{accept}, and \texttt{error} are strings that serve to
      distinguish the different kinds of result of the function 
      $\textsl{action}$.  Therefore the signature of the function \textsl{action} is given as follows:
      \\[0.2cm]
      \hspace*{1.3cm}
      $\textsl{action}: Q \times T \rightarrow \textsl{Action}$.
\item \textsl{goto} is a function that takes a state $q \in Q$ and a syntactical variable
      $v \in V$ and computes a new state.  Therefore the signature of \textsl{goto} is as follows:
      \\[0.2cm]
      \hspace*{1.3cm}
      $\textsl{goto}: Q \times V \rightarrow Q$.
\end{enumerate}
A shift-reduce parser uses two stacks:
\begin{enumerate}[(a)]
\item \textsl{States} is a stack of states from the set $Q$:
      \\[0.2cm]
      \hspace*{1.3cm}
      $\textsl{States} \in \textsl{Stack}\langle Q \rangle$.
\item \textsl{Symbols} is a stack of grammar symbols, i.e.~this stack contains both terminals and syntactical
      variables:
      \\[0.2cm]
      \hspace*{1.3cm}
      $\textsl{Symbols} \in \textsl{Stack}\langle T \cup V \rangle$.
\end{enumerate}
In order to simplify the exposition of shift-reduce parsing we assume that the set $T$ of terminals contains
the special symbol ``\textsc{Eof}'' (short for \blue{e}nd \blue{o}f \blue{f}ile).
This symbol is assumed to occur at the end of the input string but does not occur elsewhere.

In order to understand how a shift-reduce parser works we introduce the notion of a
\blue{parser configuration}\index{parser configuration}.  A parser configuration is a triple of the form
\\[0.2cm]
\hspace*{1.3cm}
$\langle\textsl{States}, \textsl{Symbols}, \textsl{Tokens} \rangle$
\\[0.2cm]
where \textsl{States} and \textsl{Symbols} are the aforementioned stacks of states and grammar symbols, while
\textsl{Tokens} is the rest of the 
tokens from the input string that have not been processed.  The stack \textsl{States} always starts with the
start state $q_0$ and has a length that is one more than the length of the stack \textsl{Symbols}.
If the input string that is to be parsed has the form
\\[0.2cm]
\hspace*{1.3cm}
$[t_1, \cdots,t_n]$
\\[0.2cm]
and we have already reduced the first part $[t_1, \cdots,t_k]$ of the input string to produce the symbols 
\\[0.2cm]
\hspace*{1.3cm}
$[X_1,\cdots, X_m]$,
\\[0.2cm]
while $[t_{k+1},\cdots,t_n]$ is the part of the input string that still needs to be processed, then we have
\\[0.2cm]
\hspace*{1.3cm}
$\textsl{States} = [q_0,q_1\cdots, q_m]$, \quad $\textsl{Symbols} = [X_1,\cdots, X_m]$,
\quad and \quad $\textsl{Tokens} = [t_{k+1}, \cdots, t_n, \mathtt{EOF}]$,
\\[0.2cm]
and the parser configuration $\langle \textsl{States}, \textsl{Symbols}, \textsl{Tokens} \rangle$ is written as
\\[0.2cm]
\hspace*{1.3cm}
$q_0, q_1, \cdots, q_m \mid X_1, \cdots. X_m \mid t_{k+1}, \cdots, t_n,\, \mathtt{EOF}$.
\\[0.2cm]
Shift-reduce parsing starts out with the configuration
\\[0.2cm]
\hspace*{1.3cm}
$q_0 \mid \;  \mid t_{1}, \cdots, t_n,\, \mathtt{EOF}$.
\\[0.2cm] 
Then, parsing proceeds iteratively.  If the current configuration is
\\[0.2cm]
\hspace*{1.3cm}
$q_0, q_1, \cdots, q_m \mid X_1, \cdots, X_m \mid t_{k+1}, \cdots, t_n, \,\mathtt{EOF}$,
\\[0.2cm]
then there is a case distinction according to the value of $\textsl{action}(q_m, t_{k+1})$.
\begin{enumerate}[(a)]
\item If $\textsl{action}(q_m, t_{k+1}) = \textsl{error}$, then we know that the given string $t_1\cdots,t_n$
      is not generated by the given grammar and parsing is aborted with an error message.
\item If $\textsl{action}(q_m, t_{k+1}) = \textsl{accept}$, then we must have $t_{k+1} = \mathtt{EOF}$
      and we also must have $X_1 \cdots X_m = S$.  In this case, we have reduced the string $t_1\cdots t_m$
      to the start symbol $S$ of the given grammar and parsing finishes with success.
\item If $\textsl{action}(q_m, t_{k+1}) = \langle\textsl{shift}, q \rangle$, then the current configuration
      is changed into the new configuration
      \\[0.2cm]
      \hspace*{1.3cm}
      $q_0, q_1, \cdots, q_m, q \mid X_1, \cdots, X_m, t_{k+1} \mid t_{k+2}, \cdots, t_n, \,\mathtt{EOF}$,
      \\[0.2cm]
      i.e.~the next token $t_{k+1}$ is moved from the unread input to the top of the symbol stack
      and the new state $q$ is pushed onto the stack \textsl{States}.
\item If $\textsl{action}(q_m, t_{k+1}) = \langle\textsl{reduce}, r \rangle$, where $r$ is a grammar rule,
      then the grammar rule $r$ must have the form
      \\[0.2cm]
      \hspace*{1.3cm}
      $A \rightarrow X_{m-k} \cdots X_{m}$,
      \\[0.2cm]
      i.e.~the right hand side of the grammar rule matches the end of the stack \textsl{Symbols}.
      In this case, the symbols stack is reduced with this grammar rule, i.e.~the symbols
      $X_{m-k} \cdots X_m$ are replaced by $A$.  Furthermore, in the stack \textsl{States} the states
      $q_{m-k} \cdots q_m$ are replaced by the state $\textsl{goto}(q_{m-k-1}, A)$.
      Therefore, the configuration changes as follows:
      \\[0.2cm]
      \hspace*{1.3cm}
      $q_0, q_1, \cdots, q_{m-k-1}, \textsl{goto}(q_{m-k-1}, A) \mid X_1, \cdots, X_{m-k-1}, A \mid t_{k+1}, \cdots, t_n, \,\mathtt{EOF}$.
\end{enumerate}





\begin{figure}[!ht]
\centering
\begin{minted}[ frame         = lines, 
                framesep      = 0.3cm, 
                firstnumber   = 1,
                bgcolor       = sepia,
                numbers       = left,
                numbersep     = -0.2cm,
                xleftmargin   = 0.8cm,
                xrightmargin  = 0.8cm,
              ]{python3}
    class ShiftReduceParser():
        def __init__(self, actionTable, gotoTable):
            self.mActionTable = actionTable
            self.mGotoTable   = gotoTable
\end{minted}
\vspace*{-0.3cm}
\caption{Implementation of a shift-reduce parser in \textsl{Python}}
\label{fig:Shift-Reduce-Parser-Pure.ipynb}
\end{figure}

The class \texttt{Shift-Reduce-Parser-Pure} that is shown in Figure \ref{fig:Shift-Reduce-Parser-Pure.ipynb}
on page \pageref{fig:Shift-Reduce-Parser-Pure.ipynb} displays the class \\
\texttt{ShiftReduceParser}, which
maintains two dictionaries.
\begin{enumerate}[(a)]
\item \texttt{mActionTable} stores the function $\textsl{action}: Q \times T \rightarrow \textsl{Action}$.
\item \texttt{mGotoTable} stores the function $\textsl{goto}: Q \times V \rightarrow Q$.
\end{enumerate}

\begin{figure}[!ht]
\centering
\begin{minted}[ frame         = lines, 
                framesep      = 0.3cm, 
                firstnumber   = 1,
                bgcolor       = sepia,
                numbers       = left,
                numbersep     = -0.2cm,
                xleftmargin   = 0.8cm,
                xrightmargin  = 0.8cm,
              ]{python3}      
    def parse(self, TL):
        index   = 0      # points to next token
        Symbols = []     # stack of symbols
        States  = ['s0'] # stack of states, s0 is start state
        TL     += ['EOF']
        while True:
            q = States[-1]
            t = TL[index]
            p = self.mActionTable.get((q, t), 'error')
            if p == 'error': 
                return False
            elif p == 'accept':
                return True
            elif p[0] == 'shift':
                s = p[1]
                Symbols += [t]
                States  += [s]
                index   += 1
            elif p[0] == 'reduce':
                head, body = p[1]
                n       = len(body)
                Symbols = Symbols[:-n]
                States  = States [:-n] 
                Symbols = Symbols + [head]
                state   = States[-1]
                States += [ self.mGotoTable[state, head] ]
\end{minted}
\vspace*{-0.3cm}
\caption{Implementation of a shift-reduce parser in \textsl{Python}}
\label{fig:Shift-Reduce-Parser-Pure.ipynb:parse}
\end{figure}

Figure \ref{fig:Shift-Reduce-Parser-Pure.ipynb:parse} on page \pageref{fig:Shift-Reduce-Parser-Pure.ipynb:parse}
shows the implementation of the method \textsl{parse} that implements \blue{shift-reduce parsing}.  
This method assumes that the function \textsl{action} is coded
as a dictionary that is stored in the member variable \texttt{mActionTable}.
The function \textsl{goto} is also represented as a dictionary relation.  It is stored in the
member variable \texttt{mGotoTable}.  The method \textsl{parse} is called with one
argument \textsl{TL}.  This is the list of tokens that have to be parsed.  We append the special token
``\textsc{Eof}'' at the end of this list.  The invocation  
$\textsl{parse}(\textsl{TL})$ returns \texttt{True} if the token list \textsl{TL} can be
parsed successfully and \texttt{false} otherwise.  The implementation of \textsl{parse} works as follows:
\begin{enumerate}
\item The variable \textsl{index} points to the next token in the token list that is to be
      read.  Therefore, this variable is initialized to 0.
\item The variable \textsl{Symbols} stores the stack of symbols.  The top of this stack is
      at the end of this list.  Initially, the stack of symbols is empty.
\item The variable \textsl{States} is the stack of states.  The start state is assumed to
      be the state ``\texttt{s0}''.  Therefore this stack is initialized to contain
      only this state.
\item The main loop of the parser 
      \begin{itemize}
      \item sets the variable $q$ to the current state,
      \item initializes $t$ to the next token, and then
      \item sets $p$ by looking up the appropriate action in the action table.
            Therefore $p$ is equal to $\textsl{action}(q, t)$.
      \end{itemize}
      What happens next depends on this value of $\textsl{action}(q, t)$.
      \begin{enumerate}
      \item $\textsl{action}(q,t) = \textsl{error}$.

            In this case the parser has found a syntax error and returns \texttt{False}.
      \item $\textsl{action}(q,t) = \texttt{accept}$.

            In this case parsing is successful and therefore the function returns \texttt{True}.
      \item $\textsl{action}(q,t) = \pair(\texttt{shift}, s)$.
        
            In this case, the token $t$ is pushed onto the symbol stack in line 16,
            while the state $s$ is pushed onto the stack of states.  Furthermore,
            the variable \textsl{index} is incremented to point to the next unread token.            
      \item $\textsl{action}(q,t) = \pair(\texttt{reduce}, A \rightarrow X_1 \cdots X_n)$.

            In this case, we use the grammar rule
            \\[0.2cm]
            \hspace*{1.3cm}
            $r = (A \rightarrow X_1 \cdots X_n)$
            \\[0.2cm]
            to reduce the symbol stack.  The  variable \textsl{head} represents the left
            hand side $A$ of this rule, while the list $[X_1, \cdots,X_n]$ is represented
            by the variable \textsl{body}.

            In this case, it can be shown that the symbols $X_1$, $\cdots$, $X_n$ are on top of the
            symbol stack.  As we are going to reduce the symbol stack with the rule $r$,
            we remove these $n$ symbols from the symbol stack and replace them with the
            variable $A$.
            
            Furthermore, we have to remove $n$ states from the stack of states.
            After that, we set \textsl{state} to the state that is then on top of the
            stack of states.  Next, the new state  $\textsl{goto}(\textsl{state}, A)$ is put on top of
            the stack of states in line 26.
      \end{enumerate}
\end{enumerate}
In order to make the function \textsl{parse} work we have to provide an implementation
of the functions $\textsl{action}$ and $\textsl{goto}$.
The tables  \ref{tab:action} and \ref{tab:goto} show these functions for the grammar given
in Figure \ref{fig:Expr.grammar}.  For this grammar, there are 16 different states, which have
been baptized as $s_0$, $s_1$, $\cdots$, $s_{15}$.  The tables use two different abbreviations:
\begin{enumerate}
\item \shft{i} is short for $\pair(\texttt{shift}, s_i)$.
\item \rdc{i} is short for  $\pair(\texttt{reduce}, r_i)$, where $r_i$ denothes the
      grammar rule number $i$.  Here, we have numbered the rules as follows:
      \begin{enumerate}
      \item $r_1 = (\textsl{expr} \rightarrow \textsl{expr} \quoted{+} \textsl{product})$
      \item $r_2 = (\textsl{expr} \rightarrow \textsl{expr} \quoted{-} \textsl{product})$
      \item $r_3 = (\textsl{expr} \rightarrow \textsl{product})$
      \item $r_4 = (\textsl{product} \rightarrow \textsl{product} \quoted{*} \textsl{factor})$
      \item $r_5 = (\textsl{product} \rightarrow \textsl{product} \quoted{/} \textsl{factor})$
      \item $r_6 = (\textsl{product} \rightarrow \textsl{factor})$
      \item $r_7 = (\textsl{factor} \rightarrow  \quoted{(} \textsl{expr} \quoted{)})$
      \item $r_8 = (\textsl{factor} \rightarrow  \textsc{Number})$
      \end{enumerate}
\end{enumerate}
The corresponding grammar is shown in Figure \ref{fig:Expr.grammar}.  The definition of the grammar rules
and the coding of the
functions \textsl{action} and \textsl{goto} is shown in the Figures
\ref{fig:parse-table.stlx:grammar}, \ref{fig:parse-table.stlx:action}, and
\ref{fig:parse-table.stlx:goto} on the following pages.   Of course, at present we do not have any
idea how the functions \textsl{action} and \textsl{goto} are computed.  This requires some theory
that will be presented in the next section.


\begin{figure}[htbp]
  \begin{center}    
  \framebox{
  \framebox{
  \begin{minipage}[t]{8cm}
  \begin{eqnarray*}
  \textsl{expr}    & \rightarrow & \;\textsl{expr} \quoted{+} \textsl{product}  \\
                   & \mid        & \;\textsl{expr} \quoted{-} \textsl{product}  \\
                   & \mid        & \;\textsl{product}                           \\[0.2cm]
  \textsl{product} & \rightarrow & \;\textsl{product} \quoted{*} \textsl{factor}\\
                   & \mid        & \;\textsl{product} \quoted{/} \textsl{factor}\\
                   & \mid        & \;\textsl{factor}                            \\[0.2cm]
  \textsl{factor}  & \rightarrow &   \quoted{(} \textsl{expr} \quoted{)}        \\
                   & \mid        & \;\textsc{Number} 
  \end{eqnarray*}
  \vspace*{-0.5cm}

  \end{minipage}}}
  \end{center}
  \caption{A grammar for arithmetical expressions.}
  \label{fig:Expr.grammar}
\end{figure}

\begin{table}[!ht]
  \centering
\framebox{
\begin{tabular}{|r|l|l|l|l|l|l|l|l|}
\hline
 State & \texttt{EOF} & \texttt{+} & \texttt{-} & \texttt{*} & \texttt{/} & \texttt{(} & \texttt{)} & \textsc{Number} \\
\hline
\hline
$s_{0}$ &              &            &            &            &            & \shft{5}   &            & \shft{2} \\
\hline
$s_{1}$ & \rdc{6}      & \rdc{6}    & \rdc{6}    & \rdc{6}    & \rdc{6}    &            & \rdc{6}    &          \\
\hline
$s_{2}$ & \rdc{8}      & \rdc{8}    & \rdc{8}    & \rdc{8}    & \rdc{8}    &            & \rdc{8}    &          \\
\hline
$s_{3}$ & \rdc{3}      & \rdc{3}    & \rdc{3}    & \shft{12}  & \shft{11}  &            & \rdc{3}    &          \\
\hline
$s_{4}$ & \accept      & \shft{8}   & \shft{9}   &            &            &            &            &          \\
\hline
$s_{5}$ &              &            &            &            &            & \shft{5}   &            & \shft{2} \\
\hline
$s_{6}$ &              & \shft{8}   & \shft{9}   &            &            &            & \shft{7}   &          \\
\hline
$s_{7}$ & \rdc{7}      & \rdc{7}    & \rdc{7}    & \rdc{7}    & \rdc{7}    &            & \rdc{7}    &          \\
\hline
$s_{8}$ &              &            &            &            &            & \shft{5}   &            & \shft{2} \\
\hline
$s_{9}$ &              &            &            &            &            & \shft{5}   &            & \shft{2} \\
\hline
$s_{10}$ & \rdc{2}     & \rdc{2}    & \rdc{2}    & \shft{12}  & \shft{11}  &            & \rdc{2}    &          \\
\hline
$s_{11}$ &             &            &            &            &            & \shft{5}   &            & \shft{2} \\
\hline
$s_{12}$ &             &            &            &            &            & \shft{5}   &            & \shft{2} \\
\hline
$s_{13}$ & \rdc{4}     & \rdc{4}    & \rdc{4}    & \rdc{4}    & \rdc{4}    &            & \rdc{4}    &          \\
\hline
$s_{14}$ & \rdc{5}     & \rdc{5}    & \rdc{5}    & \rdc{5}    & \rdc{5}    &            & \rdc{5}    &          \\
\hline
$s_{15}$ & \rdc{1}     & \rdc{1}    & \rdc{1}    & \shft{12}  & \shft{11}  &            & \rdc{1}    &          \\
\hline
  \end{tabular}}
  \caption{The function $\textsl{action}()$.}
  \label{tab:action}
\end{table}

\begin{table}[!ht]
  \centering
\framebox{
\begin{tabular}{|r|r|r|r|}
\hline
State   & \textsl{expr} & \textsl{product} & \textsl{factor} \\
\hline
\hline
$s_{0}$ & $s_{4}$       & $s_{3}$          & $s_{1}$         \\
\hline
$s_{1}$ &               &                  &                 \\
\hline
$s_{2}$ &               &                  &                 \\
\hline
$s_{3}$ &               &                  &                 \\
\hline
$s_{4}$ &               &                  &                 \\
\hline
$s_{5}$ & $s_{6}$       & $s_{3}$          & $s_{1}$         \\
\hline
$s_{6}$ &               &                  &                 \\
\hline
$s_{7}$ &               &                  &                 \\
\hline
$s_{8}$ &               & $s_{15}$         & $s_{1}$         \\
\hline
$s_{9}$ &               & $s_{10}$         & $s_{1}$         \\
\hline
$s_{10}$ &              &                  &                 \\
\hline
$s_{11}$ &              &                  & $s_{14}$        \\
\hline
$s_{12}$ &              &                  & $s_{13}$        \\
\hline
$s_{13}$ &              &                  &                 \\
\hline
$s_{14}$ &              &                  &                 \\
\hline
$s_{15}$ &              &                  &                 \\
\hline
  \end{tabular}}
  \caption{The function $\textsl{goto}()$.}
  \label{tab:goto}
\end{table}


\begin{figure}[!ht]
\centering
\begin{Verbatim}[ frame         = lines, 
                  framesep      = 0.3cm, 
                  firstnumber   = 1,
                  labelposition = bottomline,
                  numbers       = left,
                  numbersep     = -0.2cm,
                  xleftmargin   = 0.8cm,
                  xrightmargin  = 0.8cm,
                ]
    r1 = ('E', ('E', '+', 'P'))
    r2 = ('E', ('E', '-', 'P'))
    r3 = ('E', ('P'))
    
    r4 = ('P', ('P', '*', 'F'))
    r5 = ('P', ('P', '/', 'F'))
    r6 = ('P', ('F'))
    
    r7 = ('F', ('(', 'E', ')'))
    r8 = ('F', ('int',))               
\end{Verbatim}
\vspace*{-0.3cm}
\caption{Grammar rules coded in \textsl{Python}.}
\label{fig:parse-table.stlx:grammar}
\end{figure}


\begin{figure}[!ht]
\centering
\begin{Verbatim}[ frame         = lines, 
                  framesep      = 0.3cm, 
                  firstnumber   = 1,
                  labelposition = bottomline,
                  numbers       = left,
                  numbersep     = -0.2cm,
                  xleftmargin   = 0.8cm,
                  xrightmargin  = 0.8cm,
                ]
    gotoTable   := {};

    gotoTable["s0", "E"] := "s4";
    gotoTable["s0", "P"] := "s3";
    gotoTable["s0", "F"] := "s1";
    
    gotoTable["s5", "E"] := "s6";
    gotoTable["s5", "P"] := "s3";
    gotoTable["s5", "F"] := "s1";
    
    gotoTable["s8", "P"] := "s15";
    gotoTable["s8", "F"] := "s1";
    
    gotoTable["s9", "P"] := "s10";
    gotoTable["s9", "F"] := "s1";
    
    gotoTable["s11", "F"] := "s14";
    gotoTable["s12", "F"] := "s13";
\end{Verbatim}
\vspace*{-0.3cm}
\caption{Goto table coded in \textsc{Python}.}
\label{fig:parse-table.stlx:goto}
\end{figure}


\begin{figure}[!ht]
\centering
\begin{Verbatim}[ frame         = lines, 
                  framesep      = 0.3cm, 
                  firstnumber   = 1,
                  labelposition = bottomline,
                  numbers       = none,
                  numbersep     = -0.2cm,
                  xleftmargin   = 0.0cm,
                  xrightmargin  = 0.0cm,
                ]
actionTable = {}

actionTable['s0', '('  ] = ('shift', 's5');  actionTable['s8', '('  ] = ('shift', 's5')   
actionTable['s0', 'int'] = ('shift', 's2');  actionTable['s8', 'int'] = ('shift', 's2')   
                                                                                         
actionTable['s1', 'EOF'] = ('reduce', r6);   actionTable['s9', '('  ] = ('shift', 's5')   
actionTable['s1', '+'  ] = ('reduce', r6);   actionTable['s9', 'int'] = ('shift', 's2')   
actionTable['s1', '-'  ] = ('reduce', r6);                                                
actionTable['s1', '*'  ] = ('reduce', r6);   actionTable['s10', 'EOF'] = ('reduce', r2)   
actionTable['s1', '/'  ] = ('reduce', r6);   actionTable['s10', '+' ] = ('reduce', r2)    
actionTable['s1', ')'  ] = ('reduce', r6);   actionTable['s10', '-' ] = ('reduce', r2)    
                                             actionTable['s10', '*' ] = ('shift', 's12')
actionTable['s2', 'EOF'] = ('reduce', r8);   actionTable['s10', '/' ] = ('shift', 's11')  
actionTable['s2', '+'  ] = ('reduce', r8);   actionTable['s10', ')' ] = ('reduce', r2)    
actionTable['s2', '-'  ] = ('reduce', r8);                                                
actionTable['s2', '*'  ] = ('reduce', r8);   actionTable['s11', '('  ] = ('shift', 's5')  
actionTable['s2', '/'  ] = ('reduce', r8);   actionTable['s11', 'int'] = ('shift', 's2')  
actionTable['s2', ')'  ] = ('reduce', r8);                                                
                                             actionTable['s12', '('  ] = ('shift', 's5')  
actionTable['s3', 'EOF'] = ('reduce', r3);   actionTable['s12', 'int'] = ('shift', 's2')  
actionTable['s3', '+'  ] = ('reduce', r3);                                                
actionTable['s3', '-'  ] = ('reduce', r3);   actionTable['s13', 'EOF'] = ('reduce', r4)   
actionTable['s3', '*'  ] = ('shift', 's12'); actionTable['s13', '+'  ] = ('reduce', r4)   
actionTable['s3', '/'  ] = ('shift', 's11'); actionTable['s13', '-'  ] = ('reduce', r4)   
actionTable['s3', ')'  ] = ('reduce', r3);   actionTable['s13', '*'  ] = ('reduce', r4)   
                                             actionTable['s13', '/'  ] = ('reduce', r4)   
actionTable['s4', 'EOF'] = 'accept';         actionTable['s13', ')'  ] = ('reduce', r4)   
actionTable['s4', '+'  ] = ('shift', 's8');                                               
actionTable['s4', '-'  ] = ('shift', 's9');  actionTable['s14', 'EOF'] = ('reduce', r5)   
                                             actionTable['s14', '+'  ] = ('reduce', r5)   
actionTable['s5', '('  ] = ('shift', 's5');  actionTable['s14', '-'  ] = ('reduce', r5)   
actionTable['s5', 'int'] = ('shift', 's2');  actionTable['s14', '*'  ] = ('reduce', r5)   
                                             actionTable['s14', '/'  ] = ('reduce', r5)   
actionTable['s6', '+'  ] = ('shift', 's8');  actionTable['s14', ')'  ] = ('reduce', r5)   
actionTable['s6', '-'  ] = ('shift', 's9');                                               
actionTable['s6', ')'  ] = ('shift', 's7');  actionTable['s15', 'EOF'] = ('reduce', r1)   
                                             actionTable['s15', '+'  ] = ('reduce', r1)   
actionTable['s7', 'EOF'] = ('reduce', r7);   actionTable['s15', '-'  ] = ('reduce', r1)   
actionTable['s7', '+'  ] = ('reduce', r7);   actionTable['s15', '*'  ] = ('shift', 's12') 
actionTable['s7', '-'  ] = ('reduce', r7);   actionTable['s15', '/'  ] = ('shift', 's11') 
actionTable['s7', '*'  ] = ('reduce', r7);   actionTable['s15', ')'  ] = ('reduce', r1)   
actionTable['s7', '/'  ] = ('reduce', r7);
actionTable['s7', ')'  ] = ('reduce', r7);
\end{Verbatim}
\vspace*{-0.3cm}
\caption{Action table coded in \textsc{Python}.}
\label{fig:parse-table.stlx:action}
\end{figure}





 
%%% Local Variables: 
%%% mode: latex
%%% TeX-master: "formal-languages"
%%% End: 
