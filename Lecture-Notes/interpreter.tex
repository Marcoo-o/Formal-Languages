\section{Implementing a Simple Interpreter  \label{chapter:interpreter}}

\textsc{Antlr} supports \textsc{Ebnf} grammars. For example, the postfix operators
``\texttt{*}'', ``\texttt{+}'' and ``\texttt{?}'' known from regular expressions are
supported. We demonstrate the use of these operators by developing an interpreter for a
simple programming language using the parser generator \textsc{Antlr}. Figure
\ref{fig:Pure.g4} shows the \textsc{Antlr} grammar of the programming language for which
we are developing an interpreter in this section. The syntax of this language is modeled
after the syntax of the language \texttt{C}, but instead of the simple equality sign
``\texttt{=}'', we use the string ``\texttt{:=}'' as the assignment operator.

\begin{figure}[!ht]
\centering
\begin{Verbatim}[ frame         = lines, 
                  framesep      = 0.3cm, 
                  firstnumber   = 1,
                  labelposition = bottomline,
                  numbers       = left,
                  numbersep     = -0.2cm,
                  xleftmargin   = 0.8cm,
                  xrightmargin  = 0.8cm,
                ]
    grammar Pure;
    
    program  : statement+
             ;  
    statement: VAR ':=' expr ';'
             | 'print' '(' expr ')' ';'
             | 'if'    '(' boolExpr ')' '{' statement* '}'
             | 'while' '(' boolExpr ')' '{' statement* '}'
             ;
    boolExpr : expr '==' expr
             | expr '<'  expr
             ;
    expr     : expr '+' product
             | expr '-' product
             | product
             ;
    product  : product '*' factor
             | product '/' factor
             | factor
             ;
   factor    : 'read' '(' ')' 
             | '(' expr ')'
             | VAR
             | NUMBER
             ;
    VAR      : [a-zA-Z][a-zA-Z_0-9]*;
    NUMBER   : '0'|[1-9][0-9]*;
    
    MULTI_COMMENT : '/*' .*? '*/' -> skip;
    LINE_COMMENT  : '//' ~('\n')* -> skip;
    WS            : [ \t\n\r]     -> skip; 
\end{Verbatim}
\vspace*{-0.3cm}
\caption{\textsc{Antlr}-Grammatik f\"ur eine einfache Programmier-Sprache.}
\label{fig:Pure.g4}
\end{figure}

The commands of the language to be implemented include assignments, print commands,
\texttt{if} statements, and \texttt{while} loops. Figure \ref{fig:sum.sl} shows
an example program that conforms to this grammar. This program first reads a number
which is stored in the variable \texttt{n}. Subsequently, the sum
\[
\hspace*{1.3cm}
\ds\sum\limits_{\mbox{$\normalsize i=1$}}^{\mbox{$\normalsize n^2$}} \mbox{\Large $i$}
\]
is computed and printed using a \texttt{while} loop.

\begin{figure}[!ht]
\centering
\begin{Verbatim}[ frame         = lines, 
                  framesep      = 0.3cm, 
                  labelposition = bottomline,
                  numbers       = left,
                  numbersep     = -0.2cm,
                  xleftmargin   = 0.8cm,
                  xrightmargin  = 0.8cm,
                ]
    n := read();
    s := 0;
    i := 0;
    while (i < n * n) {
        i := i + 1;
        s := s + i;
    }
    print(s);
\end{Verbatim}
\vspace*{-0.3cm}
\caption{Ein Programm zur Berechnung der Summe $\sum\limits_{i=0}^{n^2} i$.}
\label{fig:sum.sl}
\end{figure}
\FloatBarrier

Similar to our program for symbolic differentiation, we will represent the individual commands as
nested tuples. The program shown in Figure \ref{fig:sum.sl} is represented by the nested tuple displayed in
Figure \ref{fig:sum.ast}. This nested tuple is nothing other than the abstract syntax tree
corresponding to the original program. 

\begin{figure}[!ht]
\centering
\begin{Verbatim}[ frame         = lines, 
                  framesep      = 0.3cm, 
                  labelposition = bottomline,
                  numbers       = left,
                  numbersep     = -0.2cm,
                  xleftmargin   = 0.8cm,
                  xrightmargin  = 0.8cm,
                ]
    ('program',
        ('read', 'n'),
        (':=', 's', 0),
        (':=', 'i', 0),
        ('while', ('<',  'i', ('*', 'n', 'n')),
            (':=', 'i', ('+', 'i', 1)),
            (':=', 's', ('+', 's', 'i'))
        ),
        ('print', 's')
    )
\end{Verbatim}
\vspace*{-0.3cm}
\caption{A nested tuple that represents the program from Figure \ref{fig:sum.sl}.}
\label{fig:sum.ast}
\end{figure}

\begin{figure}[!ht]
\centering
\begin{Verbatim}[ frame         = lines, 
                  framesep      = 0.3cm, 
                  labelposition = bottomline,
                  numbers       = left,
                  numbersep     = -0.3cm,
                  xleftmargin   = -0.3cm,
                  xrightmargin  = 0.0cm,
                ]
    grammar Simple;
    
    program returns [stmnt_list] 
        : {SL = ['program']} (s = statement {SL.append($s.stmnt)})+ {$stmnt_list = tuple(SL)}
        ;
    
    statement returns [stmnt]
        : v = VAR ':=' e = expr ';'       {$stmnt = (':=', $v.text, $e.result)}
        | 'print' '(' r = expr ')' ';'    {$stmnt = ('print', $r.result)}
        | 'if' '(' b = boolExpr ')' {SL = []} '{' (l = statement {SL.append($l.stmnt)})* '}' 
          {$stmnt = ('if', $b.result) + tuple(SL)}
        | 'while' '(' b = boolExpr ')' {SL = []} '{' (l = statement {SL.append($l.stmnt)})* '}' 
          {$stmnt = ('while', $b.result) + tuple(SL)}
        ;
    
    boolExpr returns [result]
        : l = expr '==' r = expr {$result = ('==', $l.result, $r.result)} 
        | l = expr '<'  r = expr {$result = ('<',  $l.result, $r.result)}
        ;
    
    expr returns [result]
        : e = expr '+' p = product {$result = ('+', $e.result, $p.result)}
        | e = expr '-' p = product {$result = ('-', $e.result, $p.result)}
        | p = product              {$result = $p.result}
        ;
    
    product returns [result]
        : p = product '*' f = factor {$result = ('*', $p.result, $f.result)}
        | p = product '/' f = factor {$result = ('/', $p.result, $f.result)}
        | f = factor                 {$result = $f.result}
        ;
    
    factor returns [result]
        : 'read' '(' ')' {$result = ('read',)}
        | '(' expr ')'   {$result = $expr.result}
        | v = VAR        {$result = $v.text}
        | n = NUMBER     {$result = int($n.text)}
        ;
    
    VAR    : [a-zA-Z][a-zA-Z_0-9]*;
    NUMBER : '0'|[1-9][0-9]*;
    
    MULTI_COMMENT : '/*' .*? '*/' -> skip;
    LINE_COMMENT  : '//' ~('\n')* -> skip;
    WS            : [ \t\n\r]     -> skip;
\end{Verbatim}
\vspace*{-0.3cm} %\$
\caption{\textsc{Antlr}-Spezifikation der Grammatik.}
\label{fig:Simple.g}
\end{figure}

\noindent
Figure \ref{fig:Simple.g} shows the implementation of the parser using the
\textsc{Antlr} tool.
\begin{enumerate}
\item The start symbol of the grammar is the variable \texttt{program}.
      When parsing this variable, the parser returns a tuple of commands in the variable
      \texttt{stmnt\_list}.
      To this end, we initialize the variable \texttt{SL}
      as a list containing only the string \texttt{'program'}. Subsequently, each recognized command
      is appended to this list. Finally, this list is converted into a tuple and stored in the
      attribute variable \texttt{stmnt\_list}.
\item The syntactic variable \texttt{statement} describes the various commands that 
      are supported in our simple language.
      \begin{enumerate}[(a)]
      \item The simplest commands are the assignments. These have the form:
            \\[0.2cm]
            \hspace*{1.3cm}
            $v \;\texttt{:=}\; e \texttt{;}$
            \\[0.2cm]
            Here, $v$ is a variable and $e$ is an arithmetic expression.
            Such an assignment is represented by the nested tuple
            \\[0.2cm]
            \hspace*{1.3cm}
            $(\texttt{':='}, v, e)$.
      \item The command for printing a value has the form:
            \\[0.2cm]
            \hspace*{1.3cm}
            $\texttt{print(} e \texttt{);}$
            \\[0.2cm]
            The purpose of this command is to compute the value of the expression $e$ and print it.
            This command is represented by the nested tuple
            \\[0.2cm]
            \hspace*{1.3cm}
            $(\texttt{'print'}, v)$.
      \item A conditional test has the syntax:
            \\[0.2cm]
            \hspace*{1.3cm}
            $\texttt{if}\; \texttt{(}\; b\; \texttt{)}\; \texttt{\{}\; \textsl{statements}\; \texttt{\}}$
            \\[0.2cm]
            Here, $b$ is an expression that evaluates to \texttt{True} or \texttt{False}
            and \texttt{statements} is a list of commands that are executed if
            $b$ evaluates to \texttt{True}.
            This command is represented by the nested tuple
            \\[0.2cm]
            \hspace*{1.3cm}
            $(\texttt{'if'}, b, \textsl{statements})$.
            \\[0.2cm]
            Here, $b$ is a nested tuple representing the Boolean expression.
      \item A loop has the syntax:
            \\[0.2cm]
            \hspace*{1.3cm}
            $\texttt{while}\; \texttt{(}\; b\; \texttt{)}\; \texttt{\{}\; \textsl{statements}\; \texttt{\}}$.
            \\[0.2cm]
            Here, $b$ is an expression that evaluates to \texttt{True} or \texttt{False}
            and \texttt{statements} is a list of commands that are executed as long as
            $b$ evaluates to \texttt{True}.
            This command is represented by the nested tuple
            \\[0.2cm]
            \hspace*{1.3cm}
            $(\texttt{'while'}, b, \textsl{statements})$.
      \end{enumerate}
\item The syntactic variable \texttt{boolExpr} describes an expression that takes a Boolean value.
      There are two ways to generate such a value.
      \begin{enumerate}
      \item An expression of the form
            \\[0.2cm]
            \hspace*{1.3cm}
            $l \;\texttt{==}\; r $
            \\[0.2cm]
            tests whether the evaluations of $l$ and $r$ yield the same values.
            This expression is represented by the nested tuple
            \\[0.2cm]
            \hspace*{1.3cm}
            $(\texttt{'=='}, l, r)$.
      \item An expression of the form
            \\[0.2cm]
            \hspace*{1.3cm}
            $l \;\texttt{<}\; r $
            \\[0.2cm]
            tests whether the evaluation of $l$ yields a value that is smaller than the
            value obtained from the evaluation of $r$. 
            This expression is represented by the nested tuple
            \\[0.2cm]
            \hspace*{1.3cm}
            $(\texttt{'<'}, l, r)$
      \end{enumerate}
\item In a similar manner, the syntactic variables \texttt{expr}, \texttt{product}, and \texttt{factor}
      describe arithmetic expressions.
      Since we have already discussed this in sufficient detail earlier, we will not delve further into
      the grammar rules that define these syntactic variables at this point.
\end{enumerate}
The specification of the tokens is shown in lines 40--45 of Figure \ref{fig:Simple.g}.
The scanner primarily differentiates between variables and numbers.
Variables start with an uppercase or lowercase letter, followed by additional digits and
the underscore. Sequences of digits are interpreted as numbers. If such a
sequence contains more than one character, the first digit must not be 0. Additionally, the
scanner removes whitespace and comments. We have also used the so-called \emph{non-greedy} version of the
operator ``\texttt{*}'' in the specification of multiline comments. The non-greedy version of the
operator ``\texttt{*}'' is written as ``\texttt{*?}'' and matches as little as possible. Therefore, the regular expression
\[
\hspace*{1.3cm}
\texttt{'/*' .*? '*/'}
\]
represents a string that starts with the character sequence ``\texttt{/*}'', ends with the character sequence ``\texttt{*/}'',
and is as short as possible. This ensures that in a line like
\[
\hspace*{1.3cm}
\texttt{/* Hugo */ i := i + 1; /* Anton */}
\]
two separate comments are recognized.


\begin{figure}[!ht]
\centering
\begin{Verbatim}[ frame         = lines, 
                  framesep      = 0.3cm, 
                  firstnumber   = 1,
                  labelposition = bottomline,
                  numbers       = left,
                  numbersep     = -0.2cm,
                  xleftmargin   = 0.0cm,
                  xrightmargin  = 0.0cm,
                ]
    def main(file):
        with open(file, 'r') as handle:
            program_text = handle.read()
        input_stream  = antlr4.InputStream(program_text)
        lexer         = SimpleLexer(input_stream)
        token_stream  = antlr4.CommonTokenStream(lexer)
        parser        = SimpleParser(token_stream)
        result        = parser.program()
        Statements    = result.stmnt_list
        execute_tuple(Statements)
\end{Verbatim}
\vspace*{-0.3cm}
\caption{Die Funktion \texttt{main}.}
\label{fig:Interpreter.ipynb:main}
\end{figure}

\begin{figure}[!ht]
\centering
\begin{Verbatim}[ frame         = lines, 
                  framesep      = 0.3cm, 
                  firstnumber   = 1,
                  labelposition = bottomline,
                  numbers       = left,
                  numbersep     = -0.2cm,
                  xleftmargin   = 0.0cm,
                  xrightmargin  = 0.0cm,
                ]
    def execute_tuple(Statement_List, Values={}):
        for stmnt in Statement_List:
            execute(stmnt, Values)
    
    def execute(stmnt: NestedTuple, Values: dict[str, Number]) -> None:
        match stmnt:
            case 'program':
                pass
            case (':=', var, value):
                Values[var] = evaluate(value, Values)
            case ('print', expr):
                print(evaluate(expr, Values))
            case ('if', test, *SL):
                if evaluate(test, Values):
                    execute_tuple(tuple(SL), Values)
            case ('while', test, *SL):
                while evaluate(test, Values):
                    execute_tuple(tuple(SL), Values)
            case _:
                assert False, f'{stmnt} unexpected'
\end{Verbatim}
\vspace*{-0.3cm}
\caption{The function execute \texttt{execute}.}
\label{fig:Interpreter.ipynb:execute}
\end{figure}

Figure \ref{fig:Interpreter.ipynb:main} shows the function \texttt{main}, which receives the name of a
file containing a program in our simple programming language as input. This program is parsed and
thus transformed into the nested tuple \texttt{Statements}. The function \texttt{execute\_tuple} executes
the individual commands in the tuple \texttt{Statements}. It uses the function \texttt{execute} to
execute a single command. Figure \ref{fig:Interpreter.ipynb:execute} shows the
implementation of the function \texttt{execute}. This implementation consists mainly of a large
case distinction based on the type of command to be executed.
\begin{enumerate}
\item First, we check if \texttt{statement} is the string \texttt{'program'}, which marks the beginning of
      the program.\footnote{
        This marking is only needed for the representation of the program as an abstract syntax tree.}
      Since this is only a marker and not a real command, nothing further needs to be
      done.
\item If the command is an assignment of the form
      \\[0.2cm]
      \hspace*{1.3cm}
      \texttt{(':=', var, value)}
      \\[0.2cm]   
      then the value of the expression \texttt{value} is computed using the function \texttt{evaluate}.
      This value is then stored in the dictionary \texttt{Values} under the key \texttt{var}.
\item If the command is an operation of the form
      \\[0.2cm]
      \hspace*{1.3cm}
      \texttt{('print', expr)}
      \\[0.2cm]
      then the expression \texttt{expr} is first evaluated using the function \texttt{evaluate}.
      The resulting value is then printed.
\item If the command is of the form
      \\[0.2cm]
      \hspace*{1.3cm}
      \texttt{('if', test, $s_1$, $\cdots$, $s_n$)}
      \\[0.2cm]
      then \texttt{test} is a nested tuple representing a Boolean expression and
      \texttt{($s_1$, $\cdots$, $s_n$)} is a tuple of commands stored in the variable \texttt{SL}.
      In this case, the expression \texttt{test} is first evaluated using the function \texttt{evaluate}.
      If this evaluation yields the value \texttt{True}, then the
      commands in the tuple \texttt{SL} are executed in sequence.
\item If the command is of the form
      \\[0.2cm]
      \hspace*{1.3cm}
      \texttt{('while', test, $s_1$, $\cdots$, $s_n$)}
      \\[0.2cm]
      then \texttt{test} is a Boolean expression and \texttt{($s_1$, $\cdots$, $s_n$)}
      is a tuple of commands stored in the variable \texttt{SL}.
      In this case, the expression \texttt{test} is first evaluated using the function \texttt{evaluate}.
      If this evaluation yields the value \texttt{True}, then the
      commands in the tuple \texttt{SL} are executed in sequence.
      Then the expression \texttt{test} is evaluated again. If the result is
      \texttt{False}, then the execution of the command is finished. Otherwise,
      the commands in the list \texttt{SL} are executed as long as the evaluation of \texttt{test}
      yields \texttt{False}.
\end{enumerate}


\begin{figure}[!ht]
\centering
\begin{Verbatim}[ frame         = lines, 
                  framesep      = 0.3cm, 
                  firstnumber   = 1,
                  labelposition = bottomline,
                  numbers       = left,
                  numbersep     = -0.2cm,
                  xleftmargin   = 0.0cm,
                  xrightmargin  = 0.0cm,
                ]
    def evaluate(expr: NestedTuple, Values: dict[str, Number]) -> Number:
        match expr:
            case int():
                return expr
            case str():
                return Values[expr] 
            case ('read',):
                return int(input('Please enter a natural number: '))
            case ('==', lhs, rhs):
                return evaluate(lhs, Values) == evaluate(rhs, Values)
            case ('<', lhs, rhs):
                return evaluate(lhs, Values) < evaluate(rhs, Values)
            case ('+', lhs, rhs):
                return evaluate(lhs, Values) + evaluate(rhs, Values)
            case ('-', lhs, rhs):
                return evaluate(lhs, Values) - evaluate(rhs, Values)
            case ('*', lhs, rhs):
                return evaluate(lhs, Values) * evaluate(rhs, Values)
            case ('/', lhs, rhs):
                return evaluate(lhs, Values) / evaluate(rhs, Values)
            case _:
                assert False, f'{expr} unexpected'
\end{Verbatim}
\vspace*{-0.3cm}
\caption{Die Funktion \texttt{evaluate}.}
\label{fig:Interpreter.ipynb:evaluate}
\end{figure}
\FloatBarrier

Figure \ref{fig:Interpreter.ipynb:evaluate} shows the implementation of the function \texttt{evaluate}.
This function receives an arithmetic expression and a dictionary storing the
values of the variables as input.
\begin{enumerate}[(a)]
\item If the expression to be evaluated is a number, we return this number as the result.
\item If the expression to be evaluated is a variable, we look up the value of this
      variable in the dictionary \texttt{Values} and return this value as the result.
\item If the expression to be evaluated is a call to the function \texttt{read},
      we prompt the user to enter a natural number. We then convert the string entered by the user into a number.
\item If the expression to be evaluated is a Boolean expression of the form
      \\[0.2cm]
      \hspace*{1.3cm}
      \texttt{('==', lhs, rhs)}
      \\[0.2cm]
      we recursively evaluate the expressions \texttt{lhs} and \texttt{rhs} and return \texttt{True} if and only if
      both expressions yield the same value.
\item If the expression to be evaluated is a Boolean expression of the form
      \\[0.2cm]
      \hspace*{1.3cm}
      \texttt{('<', lhs, rhs)}
      \\[0.2cm]
      we recursively evaluate the expressions \texttt{lhs} and \texttt{rhs} and return \texttt{True} if and only if
      the value obtained from the evaluation of \texttt{lhs} is less than the
      value obtained from the evaluation of \texttt{rhs}.
\item If the expression to be evaluated is a sum of the form
      \\[0.2cm]
      \hspace*{1.3cm}
      \texttt{('+', lhs, rhs)}
      \\[0.2cm]
      we recursively evaluate the expressions \texttt{lhs} and \texttt{rhs} and return the sum of these
      values. 
\item The evaluation of the arithmetic operators \texttt{'-'}, \texttt{'*'}, and \texttt{'/'}
      is analogous to the evaluation of the operator \texttt{'+'}.
\end{enumerate}
\FloatBarrier

\exerciseEng
\begin{enumerate}[(a)]
\item Extend the interpreter to support the operator ``\texttt{<=}''.
\item Add \texttt{for} loops to the interpreter.
\item Expand the interpreter to include logical operators
      ``\texttt{\&\&}'' for logical \emph{AND}, ``\texttt{||}'' for logical \emph{OR},
      and ``\texttt{!}'' for negation. The operator ``\texttt{!}'' should bind the
      strongest and the operator ``\texttt{||}'' the weakest.
\item Extend the syntax of arithmetic expressions to include predefined
      mathematical functions like $\texttt{exp}()$ or $\texttt{ln}()$.
\item Enhance the interpreter to allow for user-defined functions.
      \begin{enumerate}
      \item A function should only have access to its parameters and those variables that
            are defined locally inside the function.  
      \item A function should always return a value using a \texttt{return} statement.
            In order to facilitate that a function can contain multiple \texttt{return}
            statements and that a \texttt{return} statement can occur
            anywhere in the function,  use an \blue{exception} to communicate the
            return value and to transfer the control flow out of the function.
      \end{enumerate}
\item Support the data type of \blue{sets} of integers.  Sets can be created in the following ways:
      \begin{enumerate}[1.]
      \item The elements of a set can be enumerated as in the following  expression:
            \\[0.2cm]
            \hspace*{1.3cm}
            \texttt{\{ 1, 2, 3, 4 \}}.
      \item A set can be specified as a range as shown below:
            \\[0.2cm]
            \hspace*{1.3cm}
            \texttt{\{ 1, ..., 10 \}}.
      \item A set can be created as subset of a given set as shown in the following example:
            \\[0.2cm]
            \hspace*{1.3cm}
            \texttt{\{ x in S | mod(x, 2) == 0 \}.}
      \item A set can be created as the image of several sets, for example:
            \\[0.2cm]
            \hspace*{1.3cm}
            \texttt{\{ f(x,y) : x in S, y in T \}}
      \item Sets can be combined using binary operators:
            \begin{itemize}
            \item \texttt{S + T} is the union of \texttt{S} and \texttt{T}.
            \item \texttt{S * T} is the intersection of \texttt{S} and \texttt{T}.
            \item \texttt{S - T} is the set difference of \texttt{S} and \texttt{T}.
                  \eox
            \end{itemize}
      \end{enumerate}
\end{enumerate}

\section{Check your Understanding}
\begin{enumerate}[(a)]
\item What is the difference between an \textsc{Ebnf}-grammar and a context-free grammar?
\item Are you able to write an \textsc{Antlr} parser for a simple programming language?
\item Are you able to implement an \textsc{Antlr} parser that constructs an abstract syntax tree? 
\end{enumerate}

%%% Local Variables: 
%%% mode: latex
%%% TeX-master: "formal-languages"
%%% End: 
