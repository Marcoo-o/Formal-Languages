\chapter{Entwicklung eines einfachen Compilers} 
In diesem Kapitel konstruieren wir einen Compiler, der ein Fragment der Sprache \mytt{C}
in \textsl{Java}-Assembler �bersetzt.  
Das von dem Compiler �bersetzte Fragment der Sprache \mytt{C} bezeichnen wir als
\blue{\textsl{Integer}-\mytt{C}}, \index{\textsl{Integer}-\mytt{C}}
denn es steht dort nur der Datentyp \mytt{int} zur Verf�gung.
Zwar w�re es problemlos m�glich, auch weitere Datentypen zu unterst�tzen, allerdings w�rde der Mehraufwand dann
in keinem guten Verh�ltnis zum didaktischen Nutzen des Beispiels mehr stehen.

Ein Compiler besteht prinzipiell aus den folgenden Komponenten:
\begin{enumerate}
\item Der Scanner liest die zu �bersetzende Datei ein und zerlegt diese in eine Folge von Token.
      
      Wir werden unseren Scanner mit Hilfe des Werkzeugs \textsc{Ply} entwickeln.
\item Der Parser liest die Folge von Token und produziert als Ergebnis einen abstrakten Syntax-Baum.

      Wir werden bei der Erstellung des Parsers ebenfalls \textsc{Ply} verwenden.
\item Der Typ-Checker �berpr�ft den abstrakten Syntax-Baum auf Typ-Fehler.

      Da die von uns �bersetzte Sprache nur einen einzelnen Datentyp enth�lt, er�brigt
      sich diese Phase f�r den von uns entwickelten Compiler.
\item In realen Compilern erfolgt nun eine \emph{Optimierungsphase}, die wir aus Zeitgr�nden aber nicht mehr 
      betrachten k�nnen.
\item Der Code-Generator �bersetzt schlie�lich den Parse-Baum in eine Folge von
      \textsc{Java}-Assembler-Befehlen.  
\item Das dabei entstehende Assembler-Programm k�nnen wir mit \textsl{Jasmin} in Java-Bytecode �bersetzen.
\item Der Java-Bytecode kann mit Hilfe des Befehls \mytt{Java} ausgef�hrt werden.  
\end{enumerate}
Bei Compilern, deren Zielcode ein \textsc{Risc}-Assembler-Programm ist, wird normalerweise zun�chst auch ein
Code erzeugt, der dem 
\textsc{Jvm}-Code �hnelt.  Ein solcher Code wird als \emph{Zwischen-Code} bezeichnet.  Es bleibt
dann die Aufgabe eines sogenannten \emph{Backends}, daraus ein Assembler-Programm f�r eine gegebene
Prozessor-Architektur zur erzeugen.  Die schwierigste Aufgabe besteht hier darin, f�r die
verwendeten Variablen eine Register-Zuordnung zu finden, bei der m�glichst viele Variablen in
Registern vorgehalten werden k�nnen.  
Aus Zeitgr�nden k�nnen wir das Thema der Register-Zuordnung in dieser Vorlesung nicht behandeln.

\section{Die Programmiersprache \textsl{Integer}-\mytt{C}}

\begin{figure}[!ht]
  \begin{center}
  \begin{minipage}[t]{12.5cm}
  \begin{eqnarray*}    
\textsl{program}      & \rightarrow & \;\textsl{function}\mytt{+} \\[0.2cm]
\textsl{function}     & \rightarrow & \quoted{int} \textsc{Id} \quoted{(} \textsl{paramList} \quoted{)} \quoted{\{} \textsl{decl}\,\mytt{+}\; \textsl{stmnt}\mytt{+} \;\squoted{\}} \\[0.2cm]
\textsl{paramList}    & \rightarrow & \;(\squoted{int}\, \textsc{Id}\; (\squoted{,}\, \quoted{int} \textsc{Id})\mytt{*})?  \\[0.2cm]
\textsl{decl}  & \rightarrow & \quoted{int} \textsc{Id} \quoted{;}  \\[0.2cm]
\textsl{stmnt}    & \rightarrow & \quoted{\{} \textsl{stmnt}\mytt{*} \quoted{\}}  \\[0.0cm]
                      & \mid        & \;\textsc{Id} \quoted{=} \textsl{expr} \;\squoted{;} \\[0.0cm]
                      & \mid        &  \quoted{if} \quoted{(} \textsl{boolExpr} \quoted{)} \textsl{stmnt} \\[0.0cm]
                     & \mid         &  \quoted{if} \quoted{(} \textsl{boolExpr} \quoted{)} \textsl{stmnt} \quoted{else} \textsl{stmnt} \\[0.0cm]
                     & \mid        &  \quoted{while} \quoted{(} \textsl{boolExpr} \quoted{)} \textsl{stmnt} \\[0.0cm]
                     & \mid        &  \quoted{return} \textsl{expr}\; \squoted{;} \\[0.0cm]
                     & \mid        &  \;\textsl{expr} \;\squoted{;}               \\[0.2cm]
   \textsl{boolExpr} & \rightarrow & \textsl{expr} \;(\squoted{==} \mid \squoted{!=} \mid \squoted{<=} \mid \squoted{>=} \mid \squoted{<} \mid \squoted{>})\;  \textsl{expr} \\
         & \mid        &  \quoted{!} \textsl{boolExpr}             \\[0.0cm]
         & \mid        &  \textsl{boolExpr} \;(\squoted{\&\&} \mid \squoted{||})\; \textsl{boolExpr}   \\[0.2cm]
 \textsl{expr} & \rightarrow & \textsl{expr} \;(\squoted{+} \mid \squoted{-} \mid \squoted{*} \mid \squoted{/})\; \textsl{expr} \\[0.0cm]
     & \mid        &  \quoted{(} \textsl{expr} \quoted{)}                 \\[0.0cm]
     & \mid        &  \textsc{Number}                             \\[0.0cm]
     & \mid        &  \textsc{Id}\; (\squoted{(} (\textsl{expr}\; (\squoted{,}\; \textsl{expr})\mytt{*})?\;\squoted{)})?  
  \end{eqnarray*}
  \vspace*{-0.5cm}

  \end{minipage}
  \end{center}
  \caption{Eine \textsc{Ebnf}-Grammatik f�r \textsl{Integer}-\textsc{C}}
\label{fig:compiler.ebnf}
\end{figure}

\noindent
Wir stellen nun die Sprache \textsl{Integer}-\mytt{C} vor, die unser Compiler �bersetzen
soll.  In diesem Zusammenhang sprechen wir auch von der \emph{Quellsprache} unseres Compilers.
Abbildung \ref{fig:compiler.ebnf} auf Seite \pageref{fig:compiler.ebnf} zeigt die Grammatik der Quellsprache in 
erweiterter Backus-Naur-Form (\textsc{Ebnf}). 
Die Grammatik f�r \textsl{Integer}-\mytt{C} verwendet die folgenden Terminale:
\begin{enumerate}
\item \textsc{Id} steht f�r eine Folge von Ziffern, Buchstaben und dem Unterstrich, die 
      mit einem Buchstaben beginnt.  Eine \textsc{Id} bezeichnet entweder eine Variable oder
      den Namen einer Funktion.
\item \textsc{Number} steht f�r eine Folge von Ziffern, die als Dezimalzahl interpretiert wird.
\item Daneben haben wir eine Reihe von Operatoren wie z.B.~\quoted{+},\quoted{-}, etc., sowie Schl�sselw�rter
      wie beispielsweise \quoted{if}, \quoted{while}.
\end{enumerate}
Nach der oben angegebenen Grammatik ist ein Programm eine Liste von Funktionen.
Eine Funktion besteht aus der Deklaration der Signatur, worauf in geschweiften Klammern
eine Liste von Deklarationen (\textsl{decl}) und Befehlen (\textsl{stmt}) folgt.  
Der Aufbau der einzelnen Befehle ist dann
�hnlich wie bei der Sprache \textsc{Sl}, f�r die wir im Kapitel \ref{chapter:interpreter}
einen Interpreter entwickelt haben.
Die in Abbildung \ref{fig:compiler.ebnf} gezeigte Grammatik ist mehrdeutig:
\begin{enumerate}
\item Die Grammatik hat das \emph{Dangling-Else-Problem}.

      Dieses Problem haben wir bereits im Kapitel \ref{chapter:bottom-up} im Rahmen einer Aufgabe diskutiert.
      Wir hatten damals das Problem dadurch gel�st, dass wir das Schl�sselwort \quoted{else} shiften.
      Da dies genau das ist, was \textsc{Ply} per default in einem Shift-Reduce-Konflikt macht, brauchen wir
      uns um dieses Problem nicht weiter zu k�mmern.
\item F�r die bei arithmetischen und Boole'schen Ausdr�cken verwendeten Operatoren
      m�ssen Pr�zedenzen festgelegt werden, um die Mehrdeutigkeiten bei der Interpretation dieser Ausdr�cke
      aufzul�sen. 
\end{enumerate}

\begin{figure}[!ht]
\centering
\begin{Verbatim}[ frame         = lines, 
                  framesep      = 0.3cm, 
                  labelposition = bottomline,
                  numbers       = left,
                  numbersep     = -0.2cm,
                  xleftmargin   = 0.8cm,
                  xrightmargin  = 0.8cm,
                ]
    int sum(int n) {
        int s;
        s = 0;
        while (n != 0) {
            s = s + n;
            n = n - 1;    
        };
        return s;
    }  
    int main() {
        int n;
        n = 6 * 6;
        println(sum(n));
    }
\end{Verbatim}
\vspace*{-0.3cm}
\caption{Ein einfaches \textsc{Integer}-\mytt{C}-Programm.}
\label{fig:sum.c}
\end{figure}

\noindent
Abbildung \ref{fig:sum.c} zeigt ein einfaches \textsc{Integer}-\mytt{C}-Programm.  Die Funktion
$\textsl{sum}(n)$ berechnet die Summe
 $\sum\limits_{i=1}^n i$
und die Funktion $\textsl{main}()$ ruft die Funktion \mytt{sum} mit dem Argument $6 \cdot 6$ auf.
Die in dem Programm verwendete Funktion \mytt{println} gibt ihr Argument gefolgt von einem
Zeilenumbruch aus.  Wir gehen hier davon aus, dass diese Funktion vordefiniert ist.
\pagebreak

\section{Developing the  Scanner and the Parser}
We construct both the scanner and the parser using  \textsl{Ply}.
Figure \ref{fig:Compiler.ipynb:lex} shows the scanner.  The scanner
recognizes the operators, keywords, identifiers, and numbers.  As the scanner is mostly similar to those
scanners that we have already seen before, there are only a few points worth mentioning.

\begin{figure}[!ht]
\centering
\begin{Verbatim}[ frame         = lines, 
                  framesep      = 0.3cm, 
                  labelposition = bottomline,
                  numbers       = left,
                  numbersep     = -0.2cm,
                  xleftmargin   = 0.8cm,
                  xrightmargin  = 0.8cm,
                ]
    import ply.lex as lex
    
    tokens = [ 'NUMBER', 'ID', 'EQ', 'NE', 'LE', 'GE', 'AND', 'OR',
               'INT', 'IF', 'ELSE', 'WHILE', 'RETURN'
             ]
    
    t_NUMBER = r'0|[1-9][0-9]'
    
    t_EQ  = r'=='
    t_NE  = r'!='
    t_LE  = r'<='
    t_GE  = r'>='
    t_AND = r'&&'
    t_OR  = r'\|\|'
    
    def t_COMMENT(t):
        r'//[^\n]*'
        pass

    Keywords = { 'int'   : 'INT', 
                 'if'    : 'IF',
                 'else'  : 'ELSE', 
                 'while' : 'WHILE', 
                 'return': 'RETURN'
               }
    
    def t_ID(t):
        r'[a-zA-Z][a-zA-Z0-9_]*'
        t.type = Keywords.get(t.value, 'ID')
        return t
    
    literals = ['+', '-', '*', '/', '(', ')', '{', '}', ';', '=', '<', '>', '!']
    
    t_ignore  = ' \t\r'
    
    def t_newline(t):
        r'\n+'
        t.lexer.lineno += t.value.count('\n')

    def t_error(t):
        print(f"Illegal character '{t.value[0]}' in line {t.lineno}.")
        t.lexer.skip(1)
    
    __file__ = 'main'
    
    lexer = lex.lex()
\end{Verbatim}
\vspace*{-0.3cm}
\caption{The scanner for \textsl{Integer}-\mytt{C}.}
\label{fig:Compiler.ipynb:lex}
\end{figure}

\begin{enumerate}
\item If a token does not have to be manipulated by the scanner, then it can be defined by a simple equation 
      of the form 
      \\[0.2cm]
      \hspace*{1.3cm}
      \mytt{t\_\textsl{name} = \textsl{regexp}}
      \\[0.2cm]
      where \textsl{name} is the name of the token and \textsl{regexp} is the regular expression defining the
      token.  We have used this shortcut to define most of the tokens recognized by our scanner.
\item While we do not need to declare tokens for those operators that consist of just one token, 
      we have to declare those tokens that consist of two or more characters.  For example, the operator
      \quoted{==} is represented by the token \mytt{'EQ'} and defined in line 9.
\item Our programming language supports single line comments that start with the string \quoted{//} and extend
      to the end of the line.  These comments are implemented via the token \mytt{COMMENT} that is defined in
      line 16--18.  Note that the function \mytt{t\_COMMENT} does not return a value.  Therefore,
      these comments are simply discarded.  This is also the reason that we have to use a function to define
      this token.
\item The most interesting aspect is the implementation of keywords like \quoted{while} or \quoted{return}.
      Note that we have defined separate tokens for all of the keywords but we have not defined any
      of these tokens.  For example, we have not defined \mytt{t\_return}.
      The reason is that, syntactically, all of the keywords are identifiers and are recognized by the regular
      expression
      \\[0.2cm]
      \hspace*{1.3cm}
      \mytt{r'[a-zA-Z][a-zA-Z0-9\_]*'}
      \\[0.2cm]
      that is used for recognizing identifiers in line 28.  However, the function \mytt{t\_ID} that recognizes
      identifiers does not immediately return the token \mytt{'ID'} when it has scanned the name of an
      identifier.  Rather, it first checks whether this name is a predefined keyword.  The predefined keywords
      are the key of the dictionary \mytt{Keywords} that is defined in line 38-43.  The values of the
      keywords in this dictionary are the token types.  Therefore, the function \mytt{t\_ID} sets the
      \blue{token type} of the token that is returned to the value stored in the dictionary \mytt{Keywords}.
      In case a name is not defined in this dictionary, the token type is set to \mytt{'ID'}.
\item The scanner implements the function \mytt{t\_newline} in line 6.  This function is needed
      to update the attribute \mytt{lineno} of the scanner.  This attribute stores the line number and is
      needed for error messages.
\end{enumerate}

\begin{figure}[!ht]
\centering
\begin{Verbatim}[ frame         = lines, 
                  framesep      = 0.3cm, 
                  labelposition = bottomline,
                  numbers       = left,
                  numbersep     = -0.2cm,
                  xleftmargin   = 0.8cm,
                  xrightmargin  = 0.8cm,
                ]
    import ply.yacc as yacc
    
    start = 'program'
    
    precedence = (
        ('left', 'OR'),
        ('left', 'AND'),
        ('left', '!'),
        ('nonassoc', 'EQ', 'NE', 'LE', 'GE', '<', '>'),
        ('left', '+', '-'),
        ('left', '*', '/')
    )
    
    def p_program_one(p):
        "program : function"
        p[0] = ('program', p[1])
        
    def p_program_more(p):
        "program : function program"
        p[0] = ('program', p[1]) + p[2][1:]
    
    def p_function(p):
        "function : INT ID '(' param_list ')' '{' decl_list stmnt_list '}'"
        p[0] = ('fct', p[2], p[4], p[7], p[8])
\end{Verbatim}
\vspace*{-0.3cm}
\caption{Grammar for Integer-\texttt{C}, part 1.}
\label{fig:Compiler.ipynb-1}
\end{figure}

Now we are ready to present the specification of our parser.  For reasons of space,
we have split the grammar into six parts.  We start with Figure \ref{fig:Compiler.ipynb-1} on page
\pageref{fig:Compiler.ipynb-1}.  
\begin{enumerate}
\item Line 3 declares the start symbol \mytt{program}.
\item Line 5--12 declare the operator precedences.
  \begin{enumerate}[(a)]
  \item The operator ``\mytt{||}'' that represents \blue{logical or} has the lowest precedence and associates
        to the left.  
  \item The operator ``\mytt{\&\&}'' that represents \blue{logical and} binds stronger than ``\mytt{||}''
        but not as strong as  the negation operator ``\mytt{!}''.
  \item The negation operator ``\mytt{!}'' is right associative because we want to interpret an expression of
        the form $\mytt{!!}a$ as  $\mytt{!}(\mytt{!}a)$.
  \item The comparison operators ``\mytt{==}'', ``\mytt{!=}'', ``\mytt{<=}'', ``\mytt{>=}'',
        ``\mytt{<}'', and ``\mytt{>}'' are non-associative, since our programming languages disallows 
        the chaining of comparison operators, i.e.~an expression of the form $x < y < z$ is syntactically
        invalid.
  \item The arithmetical operators ``\mytt{+}'' and ``\mytt{-}'' have a lower precedence than the
        arithmetical operators ``\mytt{*}'' and ``\mytt{/}''. 
  \end{enumerate}
\item A \mytt{program} is a non-empty list of \mytt{function} definitions.  Therefore, it is either a single
      function definition (line 14--16) or it is a function definition followed by more function definitions
      (line 18--20).

      The purpose of the parser is to turn the given program into an abstract syntax tree.  This abstract
      syntax tree is represented as a \blue{nested tuple}.  A program consisting of the functions 
      $f_1$, $\cdots$, $f_n$ is represented as the nested tuple.
      \\[0.2cm]
      \hspace*{1.3cm}
      $(\squoted{program}, f_1, \cdots, f_n)$.
      \\[0.2cm] 
      If in line 19 a \mytt{program} consisting of a \mytt{function} and another \mytt{program} is
      parsed, the expression \mytt{p[2]} in line 20 refers to the abstract syntax tree representing the
      \mytt{program} that follows the \mytt{function}.  The expression \mytt{p[2][1:]} discards the
      keyword \squoted{program} that is at the start of this nested tuple.  Hence,  \mytt{p[2][1:]}
      is just a tuple of the functions following the first \mytt{function}.  Therefore, the assignment to \mytt{p[0]}
      creates a nested tuple that starts with the keyword \squoted{program} followed by all the functions
      defined in this program.
\item A \mytt{function} definition starts with the keyword \squoted{int} that is followed by the name of the
      function, an opening parenthesis, a list of the parameters of the function, a closing parenthesis,
      an opening brace, a list of declarations, a list of statements, and finally a closing brace.
\end{enumerate}

\begin{figure}[!ht]
\centering
\begin{Verbatim}[ frame         = lines, 
                  framesep      = 0.3cm, 
                  labelposition = bottomline,
                  firstnumber   = last,
                  numbers       = left,
                  numbersep     = -0.2cm,
                  xleftmargin   = 0.8cm,
                  xrightmargin  = 0.8cm,
                ]
    def p_param_list_empty(p):
        "param_list :"
        p[0] = ('.', )
        
    def p_param_list_one(p):
        "param_list : INT ID"
        p[0] = ('.', p[2])
        
    def p_param_list_more(p):
        "param_list : INT ID ',' ne_param_list"
        p[0] = ('.', p[2]) + p[4][1:]
    
    def p_ne_param_list_one(p):
        "ne_param_list : INT ID"
        p[0] = ('.', p[2])
        
    def p_ne_param_list_more(p):
        "ne_param_list : INT ID ',' ne_param_list"
        p[0] = ('.', p[2]) + p[4][1:]
    
    def p_decl_list_one(p):
        "decl_list :"
        p[0] = ('.',)
    
    def p_decl_list_more(p):
        "decl_list : INT ID ';' decl_list"
        p[0] = ('.', p[2]) + p[4][1:]
    
    def p_stmnt_list_one(p):
        "stmnt_list :"
        p[0] = ('.',)
    
    def p_stmnt_list_more(p):
        "stmnt_list : stmnt stmnt_list"
        p[0] = ('.', p[1]) + p[2][1:]
\end{Verbatim}
\vspace*{-0.3cm}
\caption{ Grammar for Integer-\texttt{C}, part 2.}
\label{fig:Compiler.ipynb-2}
\end{figure}

Figure \ref{fig:Compiler.ipynb-2} shows the definition of parameter lists, declaration lists, and statement lists.
\begin{enumerate}
\item The definition of a list of parameters is quite involved, because parameter lists may be empty and,
      furthermore, different parameters have to be separated by \squoted{,}.  Therefore, we have to introduce 
      the additional syntactical variable \mytt{ne\_param\_list}, which represents a non-empty parameter
      list.  The grammar rules for \mytt{param\_list} are as follows:
      \begin{verbatim}
        param_list 
             :
             | 'int' ID
             | 'int' ID ',' ne_param_list
             ;
        ne_param_list 
             : 'int' ID
             | 'int' ID ',' ne_param_list
             ;
      \end{verbatim}
      We use the \quoted{.} as a key to represent lists of any kind as nested tuples.
\item The grammar rules for a list of declarations are as follows:
      \begin{verbatim}
      decl_list
          : 
          | 'int' ID ';' decl_list
          ;
      \end{verbatim}
      Observe that with this definition a list of declarations may be empty.
      The grammar rules are simpler than for parameter lists because every variable declaration is ended 
      with a semicolon, while parameters are separated by commas and the last parameter must not be followed by
      a comma.
\item The grammar rules for a list of statements are as follows:
      \begin{verbatim}
      stmnt_list
          : 
          | stmnt stmnt_list
          ;
      \end{verbatim}
\end{enumerate}
\begin{figure}[!ht]
\centering
\begin{Verbatim}[ frame         = lines, 
                  framesep      = 0.3cm, 
                  labelposition = bottomline,
                  numbers       = left,
                  firstnumber   = last,
                  numbersep     = -0.2cm,
                  xleftmargin   = 0.8cm,
                  xrightmargin  = 0.8cm,
                ]
    def p_stmnt_block(p):
        "stmnt : '{' stmnt_list '}'"
        p[0] = p[2]
        
    def p_stmnt_assign(p):
        "stmnt : ID '=' expr ';'"
        p[0] = ('=', p[1], p[3])
        
    def p_stmnt_if(p):
        "stmnt : IF '(' bool_expr ')' stmnt"
        p[0] = ('if', p[3], p[5])   
        
    def p_stmnt_if_else(p):
        "stmnt : IF '(' bool_expr ')' stmnt ELSE stmnt"
        p[0] = ('if-else', p[3], p[5], p[7])
        
    def p_stmnt_while(p):
        "stmnt : WHILE '(' bool_expr ')' stmnt"
        p[0] = ('while', p[3], p[5])
        
    def p_stmnt_return(p):
        "stmnt : RETURN expr ';'"
        p[0] = ('return', p[2])
        
    def p_stmnt_expr(p):
        "stmnt : expr ';'"
        p[0] = p[1]
\end{Verbatim}
\vspace*{-0.3cm}
\caption{Grammar for Integer-\texttt{C}, part 3.}
\label{fig:Compiler.ipynb-3}
\end{figure}

\noindent
Figure \ref{fig:Compiler.ipynb-3} on page \pageref{fig:Compiler.ipynb-3} shows how the variable \mytt{stmnt} is
defined.  The grammar rules for statements are as follows:
\begin{verbatim}
        stmnt : '{' stmnt_list '}'
              | ID '=' expr ';'
              | 'if' '(' bool_expr ')' stmnt
              | 'if' '(' bool_expr ')' stmnt 'else' stmnt
              | 'while' '(' bool_expr ')' stmnt
              | 'return' expr ';'
              | expr ';'
              ;
\end{verbatim}

\begin{figure}[!ht]
\centering
\begin{Verbatim}[ frame         = lines, 
                  framesep      = 0.3cm, 
                  labelposition = bottomline,
                  numbers       = left,
                  firstnumber   = last,
                  numbersep     = -0.2cm,
                  xleftmargin   = 0.8cm,
                  xrightmargin  = 0.8cm,
                ]
    def p_bool_expr_or(p):
        "bool_expr : bool_expr OR bool_expr"
        p[0] = ('||', p[1], p[3])
        
    def p_bool_expr_and(p):
        "bool_expr : bool_expr AND bool_expr"
        p[0] = ('&&', p[1], p[3])
    
    def p_bool_expr_neg(p):
        "bool_expr : '!' bool_expr"
        p[0] = ('!', p[2])

    def p_bool_expr_paren(p):
        "bool_expr : '(' bool_expr ')'"
        p[0] = p[2]

    def p_bool_expr_eq(p):
        "bool_expr : expr EQ expr"
        p[0] = ('==', p[1], p[3])
    
    def p_bool_expr_ne(p):
        "bool_expr : expr NE expr"
        p[0] = ('!=', p[1], p[3])
    
    def p_bool_expr_le(p):
        "bool_expr : expr LE expr"
        p[0] = ('<=', p[1], p[3])
        
    def p_bool_expr_ge(p):
        "bool_expr : expr GE expr"
        p[0] = ('>=', p[1], p[3])
        
    def p_bool_expr_lt(p):
        "bool_expr : expr '<' expr"
        p[0] = ('<', p[1], p[3])
    
    def p_bool_expr_gt(p):
        "bool_expr : expr '>' expr"
        p[0] = ('>', p[1], p[3])
\end{Verbatim}
\vspace*{-0.3cm}
\caption{Grammar for Integer-\texttt{C}, part 4.}
\label{fig:Compiler.ipynb-4}
\end{figure}

\noindent
Figure \ref{fig:Compiler.ipynb-4} on page \pageref{fig:Compiler.ipynb-4} shows how Boolean expressions are
defined.  The grammar rules are as follows:
\begin{verbatim}
        bool_expr : bool_expr '||' bool_expr
                  | bool_expr '&&' bool_expr
                  | '!' bool_expr
                  | '(' bool_expr ')'
                  | expr '==' expr
                  | expr '!=' expr
                  | expr '<=' expr
                  | expr '>=' expr
                  | expr '<'  expr
                  | expr '>'  expr
                  ;
\end{verbatim}

\begin{figure}[!ht]
\centering
\begin{Verbatim}[ frame         = lines, 
                  framesep      = 0.3cm, 
                  labelposition = bottomline,
                  numbers       = left,
                  firstnumber   = last,
                  numbersep     = -0.2cm,
                  xleftmargin   = 0.8cm,
                  xrightmargin  = 0.8cm,
                ]
    def p_expr_plus(p):
        "expr : expr '+' expr"
        p[0] = ('+', p[1], p[3])
        
    def p_expr_minus(p):
        "expr : expr '-' expr"
        p[0] = ('-', p[1], p[3])
        
    def p_expr_times(p):
        "expr : expr '*' expr"
        p[0] = ('*', p[1], p[3])
        
    def p_expr_divide(p):
        "expr : expr '/' expr"
        p[0] = ('/', p[1], p[3])
        
    def p_expr_group(p):
        "expr : '(' expr ')'"
        p[0] = p[2]
    
    def p_expr_number(p):
        "expr : NUMBER"
        p[0] = ('Number', p[1])
    
    def p_expr_id(p):
        "expr : ID"
        p[0] = p[1]
        
    def p_expr_fct_call(p):
        "expr : ID '(' expr_list ')'"
        p[0] = ('call', p[1]) + p[3][1:]
\end{Verbatim}
\vspace*{-0.3cm}
\caption{Grammar for Integer-\texttt{C}, part 5.}
\label{fig:Compiler.ipynb-5}
\end{figure}
\noindent
Figure \ref{fig:Compiler.ipynb-5} on page \pageref{fig:Compiler.ipynb-5} shows lists of expression how arithmetic expressions
are defined.  The grammar rules are as follows:
\begin{verbatim}
        expr : expr '+' expr
             | expr '-' expr
             | expr '*' expr
             | expr '/' expr
             | '(' expr ')'
             | NUMBER
             | ID
             | ID '(' expr_list ')'
\end{verbatim}

\noindent
Finally, Figure \ref{fig:Compiler.ipynb-6} on page \pageref{fig:Compiler.ipynb-6} shows how \mytt{expr\_list}
is defined.  
\begin{verbatim}
        expr_list :
                  | expr
                  | expr ',' ne_expr_list
                  ;

        ne_expr_list : expr
                     | expr ',' ne_expr_list
                     ;
\end{verbatim}
Furthermore, this function shows the definition of the function \mytt{p\_error}, which is called if
\textsc{Ply} detects a syntax error.  \textsc{Ply} will detect a syntax error in the case that the state of the
generated shift/reduce parser has no action for the next input token.  The argument \mytt{p} to the function
\mytt{p\_error} is the first token for which the action of the shift/reduce parser is undefined.
In this case \mytt{p.value} is the part of the input string corresponding to this token.

\begin{figure}[!ht]
\centering
\begin{Verbatim}[ frame         = lines, 
                  framesep      = 0.3cm, 
                  labelposition = bottomline,
                  numbers       = left,
                  firstnumber   = last,
                  numbersep     = -0.2cm,
                  xleftmargin   = 0.8cm,
                  xrightmargin  = 0.8cm,
                ]
    def p_expr_list_empty(p):
        "expr_list :"
        p[0] = ('.',)
        
    def p_expr_list_one(p):
        "expr_list : expr"
        p[0] = ('.', p[1])     
    
    def p_expr_list_more(p):
        "expr_list : expr ',' ne_expr_list"
        p[0] = ('.', p[1]) + p[3][1:]     
    
    def p_ne_expr_list_one(p):
        "ne_expr_list : expr"
        p[0] = ('.', p[1]) 
        
    def p_ne_expr_list_more(p):
        "ne_expr_list : expr ',' ne_expr_list"
        p[0] = ('.', p[1]) + p[3][1:] 
    
    def p_error(p):
        if p:
            print(f'Syntax error at token "{p.value}" in line {p.lineno}.')
        else:
            print('Syntax error at end of input.')
    
    parser = yacc.yacc(write_tables=False, debug=True)
\end{Verbatim}
\vspace*{-0.3cm}
\caption{Grammar for Integer-\texttt{C}, part 6.}
\label{fig:Compiler.ipynb-6}
\end{figure}
\pagebreak

\exercise
Extend the parser that is available at
\\[0.2cm]
\hspace*{1.3cm}
\href{
 https://github.com/karlstroetmann/Formal-Languages/blob/master/Ply/Compiler.ipynb}{https://github.com/karlstroetmann/Formal-Languages/blob/master/Ply/Compiler.ipynb} 
\\[0.2cm]
so that it supports the ternary \texttt{C}-operator for conditional expressions.  For example, in \texttt{C},
the following assignment using the ternary operator can be used the assign \mytt{max} the maximum of the values
of \mytt{x} and \mytt{y}:
\\[0.2cm]
\hspace*{1.3cm}
\mytt{max = x > y ? x : y;}
\\[0.2cm]
Extend the parser so it can process the assignment shown above. \eox 






%%% Local Variables: 
%%% mode: latex
%%% TeX-master: "formal-languages"
%%% End: 
