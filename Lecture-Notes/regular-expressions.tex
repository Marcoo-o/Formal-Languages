\chapter{Regular Expressions \label{chapter:regular-expressions}}
\href{http://en.wikipedia.org/wiki/Regular_expression}{\emph{Regular expressions}} are terms that
specify those formal languages that are simple enough to be recognized by a so called \blue{finite state machine}.
The concept of finite state machines will be discussed in chapter \ref{chapter:finit-state-machines.tex}.
A regular expression is able to specify 
\begin{enumerate}
\item the choice between different alternatives,
\item concatenation, and
\item repetition.
\end{enumerate}
Many modern scripting languages are based on regular expression, for example
the initial popularity of the programming language \href{http://en.wikipedia.org/wiki/Perl}{\textsl{Perl}} was
largely due to its efficiency in dealing with regular expressions.
Today, all modern high-level languages, e.g.~\textsl{Python}, \textsl{Java}, \texttt{C\#}, and many others
provide extensive libraries to support regular expressions.  Furthermore, there are a number of 
\textsc{Unix} tools like \href{http://en.wikipedia.org/wiki/Grep}{\texttt{grep}}, 
\href{http://en.wikipedia.org/wiki/Sed}{\texttt{sed}} or
\href{http://en.wikipedia.org/wiki/Awk}{\texttt{awk}} that are based on regular expressions.  Hence,
every aspiring computer scientist needs to be comfortable with regular expressions.  In this chapter
we will give a definition of regular expressions that is quite concise and is different from the definition
given in most programming languages.  The advantage of this concise definition is that it is more convenient
for our theoretical analysis of regular languages, which is given in Chapter 4 and Chapter 5.  In the next chapter
we will present the syntax of regular expressions that is used in \textsl{Python}.

\section{Preliminary Definitions}
Before we can define the syntax and semantics of regular expressions, we need some auxiliary definitions. 

\begin{Definition}[Product of Languages]
  If $\Sigma$ is an  alphabet and $L_1 \subseteq \Sigma^*$ and $L_2 \subseteq \Sigma^*$ are formal
  languages, the \blue{product} \index{product} of $L_1$ and $L_2$ is written as
  $L_1 \cdot L_2$ \index{$L_1 \cdot L_2$} and is defined as the set of all concatenations $w_1 \cdot w_2$ such that $w_1 \in L_1$ and
  $w_2 \in L_2$, i.e.~we have
  \\[0.2cm]
  \hspace*{1.3cm}
  $L_1 \cdot L_2 := \bigl\{ w_1 \cdot w_2 \mid w_1 \in L_1 \wedge w_2 \in L_2 \bigr\}$. \eox
\end{Definition}

\exampleEng
If $\Sigma = \{ \texttt{a}, \texttt{b}, \texttt{c} \}$ and $L_1$ and $L_2$ are defined as
\\[0.2cm]
\hspace*{1.3cm}
$L_1 = \{ \texttt{ab}, \texttt{bc} \}$ \quad and \quad
$L_2 = \{ \texttt{ac}, \texttt{cb} \}$,
\\[0.2cm]
then the product of $L_1$ and $L_2$ is given as
\\[0.2cm]
\hspace*{1.3cm}
$L_1 \cdot L_2 = \{ \texttt{abac}, \texttt{abcb}, \texttt{bcac}, \texttt{bccb} \}$.  \eox

\begin{Definition}[Power of a Language] 
Assume $\Sigma$ is an  alphabet, $L \subseteq \Sigma^*$ is a formal language and $n\in\mathbb{N}$.
The  \blue{$n$-th power} of $L$ \index{$n$-th power of a language} is written as  $L^n$ \index{$L^n$} and is
defined by induction on  $n$.  
\begin{enumerate}
\item[B.C.:] $n = 0$: 

      $L^0 := \{ \varepsilon \}$.

      Here $\varepsilon$ denotes the empty string, i.e.~we have $\varepsilon = \texttt{\symbol{34}\symbol{34}}$.

\item[I.S.:] $n \mapsto n + 1$:

      $L^{n+1} = L^n \cdot L$  \eox
\end{enumerate}
\end{Definition}

\exampleEng
If $\Sigma = \{ \texttt{a}, \texttt{b} \}$ and $L = \{ \texttt{ab}, \texttt{ba} \}$, we have
\begin{enumerate}
\item $L^0 = \{ \varepsilon \}$,
\item $L^1 = \{ \varepsilon \} \cdot \{ \texttt{ab}, \texttt{ba} \} = \{ \texttt{ab}, \texttt{ba} \}$,
\item $L^2 = \{ \texttt{ab}, \texttt{ba} \} \cdot \{ \texttt{ab}, \texttt{ba} \} = 
             \{ \texttt{abab}, \texttt{abba}, \texttt{baab}, \texttt{baba} \}$.  \eox
\end{enumerate}

\begin{Definition}[Kleene Closure]
  Assume that $\Sigma$ is an Alphabet and $L \subseteq \Sigma^*$ is some formal language. Then the
  \blue{Kleene closure} \index{Kleene closure} of $L$ is written as $L^*$ \index{$L^*$} and is defined to be
  the union of all powers $L^n$ for all $n \in \mathbb{N}$: 
  \\[0.2cm]
  \hspace*{1.3cm}
  $L^* := \bigcup\limits_{n \in \mathbb{N}} L^n = L^0 \cup L^1 \cup L^2 \cup L^3 \cup \cdots$.
  \\[0.2cm]
  Note that $\varepsilon \in L^*$.  Therefore, $L^*$ is never the empty set, not even if $L = \{\}$.
  \qed  
\end{Definition}

\exampleEng
Assume $\Sigma = \{ \texttt{a}, \texttt{b} \}$ and  $L = \{ \texttt{a} \}$.  Then we have
\\[0.2cm]
\hspace*{1.3cm}
$L^* = \{ \texttt{a}^n \mid n \in \mathbb{N} \}$.
\\[0.2cm]
Here $\texttt{a}^n$ is the string of length $n$ that contains only the letter \texttt{a}.  Hence, we have 
\\[0.2cm]
\hspace*{1.3cm}
$\texttt{a}^n = \underbrace{\texttt{a} \cdots \texttt{a}}_n$.  \eox 


Formally, given a string $s$ and an non-negative integer Zahl $n \in \mathbb{N}$, we define the expression
 $s^n$ by induction on  $n$:
\begin{enumerate}
\item[B.C.:] $n = 0$

             $s^0 := \varepsilon$. 
\item[I.S.:] $n \mapsto n + 1$

             $s^{n+1} := s^n\cdot s$, \quad where $s^n \cdot s$ denotes the concatenation of the strings $s^n$ and $s$.
             \eox
\end{enumerate}

The previous example shows that the Kleene closure of a finite language can be infinite.
It is easy to see that the Kleene closure of a language $L$ is infinite
if $L$ contains at least one string $s$ such that $|s| > 0$.
\vspace*{0.3cm}

\section{The Formal Definition of Regular Expressions}
We proceed to define the set of regular expressions given an  alphabet $\Sigma$.  This set is
denoted as $\texttt{RegExp}_\Sigma$  \index{$\texttt{RegExp}_\Sigma$}
and is defined by induction.  Simultaneously,  we define
the function
\\[0.2cm]
\hspace*{1.3cm}
$L: \texttt{RegExp}_\Sigma \rightarrow 2^{\Sigma^*}$,
\\[0.2cm]
which interprets every regular expression  $r$ as a formal language $L(r) \subseteq \Sigma^*$.\footnote{
  Given a set $M$ the \blue{power set} \index{power set} of $M$, i.e.~the set of all subsets of $M$, is denoted
  as $2^M$. \index{$2^M$} 
}

\begin{Definition}[Regular Expressions]
  The set $\texttt{RegExp}_\Sigma$ of \blue{regular expressions} \index{regular expression, definition} on the alphabet  $\Sigma$ is defined
  by induction as follows:
  \begin{enumerate}
  \item $\emptyset \in \texttt{RegExp}_\Sigma$ \index{$\emptyset$}

        The regular expression $\emptyset$ denotes the empty language, we have
        \\[0.2cm]
        \hspace*{1.3cm}
        $L(\emptyset) := \{\}$.

        In order to avoid confusion we assume that the symbol $\emptyset$ is \underline{not} a member of the
        alphabet $\Sigma$, i.e.~we have $\emptyset \not\in \Sigma$.
  \item $\varepsilon \in \texttt{RegExp}_\Sigma$ \index{$\varepsilon$}

        The regular expression $\varepsilon$ denotes the language that only contains the empty
        string $\varepsilon$: 
        \\[0.2cm]
        \hspace*{1.3cm}
        $L(\varepsilon) := \{ \varepsilon \}$.
        \\[0.2cm]
        Observe that in this equation the two occurrences of $\varepsilon$ are interpreted differently:
        The occurrence of $\varepsilon$ on the left hand side of this equation denotes a regular
        expression, while the occurrence of $\varepsilon$ on the right hand side denotes the empty
        string.

        Furthermore, we assume that $\varepsilon \not\in \Sigma$.
  \item $c \in \Sigma \rightarrow c \in \texttt{RegExp}_\Sigma$.

        Every character from the alphabet $\Sigma$ is a regular expression.  This expression denotes
        the language that contains only the string $c$:
        \\[0.2cm]
        \hspace*{1.3cm}
        $L(c) := \{ c \}$.
        \\[0.2cm]
        Observe that we identify characters with strings of length one.
  \item $r_1 \in \texttt{RegExp}_\Sigma \wedge r_2 \in \texttt{RegExp}_\Sigma
         \rightarrow r_1 + r_2 \in \texttt{RegExp}_\Sigma$

        Starting from two regular expressions $r_1$ and $r_2$ we can use the  infix operator
        ``$+$'' to build a new regular expression.  This regular expression denotes the union of 
        the languages described by $r_1$ and $r_2$:
        \\[0.2cm]
        \hspace*{1.3cm}
        $L(r_1 + r_2) := L(r_1) \cup L(r_2)$.

        In order to avoid confusion we have to assume that the symbol  ``\texttt{+}'' does not occur
        in the alphabet $\Sigma$, i.e.~we have  $\squoted{+} \not\in \Sigma$.
  \item $r_1 \in \texttt{RegExp}_\Sigma \wedge r_2 \in \texttt{RegExp}_\Sigma 
         \rightarrow r_1 \cdot r_2 \in \texttt{RegExp}_\Sigma$

        Starting from the regular expression $r_1$ and $r_2$ we can use the  infix operator
        ``$\cdot$'' to build a new regular expression.  This regular expression denotes
        the product of the languages of $r_1$ and $r_2$:
        \\[0.2cm]
        \hspace*{1.3cm}
        $L(r_1 \cdot r_2) := L(r_1) \cdot L(r_2)$.

        Again, in order to avoid confusion we have to assume that the symbol ``$\cdot$'' does not
        occur in the alphabet $\Sigma$, i.e.~we have $\squoted{$\cdot$} \not\in \Sigma$.
  \item $r \in \texttt{RegExp}_\Sigma \rightarrow r^* \in \texttt{RegExp}_\Sigma$

        Given a regular expression $r$, the postfix operator
        ``$^*$'' can be used to create a new regular expression.  This new regular expression
        denotes the Kleene closure of the language described by  $r$:
        \\[0.2cm]
        \hspace*{1.3cm}
        $L(r^*) := \bigl(L(r)\bigr)^*$.

        We have to assume that $\squoted{$^*$} \not\in \Sigma$. 
  \item $r \in \texttt{RegExp}_\Sigma \rightarrow (r) \in \texttt{RegExp}_\Sigma$

        Regular expressions can be surrounded by parentheses.  This does not change the language
        denoted by the regular expression:
        \\[0.2cm]
        \hspace*{1.3cm}
        $L\bigl((r)\bigr) := L(r)$. 

        We have to assume that the parentheses  \qote{(} and \qote{)} do not occur
        in the alphabet $\Sigma$, i.e.~we have $\squoted{(} \not\in \Sigma$  and $\squoted{)} \not\in \Sigma$. \eox
  \end{enumerate}
\end{Definition}
Given the preceding definition it is not clear whether the regular expression
\\[0.2cm]
\hspace*{1.3cm}
$\mathtt{a}+\mathtt{b}\cdot\mathtt{c}$ 
\\[0.2cm]
is to be interpreted as 
\\[0.2cm]
\hspace*{1.3cm}
$(\mathtt{a}+\mathtt{b})\cdot\mathtt{c}$ \quad or \quad $\mathtt{a}+(\mathtt{b}\cdot\mathtt{c})$.
\\[0.2cm]
In order to ensure that regular expressions can be read unambiguously we have to assign \blue{operator precedences}:
\begin{enumerate}
\item The postfix operator ``$^*$'' has the highest precedence.
\item The precedence of the infix operator ``$\cdot$'' is lower than the precedence of  ``$^*$'' but
      stronger than the precedence of ``$+$''.
\item The operator ``$+$'' has the lowest precedence.
\end{enumerate}
Using these conventions, the regular expression 
\\[0.2cm]
\hspace*{1.3cm}
$\mathrm{a} + \mathrm{b} \cdot \mathrm{c}^*$ \quad is interpreted as  \quad
$\mathrm{a} + \bigl(\mathrm{b} \cdot (\mathrm{c}^*)\bigr)$.

\examplesEng
In the following examples, the alphabet  $\Sigma$ is defined as
\\[0.2cm]
\hspace*{1.3cm}
$\Sigma := \{ \texttt{a}, \texttt{b}, \texttt{c} \}$.
\begin{enumerate}
\item $r_1 := (\texttt{a} + \texttt{b} + \texttt{c}) \cdot (\texttt{a} + \texttt{b} + \texttt{c})$

      The expression  $r_1$ denotes the set of all strings that have the length $2$:
      \\[0.2cm]
      \hspace*{1.3cm}
      $L(r_1) = \bigl\{ w \in \Sigma^* \,\big|\; |w| = 2 \bigr\}$.
\item $r_2 := (\texttt{a} + \texttt{b} + \texttt{c}) \cdot (\texttt{a} + \texttt{b} + \texttt{c})^*$

      The expression  $r_2$ denotes the set of all strings that have at least the length $1$:
      \\[0.2cm]
      \hspace*{1.3cm}
      $L(r_2) = \bigl\{ w \in \Sigma^* \,\big|\; |w| \geq 1 \bigr\}$.
\item $r_3 := (\texttt{b} + \texttt{c})^* \cdot \texttt{a} \cdot 
              (\texttt{b} + \texttt{c})^*$

      The expression $r_3$ denotes the set of all those strings that have exactly one occurrence of
      the letter ``\texttt{a}''.  A string containing exactly one ``\texttt{a}''
      is a string that starts with an arbitrary amount of the letters \texttt{b} and \texttt{c} 
      (this is what $(\texttt{b} + \texttt{c})^*$ denotes), followed by the letter ``\texttt{a}'',
      followed by another substring containing only the letters \texttt{b} and \texttt{c}.
      \\[0.2cm]
      \hspace*{1.3cm}
      $L(r_3) = \Bigl\{ w \in \Sigma^* \;\Big|\;\; 
                        \#\bigl\{i \in  \mathbb{N} \,\big|\; w[i] = \texttt{a} \bigl\} \,= 1 \Bigr\}$.
      \\[0.2cm]
      Given a set $M$, the expression $\#M$ \index{$\#M$} denotes the number of elements of $M$.
\item $r_4 :=  (\texttt{b} + \texttt{c})^* \cdot \texttt{a} \cdot (\texttt{b} + \texttt{c})^* +
               (\texttt{a} + \texttt{c})^* \cdot \texttt{b} \cdot (\texttt{a} + \texttt{c})^*$

      The regular expression $r_4$ denotes the set of all those strings that either contain exactly
      one occurrence of the letter ``\texttt{a}'' or exactly one occurrence of the letter ``\texttt{b}''.
      \\[0.2cm]
      \hspace*{0.3cm}
      $L(r_4) = \Bigl\{ w \in \Sigma^* \;\Big|\;\; 
                        \#\bigl\{i \in \mathbb{N} \,\big|\; w[i] = \texttt{a} \bigl\} \,=
                        1 \Bigr\} \;\cup\;
                \Bigl\{ w \in \Sigma^* \;\Big|\;\; 
                        \#\bigl\{i \in \mathbb{N} \,\big|\; w[i] = \texttt{b} \bigl\} \,=
                        1 \Bigr\}$.  \eox 
\end{enumerate}

\remarkEng 
The syntax of regular expressions given here is the same as the syntax used in  \cite{hopcroft:06}.   However,
the syntax used for regular expression in programming languages 
like \textsl{Python} is different.  We will discuss these differences later.

\exerciseEng
\renewcommand{\labelenumi}{(\alph{enumi})}
\begin{enumerate}
\item Assume $\Sigma = \{ \mathtt{a}, \mathtt{b}, \mathtt{c} \}$.  Define a regular expression for the language
      $L \subseteq \Sigma^*$ that consists of those strings that contain at least one occurrence of
      the letter ``\texttt{a}'' and one occurrence of the letter ``\texttt{b}''.
\item Assume $\Sigma = \{ 0, 1 \}$.   Specify a regular expression for the language 
      $L \subseteq \Sigma^*$ that consists of those strings $s$ such that the
      \href{https://dictionary.cambridge.org/dictionary/english/antepenultimate}{antepenultimate} 
      character is the symbol  ``$1$''.
\item Again, we have $\Sigma = \{ 0, 1 \}$.   Define a regular expression for the language
      $L \subseteq \Sigma^*$ containing all those strings that do not contain the substring  $110$.

      \solutionEng
      The regular expression $r$ that is sought for can be defined as 
      \\[0.2cm]
      \hspace*{1.3cm}
      $r = (0 + 1 \cdot 0)^* \cdot 1^*$.
      \\[0.2cm]
      First, it is quite obvious that the language $L(r)$ does not contain a string $w$ such that
      $w$ contains the substring $110$.  This is so because a character $1$ that is generated by the
      part $(0 + 1 \cdot 0)^*$ is immediately followed by a $0$.  Hence if $w$ contains the
      substring $110$, the first $1$ cannot originate from the regular expression $(0 + 1 \cdot 0)^*$.
      Furthermore, if the first $1$ of the substring 110 originates from the regular expression
      $1^*$, then there cannot be a $0$ following since the language generated by $1^*$ contains
      only ones.

      Second, assume that the string $w$ does not contain the substring $110$.  We have to show that
      $w \in L(r)$.  Now if the character $1$ does not occur in the
      string $w$, then $w$ is just a bunch of zeros and therefore $w$ can be generated by the
      regular expression $(0+1\cdot 0)^*$ and hence also by $(0 + 1  \cdot 0)^* \cdot 1^*$.  If the string $w$ does contain the character $1$,
      there are two cases.
      \begin{enumerate}
      \item The first occurrence of $1$ is followed by a $0$.  Then the prefix of $w$ up to and
            including this $0$ is generated by the regular expression $(0 + 1 \cdot 0)^*$.  The
            remaining part of $w$ is shorter and, by induction, can be shown to be generated by 
            $(0 + 1 \cdot 0)^* \cdot 1^*$.
      \item The first occurrence of $1$ is followed by another $1$.  In this case, the rest of $w$
            must be made up of ones.  Hence, the part of $w$ starting with the first $1$ is
            generated by $1^*$ and obviously the preceding zeros can all be generated by 
            $(0 + 1 \cdot 0)^*$.
      \end{enumerate}
\item Again, assume $\Sigma = \{0,1\}$.  What is the language $L$ generated by the regular expression 
      \\[0.2cm]
      \hspace*{1.3cm}
      $(1 + \varepsilon)\cdot(0\cdot 0^* \cdot 1)^* \cdot 0^*$?  \eox

      \solutionEng
      This is the language $L$ such that the strings in $L$ do not contain the substring $11$.
\end{enumerate}
\renewcommand{\labelenumi}{\arabic{enumi}.}


\section{Algebraic Simplification of Regular Expressions}
Given two regular expressions $r_1$ and $r_2$, we write
\\[0.2cm]
\hspace*{1.3cm}
$r_1 \doteq r_2$ \quad iff \quad $L(r_1) = L(r_2)$,
\\[0.2cm]
i.e.~if $r_1$ and $r_2$ describe the same language.  If the equation $r_1 \doteq r_2$ \index{$r_1 \doteq r_2$} holds, then we
call $r_1$ and $r_2$ \blue{equivalent}. \index{equivalence of regular expressions}  The following algebraic laws apply:
\begin{enumerate}
\item $r_1 + r_2 \doteq r_2 + r_1$

      This equation is true because the union operator is commutative for sets:
      \\[0.2cm]
      \hspace*{1.3cm}
      $L(r_1 + r_2) = L(r_1) \cup L(r_2) = L(r_2) \cup L(r_1) = L(r_2 + r_1)$.
\item $(r_1 + r_2) + r_3 \doteq r_1 + (r_2 + r_3)$

      This equation is true because the union operator is associative for sets.
\item $(r_1 \cdot r_2) \cdot r_3 \doteq r_1 \cdot (r_2 \cdot r_3)$

      This equation is true because the concatenation of strings is associative, for any strings
      $u$, $v$, and $w$ we have
      \\[0.2cm]
      \hspace*{1.3cm}
      $(u v) w = u (v w)$.
      \\[0.2cm]
      This implies
      \\[0.2cm]
      \hspace*{1.3cm}
      $
      \begin{array}[t]{lcl}
        L\bigl( (r_1 \cdot r_2) \cdot r_3\bigr) 
        & = & \bigl\{ x w \mid x \in L(r_1 \cdot r_2) \wedge w \in L(r_3)\bigr) \\[0.1cm]
        & = & \bigl\{ (u v) w \mid u \in L(r_1) \wedge v \in L(r_2) \wedge w \in L(r_3)\bigr) \\[0.1cm]
        & = & \bigl\{ u (v w) \mid u \in L(r_1) \wedge v \in L(r_2) \wedge w \in L(r_3)\bigr) \\[0.1cm]
        & = & \bigl\{ u y \mid u \in L(r_1) \wedge y \in L(r_2 \cdot r_3)\bigr) \\[0.1cm]
        & = & L\bigl( r_1 \cdot (r_2 \cdot r_3)\bigr).
      \end{array}
      $

      The following equations are more or less obvious.
\item $\emptyset \cdot r \doteq r \cdot \emptyset \doteq \emptyset$
\item $\varepsilon \cdot r \doteq r \cdot \varepsilon \doteq r$
\item $\emptyset + r \doteq r + \emptyset \doteq r$
\item $(r_1 + r_2) \cdot r_3 \doteq r_1 \cdot r_3 + r_2 \cdot r_3$
\item $r_1 \cdot (r_2 + r_3) \doteq r_1 \cdot r_2 + r_1 \cdot r_3$
\item $r + r \doteq r$, because
      \\[0.2cm]
      \hspace*{1.3cm}
      $L(r+r) = L(r) \cup L(r) = L(r)$.
\item $(r^*)^* \doteq r^*$      

      We have
      \\[0.2cm]
      \hspace*{1.3cm}
      $L(r^*) = \bigcup\limits_{n \in \mathbb{N}} L(r)^n$ 
      \\[0.2cm]
      and that implies $L(r) \subseteq L(r^*)$.   This holds true if we replace  $r$ by  $r^*$. Therefore
      \\[0.2cm]
      \hspace*{1.3cm}
      $L(r^*) \subseteq L\bigl((r^*)^*\bigr)$
      \\[0.2cm]
      holds.  In order to prove the inclusion
      \\[0.2cm]
      \hspace*{1.3cm}
      $L\bigl((r^*)^*\bigr) \subseteq L(r^*)$,
      \\[0.2cm]
      we consider the structure of the strings  $w \in L\bigl((r^*)^*\bigr)$.
      Because of
      \\[0.2cm]
      \hspace*{1.3cm}
      $L\bigl((r^*)^*\bigr) = \bigcup\limits_{n \in \mathbb{N}} L(r^*)^n$
      \\[0.2cm]
      we have $w \in L\bigl((r^*)^*\bigr)$ if and only if there is an $n \in \mathbb{N}$
      such that there are strings $u_1, \cdots,u_n \in L(r^*)$ satisfying
      \\[0.2cm]
      \hspace*{1.3cm}
      $w = u_1 \cdots u_n$.
      \\[0.2cm]
      Because of $u_i \in L(r^*)$ we find a number $m(i) \in \mathbb{N}$ for every $i \in \{1,\cdots,n\}$ 
      such that for  $j=1,\cdots, m(i)$ there are strings $v_{i,j} \in L(r)$ satisfying
      \\[0.2cm]
      \hspace*{1.3cm}
      $u_i = v_{1,i} \cdots v_{m(i),i}$.
      \\[0.2cm]
      Combining these equations yields
      \\[0.2cm]
      \hspace*{1.3cm}
      $w = v_{1,1} \cdots v_{m(1),1} v_{1,2} \cdots v_{m(2),2} \cdots v_{1,n} \cdots v_{m(n),n}$.
      \\[0.2cm]
      Hence $w$ is a concatenation of strings from the language $L(r)$ and hence we have
      \\[0.2cm]
      \hspace*{1.3cm}
      $w \in L\bigl(r^*\bigr)$.
      \\[0.2cm]
      This shows the inclusion
      $L\bigl((r^*)^*\bigr) \subseteq L(r^*)$.
\item $\emptyset^* \doteq \varepsilon$
\item $\varepsilon^* \doteq \varepsilon$
\item $r^* \doteq \varepsilon + r^* \cdot r$
\item $r^* \doteq (\varepsilon + r)^*$
\end{enumerate}


\section{Check your Understanding}
\begin{enumerate}[(a)]
\item Given a formal language $L$, what is the definition of $L^*$?
\item How is the set $\mathtt{RegExp}_\Sigma$ defined?
\item How is the function $L:\mathtt{RegExp}_\Sigma \rightarrow 2^{\Sigma^*}$ defined?
\item Given $r_1, r_2 \in \mathtt{RegExp}_\Sigma$, how is the notion $r_1 \doteq r_2$ defined?
\end{enumerate}



%%% Local Variables: 
%%% mode: latex
%%% TeX-master: "formal-languages.tex"
%%% End: 
